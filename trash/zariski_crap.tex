\section{The Zariski Topology}
\begin{dfn}[Zariski topology]
    In commutative algebra, 
    we have the result that for any ring $A$ there is a topology $\spec(A)$
    (the spectrum of $A$),
    the set of all prime ideals in $A$.
    This is done by generating specifying the closed sets, 
    namely for any $E \subs A$ the set 
    \[V(E) := \set{\f{p} \in \spec(A) \st E \subs \f{p}}\]
    the `vanishing' is closed.
    It can be shown that under finite union and arbitrary intersection of these
    sets are still closed thus it defines a topology on $\spec(A)$.
    Furthermore we have that for any $E \subs A$, 
    $V(E) = V(\<E\>)$ where the latter is the ideal generated by $E$.
\end{dfn}
We wish to translate all this into the classical setting, 
where $K$ be a field and $n \in \N$ 
and we want to construct a topology on $K^n$.
We take our ring $A$ to be $K[x_1,\dots,x_n]$.
We will show that the set of vanishings in $\spec(K[x_1,\dots,x_n])$
bijects with
the set of varieties of finite sets in $K[x_1, \dots, x_n]$.
Then we will take the varieties as the closed sets in $K[x_1,\dots,x_n]$.

\begin{dfn}[Ideal generated by varieties]
    For a subset $X \subs K^n$, 
    we write $I(X)$ to mean the ideal of $X$ in $K[x_1,\dots,x_n]$ to mean
    \[\set{f \in K[x_1,\dots, x_n] \st \forall a \in X, f(a) = 0}\]
\end{dfn}
\begin{prop}[Correspondence between vanishings and varieties]
    Given $K$ a field and a vanishing in the spectrum of 
    $K[x_1,\dots, x_n]$, there exists a unique variety of some finite 
    subset of $K[x_1,\dots,x_n]$.
    and vice versa.
\end{prop}
\begin{proof}
    Let $V(E) \in B$ then 
    $\<E\>$ is finitely generated by the 
    \linkto{hilbert_basis}{Hilbert basis theorem}
    so there exists some finite subset $S \subs K[x_1, \dots, x_n]$ such that
    \[V(E) = V(\<E\>) = V(\<S\>)\]
    We send $V(E)$ to $\V_K(S)$.
    This is well defined: suppose $V(\<S\>) = V(\<T\>)$ then
    for any prime ideal $\f{p}$, 
    $\<S\> \subs \f{p} \iff \<T\> \subs \f{p}$.
    Hence 
    \[
        r(S) = \bigcap_{\<S\> \subs \f{p} \text{ prime }} \f{p}
        = \bigcap_{\<T\> \subs \f{p} \text{ prime }} \f{p}
        = r(T)
    \]
    Let $a \in \V_K(S)$, then for any $f \in <S>$, $f(a) = 0$.
    To show that $a \in \V_K(T)$,
    let $f \in T$,
    then there exists some $n$ such that $f^n \in <S>$ as $r(S) = r(T)$.
    Hence $f^n(a) = 0$ and since $K[x_1,\dots,x_n]$ is an integral domain,
    by induction $f(a) = 0$.
    Thus $a \in \V_K(T)$.

    This map is clearly surjective.
    It is injective due to 
    \linkto{strong_nullstellensatz}{strong Nullstellensatz}:
    If $\V_K(S) = \V_K(T)$ then 
    $I(\V_K(S)) = I(\V_K(T))$ and so 
    \[r(S) = I(\V_K(S)) = I(\V_K(T)) = r(T)\]
    hence 
    \[  
        \<S\> \subs \f{p} \iff r(S) \subs \f{p} 
        \iff r(T) \subs \f{p} \iff \<T\> \subs \f{p}
    \]
    Thus $V(S) = V(T)$.
\end{proof}
