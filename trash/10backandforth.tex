\subsection{Back and Forth}
%% Refs to openlogicproject, Pillay and Bruno Poizat

`Back and forth' is a technique used to determine 
elementary equivalence of models,
quantifier elimination of theories
and completeness of theories.
This section draws its theory from \cite{poizat}, \cite{openlogicproject},
and \cite{pillay}.

\begin{dfn}[Substructure generated by a subset]
    Let $\MM$ be a $\Si$-structure.
    Let $A \subs \MM$.
    Then the following are equal:
    \begin{itemize}
        \item The set $\<A\>$ defined inductively: $A\subs \<A\>$;
        if $c \in \const{\Si}$ then 
        $\mmintp{c} \in \<A\>$; if $f \in \func{\Si}$ and 
        $\al \in \<A\>^{n_f}$ then $\mmintp{f}(\al) \in \<A\>$.
        \item $\bigcap \set{\NN \text{ substructure of } \MM \st A \subs \NN}$
    \end{itemize}
    and define a substructure of $\MM$.
    We say it is the `substructure of $\MM$ generated by $A$'.
    
    We say a substructure is finitely generated if there exists a finite set 
    $A$ such that it is equal to $\<A\>$.
\end{dfn}
\begin{proof}
    We note show that $\<A\>$ is a substructure of $\MM$ containing $A$:
    It contains the interpretations of constant symbols from $\MM$.
    By definition $\modintp{\<A\>}{f} := \mmintp{f}$ is well defined.
    Each relation $r$ is naturally interpreted as the intersection of relations
    on $\MM$ intersected with $\<A\>^{m_r}$.
    Hence $\bigcap \NN \subs \<A\>$.

    For the other direction note that if $a \in \<A\>$ then it is in $A$,
    $\mmintp{c}$
    or $\mmintp{f}(\al)$ for some $\al \in \<A\>^{n_f}$.
    If it is in $A$ then we are done.
    Any substructure of $\MM$ contains the $\mmintp{c}$ for each constant symbol
    hence the first case is fine.
    Any substructure of $\MM$ is closed under $\mmintp{f}$ and by induction
    $\al \in \NN^{n_f}$ for any substructure $\NN$.
    Hence $f(\al) \in \NN$ for any substructure.
    Thus $\<A\> \subs \bigcap \NN$ and we are done.
\end{proof}

\begin{prop}[Image of generators are generators of the image]
    \link{im_of_gen_are_gen_of_im_substructures}
    The image of a substructure generated by a subset is a substructure 
    generated by the image of a set.
    In particular,
    a finitely generated substructure has finitely generated image under a 
    $\Si$-morphism given by the image of the generators.
\end{prop}
\begin{proof}
    Let $\io : \<A\> \to \NN$ be a $\Si$-morphism.
    We show that $\<\io(A)\> = \io(\<A\>)$.
    If $b \in \<\io(A)\>$ then $b = \nnintp{c}$ or $b = \nnintp{f}(\io(\al))$ 
    for $al \in \<A\>$.
    Hence $b = \nnintp{c} = \io(\mmintp{c}) \in \io(\<A\>)$
    or 
    \[b = \nnintp{f}(\io(\al)) = \io(\mmintp{f}(\al)) \in \io(\<A\>)\]
    Thus $\<\io(A)\> \subs \io(\<A\>)$.
    The other direction is similar.
\end{proof}

\begin{dfn}[Partial isomorphisms]
    Let $\MM$ and $\NN$ be $\Si$-structures.
    A partial isomorphism from $\MM$ to $\NN$ is a $\Si$-isomorphism $p$
    with finitely generated domain in of $\MM$ 
    and codomain in $\NN$.
\end{dfn}

\begin{prop}[Equivalent definition of partial isomorphism]
    \link{equiv_def_partial_iso}
    Let $\MM$ and $\NN$ be $\Si$-structures. 
    Let $a \in \MM^n$ and $b \in \NN^n$.
    The following are equivalent:
    \begin{itemize}
        \item There exists a partial isomorphism $p : \<a\> \to \<b\>$ 
            such that $p(a) = b$.
        \item $\subintp{\nothing,n}{\MM}{\qftp}(a) = 
            \subintp{\nothing,n}{\NN}{\qftp}(b)$
    \end{itemize}
\end{prop}
\begin{proof}
    \begin{forward}
        We induct on terms to show that $\mmintp{t}(a) = \modintp{\<a\>}{t}(a)$
        for each term $t$:
        \begin{itemize}
            \item If $t$ is a constant symbol or a variable 
                then by definition of the 
                the substructure interpretation they are equal.
            \item If $t$ is $f(s)$ and we have the inductive hypothesis 
                $\mmintp{s}(a) = \modintp{\<a\>}{s}(a)$
                then by definition of the substructure interpretation
                \[
                    \mmintp{t}(a) = \mmintp{f}(\mmintp{s}(a))
                    = \mmintp{f}(\modintp{\<a\>}{s}(a))
                    = \modintp{\<a\>}{f}(\modintp{\<a\>}{s}(a))
                    = \modintp{\<a\>}{t}(a)
                \]
        \end{itemize}
    
    Let $\phi$ be a quantifier free $\Si$-formula with up to $n$ variables.
    We show by induction on $\phi$ that 
    \[
        \MM \model{\Si} \phi(a)
        \iff \<a\> \model{\Si} \phi(a)
    \]
    \begin{itemize}
        \item If $\phi$ is $\top$ it is trivial.
        \item If $\phi$ is $t = s$ then it is clear that
            \[
                \mmintp{t}(a) = \mmintp{s}(a)
                \iff \modintp{\<a\>}{t}(a) = \modintp{\<a\>}{s}(a)
            \]
            by what we showed for terms.
        \item If $\phi$ is $r(t)$ then 
            \[  
                (a_{i_1},\dots, a_{i_m}) \in \mmintp{r}
                \iff (a_{i_1},\dots, a_{i_m}) \in 
                \mmintp{r} \cap \<a\> = \modintp{\<a\>}{r}
            \]
        \item If $\phi$ is $\NOT \psi$ or $\psi \OR \chi$ then it is 
            clear by induction.
    \end{itemize}
    As $p$ is an $\Si$-isomorphism, for any quantifier free $\Si$-formula
    with up to $n$ variables,
    \[\MM \model{\Si} \phi(a) \iff \<a\> \model{\Si} \phi(a)
    \iff \<b\> \model{\Si} \phi(b)
    \NN \model{\Si} \phi(b)\]
    \end{forward}

    \begin{backward}
        Suppose $\subintp{\nothing,n}{\MM}{\qftp}(a) = 
        \subintp{\nothing,n}{\NN}{\qftp}(b)$.
        We define $p : \<a\> \to \NN$ by the following:
        if $\al \in \<a\>$ then one can write $\al$ as a term $t$ evaluated
        at $a$: $\al = \mmintp{t}(a)$; $p$ maps $a$ to $\nnintp{t}(b)$.
        To show that $p$ is well-defined, note that if two terms $t$ and $s$
        are such that $\mmintp{t}(a) = \mmintp{s}(a)$ then 
        $t = s$ is a formula in $\subintp{\nothing,n}{\MM}{\qftp}(a) = 
        \subintp{\nothing,n}{\NN}{\qftp}(b)$ and so 
        $\nnintp{t}(b) = \nnintp{s}(b)$.
        it is injective because if two terms $t$ and $s$
        are such that $\nnintp{t}(b) = \nnintp{s}(b)$ then 
        $t = s$ is a formula in $\subintp{\nothing,n}{\NN}{\qftp}(b) = 
        \subintp{\nothing,n}{\MM}{\qftp}(a)$ and so 
        $\mmintp{t}(a) = \mmintp{s}(a)$.

        By definition $p$ commutes with the interpretation of constant symbols,
        function symbols, and relations.
        Furthermore, for each $i$, $p(a_i) = b_i$ by taking the term to be a 
        variable and evaluating at $a_i$. 
        \linkto{im_of_gen_are_gen_of_im_substructures}{The 
            image of $p$ is $\<b\>$ as the image of $a$ is $b$}.
        Hence it is a partial isomorphism $\<a\> \to \<b\>$ 
        such that $p(a) = b$.
    \end{backward}
\end{proof}

\begin{prop}[Basic facts about partial isomorphisms]
    \link{basic_facts_partial_isomorphisms}
    Let $\MM$ and $\NN$ be $\Si$-structures.
    \begin{itemize}
        \item The inverse of a partial isomorphism is a partial isomorphism.
        \item The restriction of a partial isomorphism is a partial isomorphism.
        \item The composition of partial isomorphisms is a partial isomorphism.
    \end{itemize}
\end{prop}

\begin{dfn}[Partially isomorphic structures]
    Let $\MM$ and $\NN$ be $\Si$-structures.
    A partial isomorphism from $\MM$ to $\NN$ is said to have 
    the back and forth property if 
    \begin{itemize}
        \item (Forth) For each $a \in \MM$
            there exists a partial isomorphism $q$ such that 
            $q$ extends $p$ and $a \in \dom{p}$.
        \item (Back) For each $p \in I$ 
            there exists a partial isomorphism $q$ such that 
            $q$ extends $p$ and $b \in \codom{q}$.
    \end{itemize}

    We say $\MM$ and $\NN$ are back and forth equivalent 
    when all partial isomorphisms from $\MM$ to $\NN$ 
    have the back and forth property.
\end{dfn}

\begin{prop}[Equivalent definition of back and forth property]
    \link{equiv_def_back_and_forth}
    Let $\MM$ and $\NN$ be $\Si$-structures.
    Let $p : \<a\> \to \<b\>$ for $a \in \MM^n$ and $b \in \NN^n$
    be a partial isomorphism such that $p(a) = b$.
    It has the back and forth property if and only if the two conditions hold
    \begin{itemize}
        \item (Forth) For any $\al \in \MM$, 
            there exists $\be \in \NN$ such that 
            $\subintp{\nothing,n+1}{\MM}{\qftp}(a,\al) = 
            \subintp{\nothing,n+1}{\NN}{\qftp}(b,\be)$
        \item (Back) For any $\be \in \NN$, 
        there exists $\al \in \MM$ such that 
        $\subintp{\nothing,n+1}{\MM}{\qftp}(a,\al) = 
        \subintp{\nothing,n+1}{\NN}{\qftp}(b,\be)$
    \end{itemize}
\end{prop}
\begin{proof}
    \begin{forward}
        Suppose $p$ has the back and forth property.
        We only show `forth' as the `back' case is similar.
        Let $\al \in \MM$.
        By `forth' there exists $q$ 
        a partial isomorphism extending $p$ such that 
        $\al \in \dom(q)$.
        By \linkto{basic_facts_partial_isomorphisms}{restriction}
        and the fact that \linkto{image_of_generators}{the image of 
        generators generates the image},
        there exists $\be \in \NN$ such that 
        \[\res{q}{\<a,\al\> \to \<b,\be\>}\]
        is a local isomorphism.
        Using the \linkto{equiv_def_of_partial_iso}{the equivalent definition}
        we obtain ${\qftp}(a,\al) = {\qftp}(b,\be)$.
    \end{forward}

    \begin{backward}
        We show that $p$ has the `forth' property.
        Let $\al \in \MM$.
        By assumption there exists $\be \in \MM$ such that 
        \[\subintp{\nothing,n}{\MM}{\qftp}(a,\al) = 
        \subintp{\nothing,n}{\NN}{\qftp}(b,\be)\]
        Thus \linkto{equiv_def_of_partial_iso}{there exists 
        $q : \<a,\al\> \to \<b,\be\>$} such that $q(a) = b$ and $q(\al) = \be$.
        Hence $p$ is extended by $q$ with $\al$ in its domain.
    \end{backward}
\end{proof}

\begin{lem}[Back and forth equivalence implies quantifier elimination for types]
    \link{back_and_forth_gives_quantifier_elimination_lem}
    Let $\MM$ and $\NN$ be $\Si$-structures.
    If $\MM$ and $\NN$ are back and forth equivalent,
    $a \in \MM^n$ and $b \in \NN^n$ are such that
    \[\subintp{\nothing,n}{\MM}{\qftp}(a) = 
    \subintp{\nothing,n}{\NN}{\qftp}(b)\]
    then \[\subintp{\nothing,n}{\MM}{\tp}(a) = 
    \subintp{\nothing,n}{\NN}{\tp}(b)\]
\end{lem}
\begin{proof}
    Let $\phi \in F(\Si,n)$.
    If $\phi$ is quantifier free then 
    $\MM \modelsi \phi(a) \iff \NN \modelsi \phi(b)$.
    By induction on formulas it suffices to show that if 
    $\phi$ is the formula $\forall v, \psi$ and 
    for any $\al \in \MM$ there exists $\be \in \NN$ such that 
    $\MM \modelsi \psi(a,\al) \iff \NN \modelsi \psi(b,\be)$, 
    then we have 
    $\MM \modelsi \forall v, \psi(a) \iff \NN \modelsi \forall v, \psi(b)$.

    By the \linkto{equiv_def_partial_iso}{
        equivalent definition of partial isomorphisms,}
    there exists $p : \<a\> \to \<b\>$ 
    a partial isomorphism in $p$ such that $p(a) = b$.
    Suppose $\MM \modelsi \forall v, \psi(a)$ and let $\be \in \NN$, 
    then $\MM \modelsi \forall v, \psi(a,\al)$.
    By `back' in 
    \linkto{equiv_def_back_and_forth}{
        the equivalent definition of the back and forth property}
    there exists $\al \in \MM$ such that 
        $\subintp{\nothing,n+1}{\MM}{\qftp}(a,\al) = 
        \subintp{\nothing,n+1}{\NN}{\qftp}(b,\be)$
    Hence $\NN \modelsi \forall v, \psi(a,\al)$.
    The other direction is similar.
\end{proof}

\begin{cor}[Elementary equivalence]
    Let $\MM$ and $\NN$ be $\Si$-structures.
    If $\MM$ and $\NN$ are back and forth equivalent
    then they are elementarily equivalent.
\end{cor}
\begin{proof}
    Let $\phi$ be a quantifier free $\Si$-formula with $0$ variables,
    i.e. a quantifier free sentence.
    As the empty set is a partial isomorphism.
    Thus by the \linkto{equiv_def_partial_iso}{equivalent 
        definition of a partial isomorphism,}
    \[\subintp{\nothing,0}{\MM}{\qftp}(\nothing) = 
    \subintp{\nothing,0}{\NN}{\qftp}(\nothing)\]
    
    Hence 
    \[\subintp{\nothing,0}{\MM}{\tp}(\nothing) = 
    \subintp{\nothing,0}{\NN}{\tp}(\nothing)\]
    by the fact that 
    \linkto{back_and_forth_gives_quantifier_elimination_lem}{back 
        and forth equivalence implies quantifier elimination for types}.
    Thus for any $\Si$-sentence $\phi$, $\MM \modelsi \phi$ if and only if 
    \[\phi \in \subintp{\nothing,0}{\MM}{\tp}(\nothing) = 
    \subintp{\nothing,0}{\NN}{\tp}(\nothing)\]
    if and only if $\NN \modelsi \phi$.
\end{proof}

\begin{prop}[$\infty$-equivalence]
    \link{infty_equivalence_01}
    Let $\MM$ and $\NN$ be $\om$-saturated $\Si$-structures.
    If $a \in \MM^n$ and $b \in \NN^n$ such that 
    \[\subintp{\nothing,n}{\MM}{\tp}(a) = 
    \subintp{\nothing,n}{\NN}{\tp}(b)\]
    then 
    \begin{itemize}
        \item (Forth) For any $\al \in \MM$ there exists $\be \in \NN$ such that
        \[\subintp{\nothing,n+1}{\MM}{\tp}(a,\al) = 
        \subintp{\nothing,n+1}{\NN}{\tp}(b,\be)\]
        \item (Back) For any $\be \in \NN$ there exists $\al \in \MM$ such that
        \[\subintp{\nothing,n+1}{\MM}{\tp}(a,\al) = 
        \subintp{\nothing,n+1}{\NN}{\tp}(b,\be)\]
    \end{itemize}
    If this property holds for any pair $a,b$ related by a partial isomorphism
    we say $\MM$ and $\NN$ are $\infty$-equivalent.
\end{prop}
\begin{proof}
    Let $\al \in \MM$ and consider 
    \[p(a,v) := \subintp{a,1}{\MM}{\tp}(\al) \in S_1(\Theory_\MM(a))\]
    Any formula in $p(a,v)$ can be written as a 
    $\Si$-formula $\phi(w,v)$ with variables $w$
    replaced with elements of $a$ 
    ($v$ represents a single variable to be replaced by $\al$). 
    Let 
    \[p(w,v) := \set{\phi(w,v) \st \phi(a,v) \in p(a,v)}\]
    We claim that 
    \[p(b,v) := \set{\phi(b,v) \st \phi \in p(w,v)} \in S_1(\Theory_\NN(b))\]
    To this end, we note that it is indeed 
    a maximal subset of $F(\Si(b),1)$ since for any $\phi(b) \in F(\Si(b),1)$
    \[\phi(a) \in p(a,v) \text{ or } \NOT \phi(a) \in p(a) \implies 
    \phi(b) \in p(b,v) \text{ or } \NOT \phi(b) \in p(b)\]
    We just need to show that it is consistent with $\Theory_\NN(b)$.

    By \linkto{compactness_for_types}{compactness for types} 
    and noting that $\NN$ is a $\Si(b)$-model of $\Theory_\NN(b)$,
    it suffices to show
    that for any finite subset $\De(w,v) \subs p(w,v)$ 
    there exists $\be \in \NN^m$ such that 
    $\NN \model{\Si(b)} \De(b,\be)$.
    \begin{align*}
        &\MM \model{\Si(a)} \bigand{\phi \in \De}{} \phi(a,\al)\\
        \implies &\MM \model{\Si} \exists v, \bigand{\phi \in \De}{} \phi(a,v)\\
        \implies &\brkt{\exists v, \bigand{\phi \in \De}{} \phi(a,v)} \in 
        \subintp{\nothing,n}{\MM}{\tp}(a) = 
        \subintp{\nothing,n}{\NN}{\tp}(b)\\
        \implies &\NN \modelsi \exists v, \bigand{\phi \in \De}{} \phi(b,v)\\
        \implies &\exists \be \in \NN, 
        \NN \modelsi \bigand{\phi \in \De}{} \phi(b,\be)\\
        \implies &\exists \be \in \NN, \NN \model{\Si(b)} \De(b,\be)
    \end{align*}
    Thus $p(b,v)\in S_1(\Theory_\NN(b))$ and since $\NN$ is $\om$-saturated
    $p(b,v)$ is realised in $\NN$ by some $\be$.
    Thus by maximality, $p(b,v) = \subintp{b,1}{\NN}{\tp}(\be)$.

    Finally, for $\phi(v,w) \in F(\Si,n+1)$
    \begin{align*}
        &\phi(v,w) \in \subintp{\nothing,n+1}{\MM}{\tp}(a,\al)
        \iff &\MM \model{\Si} \phi(a,\al) \iff \MM \model{\Si(a)} \phi(a,\al)\\
        \iff &\phi(a,v) \in \subintp{a,1}{\MM}{\tp}(\al) = p(a,v)\\
        \iff &\phi(b,v) \in p(b,v) = \subintp{b,1}{\MM}{\tp}(\be)\\
        \iff &\NN \model{\Si(b)} \phi(b,\be) \iff \NN \modelsi \phi(b,\be)\\
        \iff &\phi(w,v) \in \subintp{\nothing,n+1}{\NN}{\tp}(b,\be)
    \end{align*}
\end{proof}

\begin{prop}[Back and forth method for showing quantifier elimination]
    \link{om_sat_models_and_quantifier_elimination}
    Let $T$ be a $\Si$-theory.
    If $T$ has quantifier elimination then
    for any two $\om$-saturated $\Si$-models of $T$
    are back and forth equivalent.

    If any two $\Si$-models of $T$
    are back and forth equivalent then $T$ has quantifier 
    elimination.
    Note that the backwards implication does not involve saturation.%?
\end{prop}
\begin{proof}
    \begin{forward}
        Let $p$ be a partial isomorphism from $\MM$ to $\NN$.
        By the \linkto{equiv_def_partial_iso}{
            equivalent definition of partial isomorphisms}
        there exists $a \in \MM^n$ and $b \in \NN^n$ such that 
        $p(a) = b$ and 
        \[\subintp{\nothing,n}{\MM}{\qftp}(a) = 
        \subintp{\nothing,n}{\NN}{\qftp}(b)\]
        By \linkto{quant_elim_for_types}{
            quantifier elimination for types}
        \[\subintp{\nothing,n}{\MM}{\tp}(a) = 
        \subintp{\nothing,n}{\NN}{\tp}(b)\]
        The models are $\om$-saturated, hence  
        \linkto{infty_equivalence_01}{by $\infty$-equivalence}
        for any $\al \in \MM$ there exists $\be \in \NN$ such that 
        \[\subintp{\nothing,n+1}{\MM}{\tp}(a,\al) = 
        \subintp{\nothing,n+1}{\NN}{\tp}(b,\be)\]
        Taking only the quantifier free elements,
        we obtain 
        \[\subintp{\nothing,n+1}{\MM}{\qftp}(a,\al) = 
        \subintp{\nothing,n+1}{\NN}{\qftp}(b,\be)\]
        and by the
        \linkto{equiv_def_back_and_forth}{
            equivalent definition of the back and forth property}
        we have that $p$ has the back and forth property.
    \end{forward}

    \begin{backward}
        Let $n \in \N$, $\MM$ and $\NN$ be models of $T$,
        $a \in \MM^n$ and $b \in \NN^n$.
        By \linkto{quant_elim_for_types}{
            quantifier elimination for types} it suffices to show that 
        if \[\subintp{\nothing,n}{\MM}{\qftp}(a) = 
        \subintp{\nothing,n}{\NN}{\qftp}(b)\]
        then \[\subintp{\nothing,n}{\MM}{\tp}(a) = 
        \subintp{\nothing,n}{\NN}{\tp}(b)\]

        This is satisfied as
        \linkto{back_and_forth_gives_quantifier_elimination_lem}{any 
        two models of $T$ are back and forth equivalent}.
    \end{backward}
\end{proof}

\begin{eg}[Quantifier elimination for theory of infinite equivalence relations]
    \[\Si_{\ER} := (\nothing,\nothing,n_f,\set{E},m_r)\]
    where $m_E = 2$, is the signature of equivalence relations.
    We write for variables $x$ and $y$, 
    we write $x \sim y$ as notation for $E(x,y)$
    The theory of equivalence relations $\ER$ 
    is set set containing the following formulas:
    \begin{align*}
        &\text{Reflexivity - } \forall x, x \sim x\\
        &\text{Symmetry - } \forall x \forall y, x \sim y \to y \sim x\\
        &\text{Transitivity - } 
        \forall x \forall y \forall z, (x \sim y \AND y \sim z) \to x \sim z
    \end{align*}
    For $n \in \N_{>1}$ define 
    \begin{align*}
        &\phi_n := \bigexists{i = 1}{n} x_i, \bigand{i < j}{} x_i \nsim x_j\\
        &\psi_n := \forall x, \bigexists{i = 1}{n} x_i, 
            \bigand{i = 1}{n} \brkt{x \sim x_i} \AND 
            \bigand{i < j}{} \brkt{x_i \ne x_j}
    \end{align*}
    Show that the theory $T = \ER \cup {\phi_n, \psi_n}_{1 < i}$ has 
    quantifier elimination.
    (You may wonder if it is indeed a theory
    and what nasty induction must be done to 
    show that its formulas can be constructed.)
\end{eg}
\begin{proof}
    We first define the projection into the quotient:
    if $\MM \model{\Si_\ER} T$ and $a \in {\MM}$ then 
    \[\pi_\MM(a) := \set{b \in {\MM} \st \MM \model{\Si_\ER} a \sim b}\]
    Note that this is a definable set.
    If $A \subs \MM$ we write $\pi_\MM(A)$ to be the image
    \[\{\pi_\MM(a) \st \exists a \in A\]
    Note that the quotient is $\pi_\MM(\MM)$.

    Let $\MM, \NN$ be $\om$-saturated $\Si_\ER$-models of $T$
    and let $p$ be a partial isomorphism from $\MM$ to $\NN$.
    By \linkto{om_sat_models_and_quantifier_elimination}{the back and forth
        condition for quantifier elimination}
    it suffices to show that $p$ has the back and forth property,
    (taking $P$ to be the set of all partial isomorphisms from $\MM$ to $\NN$).
    
    We only show `forth'.
    Let $\al \in \MM$.
    Suppose $\pi_\MM(\al) \cap \dom p$ is empty.
    We can show that $\pi_\NN(\NN)$ is infinite whilst 
    $\pi_\NN(\codom p)$ is finite, 
    hence there exists $\be \in \NN$ such that 
    $\pi(\be) \in \pi_\NN(\NN) \setminus \pi_\NN(\codom p)$ is non-empty.
    Then define $q : \dom p \cup \set{\al} \to \codom p \cup \set{\be}$
    to agree with $p$ on its domain and send $\al$ to $\be$.
    Note that the domain and codomain of $q$ are substructures
    as the language only contains a relation symbol 
    (thus all subsets are substructures).
    We show that $q$ is an isomorphism.
    It is clearly bijective, and to be an embedding it just needs to preserve
    interpretation of the relation.
    Let $a,b \in \dom q$, if 
    both are in $\dom p$ then as $p$ is a partial isomorphism
    \[a \modintp{\MM}{\sim} b \iff p(a) \modintp{\NN}{\sim} p(b) \iff 
    q(a) \nnintp{\sim} q(b)\]
    Otherwise WLOG $a = \al$.
    If $b = \al$ then it is clear.
    If $b \in \dom p$ then by assumption $b \notin \pi_\MM(\al) = \pi(a)$
    hence $\NOT a \mmintp{\sim} b$.
    By construction 
    \[q(a) = q(\al) = \be \implies 
    \pi_\NN(q(a)) \notin \pi_\NN(\codom p) \quad \text{ and } \quad
    q(b) = p(b) \in \codom p\]
    hence $\NOT q(a) \nnintp{\sim} q(b)$.
    Thus $q$ is a local isomorphism extending $p$.

    Suppose $\pi_\MM(\al) \cap \dom p$ is non-empty,
    i.e. there exists $a \in \dom p$ such that $\al \mmintp{\sim} a$
    We can show that 
    $\pi_\NN(p(a))$ is infinite and $\codom p$ is finite
    hence there exists $\be \in \pi_\NN(p(a)) \setminus \codom p$.
    Then define $q : \dom p \cup \set{\al} \to \codom p \cup \set{\be}$
    to agree with $p$ on its domain and send $\al$ to $\be$.
    Again $p$ is clearly a bijection on substructures, 
    and we show that the relation is preserved.
    Let $b,c \in \dom q$. 
    If $b,c \in \dom p$ then it is clear as $p$ is an isomorphism,
    it is also clear if $b,c = \al$.
    Otherwise WLOG $c = \al$ and $b \in \dom p$.
    Then $c = \al \mmintp{\sim} a$ and
    by construction of $\be$ 
    \[q(c) = q(\al) = \be \nnintp{\sim} p(a)\]
    Noting $a \mmintp{\sim} b$ if and only if $p(a) \nnintp{\sim} p(b)$
    as $p$ is a partial isomorphism
    thus $c \mmintp{\sim} a \mmintp{\sim} b$ 
    if and only if $q(c) \nnintp{\sim} p(a) \nnintp{\sim} p(b) = q(b)$.
    Hence $q$ is a local isomorphism extending $p$.
    Thus $p$ has the `forth' property (and similarly the `back' property).
\end{proof} 


%%%%%%%%%%%%%%%%

\begin{prop}[Countable back and forth equivalent structures are isomorphic]
    Let $\MM$ and $\NN$ be $\Si$-structures with
    cardinality less than or equal to $\om$.
    If $\MM$ and $\NN$ are back and forth equivalent then 
    $\MM$ and $\NN$ are isomorphic. 
\end{prop}
\begin{proof}
    Write $\MM = \set{a_i}_{i \in \N}$ and $\NN = \set{b_i}_{i \in \N}$
    (if $\MM$ and $\NN$ are finite then after some $i$ all the elements are 
    equal).
    Inductively define partial isomorphisms $p_n$ for $n \in \N$:
    \begin{itemize}
        \item Take $p_0$ to be the empty function.
        \item If $n + 1$ is odd then ensure $a_{n/2}$ 
            is in the domain:
            by the `forth' property of $P$ there exists $p_{n+1}$ 
            extending $p_n$ such that $a_{n/2} \in \dom(p_{n+1})$.
        \item If $n + 1$ is even then ensure $b_{(n+1)/2}$ is in the codomain:
            by the `back' property of $P$ there exists $p_{n+1}$ 
            extending $p_n$ such that $b_{(n-1)/2} \in \codom(p_{n+1})$.
    \end{itemize}
    We claim that $p$, the union of the partial isomorphisms 
    $p_n$ for each $n \in \N$, is an isomorphism.
    Note that it is well-defined and has image $\NN$ as the $p_i$ are nested and
    for any $a_i \in \MM$ and $b_i \in \NN$, 
    $a_i \in \dom(p_{2i+1})$ and $b_i \in \dom(p_{2i+2})$.
    It is injective: if $a_i, a_j \in \MM$ and $p(a_i) = p(a_j)$ then 
    $p_{2i+2}(a_i) = p_{2i+2}(a_j)$ and so $a_i = a_j$ as $p_{2i+2}$ is a 
    partial isomorphism.
    One can show that it is an $\Si$-embedding.
\end{proof}
