\subsection{The Compactness Theorem}

Read ahead for the statement of \linkto{compactness}{the Compactness Theorem}.
The first two parts of the theorem are easy to prove.
This chapter focuses on proving the final part.
Compactness is a useful tool in classical model theory,
allowing us to reduce a proof of consistency to a 
proof about finite consistency,
which is much easier to work with.
In the finite case we can even just take conjunction of a set of 
sentences and make it just a question about the validity of one sentence.
The term `compactness' corresponds to the topological notion when 
looking at the \linkto{properties_of_stone_space}{Stone space of a theory},
though it is easier to 
\linkto{visualisation_of_Stone}{visualise} once the notion of an 
\linkto{dfn_types_on_theories}{$n$-type} is introduced.

\begin{nttn}
    Write the `size of a model' to be 
    \[
        \abs{\MM} := \abs{\bigcup_{A \in \sort{\Si}} \mmintp{A}}
    \]
\end{nttn}

\begin{dfn}[Witness property]
    Given a signature $\Si$ and a $\Si$-theory $T$, 
    we say that $T$ has the witness property when
    for any $\Si$-formula $\phi$ with 
    exactly one typed free variable $v : A$,
    there exists a constant symbol $c : A$ such that if
    $\exists v : A, \phi(v) \in T$ then $\phi(c) \in T$.
\end{dfn}
We will try to extend our theory such that it is maximal,
finitely consistent, and has the witness property
in what's called a `Henkin construction'.
The following lemma says once we have such a theory then 
it is consistent, not just finitely consistent.
The reason for wanting the witness property is that 
the $\Si$-structure we build in the lemma will interpret sorts as the 
sort sets themselves (modulo some equivalence), 
so we need the language to have enough constant symbols and
for $T$ to `witness' all existence statements using constant symbols.

\begin{lem}[From finitely consistent to consistent]
    \link{henkin}
    Let $\Si$ be a signature.
    If a $\Si$-theory $T$ is maximal, finitely consistent
    and has the witness property then it is consistent.

    Furthermore, let $0 < \ka$ be a cardinal such that 
    $\abs{\const{\Si}} \leq \ka$
    then we can construct a $\Si$-model $\MM$ such that the 
    `size of $\MM$' is bounded by $\abs{\MM} \leq \ka$.
\end{lem}
\begin{proof}
    \textbf{The $\Si$-structure}: 
    For each sort $A$ in $\Si$ we quotient the set $\const{A}$
    of constant symbols of type $A$ by the equivalence relation
    $c \sim d$ if and only if $c = d \in T$.
    Let $\pi_A : \const{A} \to \const{A} / \sim $.
    This defines a $\Si$-structure $\MM$ in the following way:
    \begin{enumerate}
        \item Interpret each sort $A$ as $\pi_A(A)$.
            Clearly the size of $\MM$ is bounded by $\ka$.
        \item To interpret functions we must use the witness property.
        Let $f : \prod A \to B$ be a function symbol and take
        an element of the interpretation of its domain $\prod \mmintp{A}$.
        Since each $\pi_A$ is surjective this element is of the form
        $\mmintp{c} = (\pi_{A}(c_A))_A$ for some
        constant symbols $c = (c_A)_A : \prod A$.
        As $T$ is maximal either $\exists v : B, f(c) = v$ 
        or its negation is in $T$,
        but the latter would imply $T$ is not finitely consistent:
        if there exists $\MM \modelsi \forall v : B, f(c) \ne v$ then 
        the interpretation of $\MM \modelsi f(c) \ne \mmintp{f}\mmintp{c}$,
        a contradiction.
        Hence by the witness property there exists $b : B$ such that 
        $f(c) = b \in T$.
        We define $\mmintp{f}$ to map $\mmintp{c} \mapsto \mmintp{b}$.
        
        To show that $\mmintp{f}$ is well-defined,
        suppose $\mmintp{c_0} = \mmintp{c_1}$ and
        $f(c_0) = b_0, f(c_1) = b_1 \in T$.
        We want that $b_0 = b_1 \in T$.
        By definition of $\MM$, $c_0 = c_1 \in T$.
        If $b_0 = b_1 \notin T$ then its negation is, 
        giving a finitely inconsistent subset 
        \[\set{c_0 = c_1, f(c_0) = b_0, f(c_1) = b_1, b_0 \ne b_0} \subs T\]
        which is a contradiction.
        \item Let $r \hookr A_1 \tdt A_n$. 
        We define 
        \[
            \mmintp{r} := 
            \set{\mmintp{t} \st \text{ variable free term } 
            t : \mmintp{A_1} \tdt \mmintp{A_n} 
            \text{ and }
            r(t) \in T}
        \]
    \end{enumerate}
    Now that we have a $\Si$-structure
    we want to show that it is a $\Si$-model of $T$.

    \textbf{Terms}: We first need that terms are represented 
    by constant symbols: 
    given term $t : B$ with no typed variables $\tv{t} = \nothing$
    and constant symbol
    $d : B$ then $t = d \in T$ 
    if and only if $\MM \model{\Si} t = d$.
    Note that since $t$ has no typed variables it is not a variable itself.
    Hence $t = f(s)$ for some function symbol 
    $f : \prod_i E_i \to B$ and term $s : \prod_i E_i$ with no typed variables.
    By induction we assume the statement to be true for each $s_i : E_i$.

        \begin{forward} 
            If we can find constant symbols
            $e = (e_i)_i : \prod_i E_i$ 
            such that each $e_i$ satisfies
            $s_i = e_i \in T$
            and $f(e) = d \in T$
            then by definition 
            of interpretation of functions 
            $\mmintp{f}(\mmintp{e}) = \mmintp{d}$
            and by induction 
            $\mmintp{s_i}
            = \mmintp{e_i}$ so
            \[
                \mmintp{t} = \mmintp{(f(s))}
                = \mmintp{f}(\mmintp{s})
                = \mmintp{f}(\mmintp{e})
                = \mmintp{(f(e))} 
                = \mmintp{d}
            \]
            Using the witness property for each $i$ 
            we can construct $e$ inductively.
            Suppose we have $e_1 : E_1,\dots, e_{i-1} : E_{i-1}$ 
            such that 
            \[  
                \De_i := \set{e_1 = s_1, \dots, e_{i-1} = s_{i-1},
                \phi_1(e_1),\dots,\phi_{i-1}(e_{i-1})} \subs T
            \]
            where each $\phi_j$ is the formula
            \[
                \exists x_{j+1}, \dots, \exists x_{n}, 
                f(e_1, \dots, e_{j-1},v,x_{j+1},\dots, x_{n}) = d
            \]
            with a single typed free variable $v : E_i$ and bounded variables 
            $x_j : E_j$.
            To complete the induction we find $e_i$ such that 
            $e_i = s_i, \phi_i(e_i) \in T$.
            The witness property gives a constant symbol 
            $e_i : E_i$ such that $\exists v : E_i, \phi_i(v) \in T$ 
            implies $\phi_i(e_i) \in T$.
            
            Suppose for a contradiction $\exists v : E_i, \phi_i(v) \notin T$
            then we have a finite subset
            $\De_i \cup {f(s) = d, \forall v : E_i, \NOT \phi_i(v)} \subs T$
            which we claim is inconsistent.
            Suppose $\NN$ is a model of the finite set.
            Then 
            \[
                \NN \modelsi \forall v, \forall x_{i+1}, \forall x_{n}, 
                \NOT \phi_i
            \]
            So taking $v = \nnintp{s_i}$ and each $x_j = \nnintp{s_j}$
            \[
                \NN \modelsi 
                f(e_1, \dots, e_{j-1},s_i,s_{i+1},\dots, s_{n}) \ne d
            \]
            Now for each $j < i$ we also have $\NN \modelsi e_j = s_j$ and so 
            $\NN \modelsi f(s) \ne d$, a contradiction.
            With the induction done, we have the final $\phi_n$ giving 
            \[f(e_1, \dots, e_{n}) = d \in T \text{ and all } e_i = s_i \in T\]
        \end{forward}

        \begin{backward} 
            Note that each $\mmintp{s_i} = \mmintp{e_i}$ 
            for some $e_i : E_i$ since each $\pi_{E_i}$ is surjective.
            By the induction hypothesis for each $i$ we have 
            $s_i = e_i \in T$.
            Hence 
            \[
                \mmintp{f(e)}
                = \mmintp{f}(\mmintp{e}) 
                = \mmintp{f}(\mmintp{s}) 
                = \mmintp{t} = \mmintp{d}
            \]
            Hence $f(e) = d \in T$ by induction.
            To conclude $f(s) = d \in T$ we note that 
            \[\set{f(s) \ne d, s_1 = e_1,\dots,s_n = e_n, f(e) = d}\]
            is a finite inconsistent subset of $T$.
        \end{backward}
    Thus $t = d \in T \iff \mmintp{t} = \mmintp{d}$.

    \textbf{Formulas}:
    we now show $\MM \modelsi T$. 
    It suffices to show a stronger statement
    which will be needed for the induction:
    for all $\Si$-sentences $\phi$
    \[
        \phi \in T \iff \MM \modelsi \phi
    \]
    We case on what $\phi$ is:
    \begin{itemize}
        \item Case $\phi$ is $\top$:
        all $\Si$-structures satisfy $\top$
        so $\MM \modelsi \top$.
        If $\bot \in T$ then $T$ wouldn't be finitely consistent, 
        hence $\top \in T$.
        \item Case $\phi$ is $t = s$ (note $\phi$ is a sentence so 
        $\tv{t} = \tv{s} = \nothing$): 
        first note $\exists v, t = v \in T$,
        otherwise $\forall v, t \ne v \in T$ so
        $T$ cannot be finitely consistent 
        (any model interprets $t$ as something).
        We apply the witness property to 
        obtain constant symbol $d$ such that
        $t = d \in T$.
        Since terms are represented by constant symbols
        we also deduce $\mmintp{t} = \mmintp{d}$.

        Note $t = s \in T$ if and only if $s = d \in T$:
        if $t = s \in T$ then $s \ne d$ cannot be in the theory 
        by considering finite inconsistent subset $\set{t = d, t = s, s \ne d}$.
        Hence $s = d \in T$. The converse is similar.
        Since terms are represented by constant symbols, 
        \[
            t = s \in T \quad \iff \quad
            d = s \in T \quad \iff \quad 
            \mmintp{d} = \mmintp{s} \quad \iff \quad 
            \mmintp{t} = \mmintp{s}
        \]

        \item Case $\phi$ is $r(t_1,\dots,t_n)$ 
            where $r \hookr A_1 \tdt A_n$ (again $\tv{t_i} = \nothing$):
            \begin{forward}
                Suppose $r(t) \in T$.
                By induction, 
                apply the witness property 
                to the formulas
                \[
                    \phi_i := \exists x_{i + 1} : A_{i+1}, \dots, 
                    \exists x_{n} : A_n, r(\dots, e_{i-1}, v , x_{i+1},\dots) 
                    \AND v = t_i
                \]
                each time obtaining $e_i : A_i$ such that $\phi_i(e_i) \in T$.
                The result is $r(e) \in T$ and each
                $e_i = t_i \in T$.
                Using the claim for terms and 
                how we interpreted relations in $\MM$ this implies
                $\mmintp{t} = \mmintp{e} \in \mmintp{r}$,
                and so $\MM \modelsi r(t)$.
            \end{forward}

            \begin{backward}
                Suppose $\MM \modelsi r(t)$.
                Since $\pi_{A_i}$ are all surjective, 
                there exists $e \in \prod_i A_i$ 
                such that $\mmintp{e} = \mmintp{t} \in \mmintp{r}$.
                Using the claim for terms again
                we obtain $t = e \in T$ and 
                using how $\MM$ interprets relations,
                $r(e) \in T$ so
                $r(t) \in T$.
            \end{backward}
        
        \item Case $\phi$ is $\NOT \chi$:
            Note that by finite consistency
            we cannot have the subset 
            $\set{\chi,\NOT \chi} \subs T$.
            Hence by maximality
            \[\NOT \chi \in T \quad \iff \quad \chi \notin T\]
            Using the induction hypothesis for $\chi$,
            \[
                \chi \notin T \quad \iff \quad 
                \MM \nodel{\Si} \chi \quad \iff \quad
                \MM \model{\Si} \NOT \chi
            \]

        \item Case $\phi$ is $\chi_0 \OR \chi_1$: 
            given the induction hypothesis
            \[
                \MM \model{\Si} \chi_0 \OR \chi_1 
                    \quad \iff \quad \MM \model{\Si} \chi_0
                    \text{ or } \MM \model{\Si} \chi_1 \\
                    \quad \iff \quad \chi_0 \in T \text{ or } \chi_1 \in T
            \]
            Hence it suffices to show that 
            \[
                \chi_0 \in T \text{ or } \chi_1 \in T
                \quad \iff \quad \chi_0 \OR \chi_1 \in T
            \]
            \begin{forward}
                WLOG suppose $\chi_0$ is in $T$.
                If $\chi_0 \OR \chi_1 \notin T$ then 
                $\set{\chi_0, \NOT \chi_0 \AND \NOT \chi_1}$ is a finite
                inconsistent subset of $T$, a contradiction.
            \end{forward}

            \begin{backward}
                Suppose $\chi_0 \OR \chi_1 \in T$.
                If neither $\chi_0$ nor $\chi_1$ is in $T$ then
                $\set{\NOT \chi_0, \NOT \chi_1, \chi_0 \OR \chi_1}$
                is a finite inconsistent subset of $T$.
            \end{backward}

        \item Case $\phi$ is $\forall v : A, \chi(v)$
            \begin{forward}
                Take any element $\mmintp{c} \in \mmintp{A}$
                (using surjectivity of $\pi_A$ have symbol $c : A$).
                We see that $\phi(c) \in T$ since otherwise 
                $\set{\NOT \phi(c), \forall v : A, \chi(v)}$ is a 
                finitely inconsistent subset of $T$.
                Hence by induction $\MM \modelsi \phi(c)$ 
                and we conclude $\MM \modelsi \forall v : A, \chi(v)$.
            \end{forward}

            \begin{backward}
                We show the contrapositive.
                If $\forall v : A, \chi(v) \notin T$, 
                then by maximality $\exists v : A, \NOT \chi(v) \in T$.
                Applying the witness property we obtain constant symbol 
                $c : A$ such that $\NOT \chi(c) \in T$.
                By induction $\MM \nodelsi \chi(c)$ and so 
                $\MM \nodelsi \forall v:A, \chi(v)$.
            \end{backward}
    \end{itemize}
    Thus $\phi \in T \iff \MM \modelsi \phi$ and we are done.
\end{proof}

\begin{nttn}
    We write $\Si \leq \Si^*$ for two signatures if
    $\sort{\Si} \subs \sort{\Si^*}$,
    $\func{\Si} \subs \func{\Si^*}$ and $\rel{\Si} \subs \rel{\Si^*}$.
\end{nttn}

For the sake of formality we need the following two lemmas, 
neither of which are particularly exciting,
but they do allow us to move freely between signatures without worry.
\begin{lem}[Moving models down signatures]
    \link{move_down_mod}
    Given two signatures
    $\Si \leq \Si^*$ and
    a $\Si^*$-structure $\NN$ we naturally have
    a \linkto{move_down_mod_0}{$\Si$-structure $\MM$} such that
    for each sort, function and relation symbol in $\Si$ has the 
    same interpration under $\MM$ and $\NN$.
    Then 
    \begin{itemize}
        \item (Preserves satisfaction) For any $\Si$-formula $\phi$ 
            with $\tv{\phi} = \set{x_i : A_i}_i$ and 
            $a \in \prod_i \mmintp{A_i}$
            \[\MM \model{\Si} \phi(a) \iff \NN \model{\Si^*} \phi(a)\]
        \item (Preserves theories) 
            If $T$ is a $\Si$-theory and $T^*$ is a $\Si^*$-theory 
            such that $T \subs T^*$ and $\NN$ a $\Si^*$-model of $T^*$,
            then $\MM$ is a $\Si$-model of $T$.
    \end{itemize}
    From now on we will write 
    $\NN$ to mean either of the two and let subscripts involving 
    $\Si$ and $\Si^*$ describe which one we mean.
\end{lem}
\begin{proof}
    We will need that for any $\Si$-term $t$,
    the interpretation of terms is equal:
    $\subintp{\Si}{\MM}{t} = \subintp{\Si^*}{\NN}{t}$.
    Indeed:
    \begin{itemize}
        \item If $t$ is a variable $x : A$ then 
        $\subintp{\Si}{\MM}{t} = \id{\subintp{\Si^*}{\MM}{A}} =
        \id{\subintp{\Si^*}{\NN}{A}} = \subintp{\Si^*}{\NN}{t}$
        \item If $t$ is $f(s)$ then 
        $\subintp{\Si}{\MM}{t} = \subintp{\Si}{\MM}{f}(\subintp{\Si}{\MM}{s}) =
        \subintp{\Si^*}{\MM}{f}(\subintp{\Si}{\MM}{s})$ which by induction is
        $\subintp{\Si^*}{\NN}{f}(\subintp{\Si^*}{\NN}{s}) = 
        \subintp{\Si^*}{\NN}{t}$
    \end{itemize}
    
    Let $\phi$ be a $\Si$-formula with $\tv{\phi} = \set{x_i : A_i}_i$ and 
    $a \in \prod_i \mmintp{A_i}$.
    Case on $\phi$ to show that 
    $\MM \model{\Si} \phi(a) \iff \NN \model{\Si^*} \phi(a)$:
    \begin{itemize}
        \item If $\phi$ is $\top$ then both satisfy $\phi$.
        \item If $\phi$ is $t = s$ then since the interpretation 
            of terms are equal
        \[
            \MM \model{\Si} \phi(a) \iff \subintp{\Si}{\MM}{t} = 
            \subintp{\Si}{\MM}{s}
            \iff \subintp{\Si^*}{\MM}{t} = \subintp{\Si^*}{\MM}{s} 
            \iff \NN \model{\Si^*} \phi(a)
        \]
        \item If $\phi$ is $r(t)$ then by definition of
        $\subintp{\Si}{\MM}{r}$ and since interpretation 
        of terms are equal
        \[
            \MM \model{\Si} \phi(a) \iff \subintp{\Si}{\MM}{t}(a) 
            \in \subintp{\Si}{\MM}{r}
            \iff \subintp{\Si^*}{\NN}{t}(a) \in 
            \subintp{\Si^*}{\NN}{r}
            \iff \NN \model{\Si^*} \phi(a) 
        \]
        \item If $\phi$ is $\NOT \psi$ then using the induction hypothesis
            \[
                \MM \model{\Si} \phi(a) \iff \MM \nodel{\Si} \psi(a)
                \iff \NN \nodel{\Si^*} \psi(a)
                \iff \NN \model{\Si^*} \phi(a) 
            \]
        \item If $\phi$ is $\psi \OR \chi$ then using the induction hypothesis
            \begin{align*}
                \MM \model{\Si} \phi(a) \iff \MM \model{\Si} \psi(a) 
                \text{ or } \MM \model{\Si} \chi(a)
                \iff \NN \model{\Si^*} \psi(a) \text{ or } 
                \NN \model{\Si^*} \chi(a)
                \iff \NN \model{\Si^*} \phi(a) 
            \end{align*}
        \item If $\phi$ is $\forall v : B, \psi$ then by induction
        \begin{align*}
            \MM \model{\Si} \phi(a) &\iff \forall b \in \mmintp{B}, 
            \MM \model{\Si} \psi(a,b)\\
                &\iff \forall b \in \mmintp{B}, \NN \model{\Si^*} \psi(a,b)\\
                &\iff \forall b \in \nnintp{B}, \NN \model{\Si^*} \psi(a,b)\\
                & \iff \NN \model{\Si^*} \phi(a) 
        \end{align*}
    \end{itemize}
    Hence $\MM \model{\Si} \phi(a) \iff \NN \model{\Si^*} \phi(a)$.
    It follows that if $T \subs T^*$ are respectively $\Si$ and $\Si^*$-theories 
    and $\NN \model{\Si^*} T^*$ then $\MM \model{\Si} T$.
\end{proof}

\begin{lem}[Moving models and theories up signatures] 
    \link{move_up_mod}
    Suppose $\Si \leq \Si^*$.
    \begin{enumerate}
        \item Suppose $\MM$ is a $\Si$-structure.
            Then i
            whose interpretation agrees with $\MM$
            on symbols from $\Si$,
            then for any $\Si$-formula $\phi$ 
            with $\tv{\phi} = \set{x_i : A_i}_i$ and 
            $a \in \prod_i \mmintp{A_i}$
            \[\MM \model{\Si} \phi(a) \iff \NN \model{\Si^*} \phi(a)\]
            In particular if $\MM$ is a model of $\Si$-theory $T$ then 
            $\MM^*$ is a $\Si^*$-model of $T$ viewed as a $\Si^*$-theory.
        \item Suppose $T$ is a $\Si$-theory and $\phi$ is a 
            $\Si$-sentence such that $T \model{\Si} \phi$.
            Then $T \model{\Si^*} \phi$.
    \end{enumerate}
    Again, if we have constructed such a $\MM^*$ from $\MM$
    we tend to just refer to it as $\MM$ and let subscripts involving 
    $\Si$ and $\Si^*$ describe which one we mean.
\end{lem}
\begin{proof}~
    The proof that satisfaction is preserved is exactly the same as for 
    \linkto{move_down_mod}{moving models down signatures}.
    Suppose $T \model{\Si} \phi$.
    If $\MM^* \model{\Si^*} T$ then by 
    \linkto{move_down_mod}{moving $\MM^*$ down to $\Si$},
    we have a corresponding $\Si$-structure $\MM \model{\Si} T$.
    Hence $\MM \model{\Si} \phi$, and
    since both models agree on interpretation
    of constants, functions and relations of $\Si$ we have
    $\MM^* \model{\Si^*} \phi$ by the previous part.
\end{proof}

\begin{prop}[Henkin (witness)]
    \link{make_wit}
    Suppose $\Si(0)$-theory 
    $T(0)$ is finitely consistent.
    Then there exists a signature 
    $\Si^*$ and $\Si^*$-theory $T^*$ such that 
    \begin{enumerate}
        \item $\const{\Si(0)} \subs \const{\Si^*}$
        and otherwise share the same function and relation symbols.
        \item $|\const{\Si^*}| = |\const{\Si(0)}| + \aleph_0$
        \item $T(0) \subs T^*$
        \item $T^*$ is finitely consistent
        \item Any maximal finitely consistent 
        $\Si^*$-theory $S$ such that $T^* \subs S$ 
        has the witness property
    \end{enumerate}
\end{prop}
\begin{proof}
    We want to define $\Si(i),T(i)$, 
    for each $i \in \N$.    
    By induction, 
    we assume we have signature $\Si(i)$ and 
    $\Si$-theory $T(i)$ such that 
    \begin{enumerate}
        \item $\const{\Si(0)} \subs \const{\Si(i)}$
        and they share the same function and relation symbols.
        \item $|\const{\Si(i)}| = |\const{\Si(0)}| + \aleph_0$
        \item $T(0) \subs T(i)$
        \item $T(i)$ is finitely consistent
    \end{enumerate}
    Note that $\Si(0)$ satisfies this.
    Let 
    \[
        W(i) := \set{\phi \in \form{\Si(i)} 
        \st \abs{\tv{\phi}} = 1}
    \]
    We construct $\Si(i+1)$ 
    by adding constant symbols $c_\phi : A$ for each $\phi \in W(i)$ 
    with $\tv{\phi} = \set{v : A}$ and 
    keeping all the other symbols from $\Si(i)$:
    \[
        \func{\Si(i+1)} :=
        \func{\Si(i)} \sqcup \set{c_\phi : A
        \st \phi \in W(i) \text{ and } \exists v : A, \phi(v) \in T(i)}
    \]
    Then add to $T(i)$ a witness formula 
    for each $\phi \in W(i)$
    \[
        T(i+1) := T(i) \cup \set{(\exists v : A, \phi(v)) \to \phi(c_\phi) \st 
        \phi \in W(i)}
    \]
    Certainly
    $T(i+1)$ is a $\Si(i+1)$-theory such that $T(0) \subs T(i+1)$,
    $\const{\Si(0)} \subs \const{\Si(i+1)}$ with symbols otherwise unchanged.
    Since $W(i)$ is countibly infinite, 
    \[
        |\const{\Si(i+1)}| = |\const{\Si(i)}| + \aleph_0 = 
        |\const{\Si(0)}| + \aleph_0
    \]
    We need to check that $T(i+1)$ is finitely consistent.
    Take a finite subset of $T(i+1)$.
    It is a union of two finite sets 
    \[
        \De_T \subs T(i) \text{ and }
        \De \subs \set{(\exists v : A, \phi(v)) \to \phi(c_\phi) \st \dots}
    \]
    Since $T(i)$ is finitely consistent there exists a $\Si(i)$-model
    $\MM$ of $\De_T$.

    We want $\MM$ to be a $\Si(i+1)$-structure.
    For any sort, function and relation symbol from $\Si(i)$ 
    we take the same interpretation as in $\MM$.
    It remains to interpret the new constant symbols in $\Si(i+1)$, suppose 
    $c_\phi : A$ and $\exists v : A, \phi(v) \in T(i)$ then
    if there exists some $a \in \modintp{\MM}{A}$ such that 
    $\MM \model{\Si(i)} \phi(a)$ then we interpret $c_\phi$ as $a$.
    Otherwise (since interpretation of sorts are non-empty\footnote{
        This is the reason we need to interpret into $* / \SET$.
    }) we take some (any) $a \in \mmintp{A}$ and interpret $c_\phi$ as $a$.    

    We check is is a $\Si(i+1)$-model of $\De_T \cup \De_w$.
    Since $\MM$ is a $\Si(i)$-model
    of $\De_T$ \linkto{move_up_mod}{it is a $\Si(i+1)$-model of $\De_T$}.
    Also for any $(\exists v : A, \phi(v)) \to \phi(c_\phi) \in \De$, 
    if $\MM \model{\Si(i+1)} \exists v : A, \phi(v)$ 
    \linkto{move_down_mod}{then} $\MM \model{\Si(i)} \exists v : A, \phi(v)$,
    hence $\MM$ interprets $c_\phi : A$ as some element 
    $a \in \modintp{\MM(i)}{A}$ such that 
    $\MM \model{\Si(i)} \phi(a)$ 
    \linkto{move_up_mod}{so}
    $\MM \model{\Si(i+1)} \phi(c_\phi)$. 
    Hence $T(i+1)$ is finitely consistent.

    Let $\Si^*$ be the signature with its constant symbols 
    $\const{\Si^*} = \bigcup_{i \in \N} \const{\Si(i)}$
    and otherwise the same symbols as $\Si(0)$.
    Then 
    \[
        |\const{\Si^*}| = |\bigcup_{i \in \N} \const{\Si(i)}| = 
        \aleph_0 \times (\aleph_0 + \const{\Si(0)}) = 
        \aleph_0 + \const{\Si(0)}
    \]

    Let $T^* = \bigcup_{i \in \N} T(i)$.
    Any finite subset of $T^*$ is a subset of some $T(i)$, 
    hence has a $\Si(i)$-model $\MM$.
    By interpreting the added constant symbols as any
    element in the corresponding interpreted sort,
    we \linkto{move_up_mod}{make $\MM^*$ a $\Si^*$-model} of the finite set.
    Hence $T^*$ is finitely consistent.
    
    Suppose $S$ is a maximal finitely consistent 
    $\Si^*$-theory such that $T^* \subs S$,
    and $\phi$ is a $\Si^*$-formula of exactly one typed variable.
    There exists an $i \in \N$ such that $\phi \in \form{\Si(i)}$.
    Hence $(\exists v : A, \phi(v)) \to \phi(c_\phi)$ is in $T(i+1) \subs S$.
    We deduce $\phi(c_\phi) \in S$ since $S$ is maximal and finitely consistent,
    by considering finite inconsistent set
    \[
        \set{\exists v : A, \phi(v), 
        (\exists v : A, \phi(v)) \to \phi(c_\phi), 
        \NOT \phi(c_\phi)}
    \]
\end{proof}

\begin{lem}[Henkin (adding formulas to finitely consistent theories)]
    \link{adding_formulas}
    If $T$ is a finitely consistent $\Si$-theory 
    and $\phi$ is a $\Si$-sentence then at one of
    $T \cup \set{\phi}$ or $T \cup \set{\NOT \phi}$ is finitely consistent.
\end{lem}
\begin{proof}
    We show that for any finite $\De \subs T \cup \set{\phi}$ and 
    finite $\De_\NOT \subs T \cup \set{\NOT \phi}$,
    one of $\De$ or $\De_\NOT$ is consistent. 
    By finite consistency of $T$ the finite subset
    \[
        (\De \setminus \set{\phi}) \cup 
        (\De_\NOT \setminus \set{\NOT \phi}) \subs T
    \]
    is consistent.
    Let $\MM$ be the model of 
    $(\De \setminus \set{\phi}) \cup 
    (\De_\NOT \setminus \set{\NOT \phi})$.
    If $\MM \model{\Si} \phi$ then $\MM \model{\Si} \De$,
    otherwise $\MM \model{\Si} \NOT \phi$ so $\MM \model{\Si} \De_\NOT$.
    Hence $T \cup \set{\phi}$ or $T \cup \set{\NOT \phi}$ 
    is finitely consistent.
\end{proof}

\begin{ex}
    Find a signature $\Si$, a consistent $\Si$-theory $T$
    and $\Si$-sentence $\phi$ such that  
    $T \cup \set{\phi}$ and 
    $T \cup \set{\NOT \phi}$ are both consistent.
\end{ex}
    
\begin{prop}[Extending a finitely consistent theory to a maximal theory (Zorn)]
    \link{make_max}
    Given a finitely consistent $\Si$-theory $T(0)$
    there exists a $\Si$-theory $T^*$ such that 
    \begin{enumerate}
        \item $T(0) \subs T^*$
        \item $T^*$ is finitely consistent.
        \item $T^*$ is a maximal $\Si^*$-theory.
    \end{enumerate}
\end{prop}
\begin{proof}
    We use Zorn's Lemma.
    Consider
    \[\set{T \in \theory{\Si} \st T 
    \text{ finitely consistent and } T(0) \subs T}\]
    be ordered by inclusion.
    Let $(T_i)_{i \in I}$ be a chain.
    Then $\bigcup_{i \in I} T_i$ is a $\Si$-theory
    such that any finite subset is a subset of some $T_i$,
    hence is consistent by finite consistency of $T_i$.
    Zorn's lemma implies there exists a $T^*$ in the set
    that is maximal (in the order theory sense).
    
    To show that it is maximal as a $\Si$-theory, 
    take a $\Si$-sentence $\phi$, then
    \linkto{adding_formulas}{either $T^* \cup \set{\phi}$ or 
    $T^* \cup \set{\NOT \phi}$ is finitely consistent}.
    Hence $T^* \cup \set{\phi} = T^*$ or 
    $T^* \cup \set{\NOT \phi} = T^*$ by (order theoretic) maximality,
    so $\phi \in T^*$ or $\NOT \phi \in T^*$.
\end{proof}

\begin{nttn}[Cardinalities of signatures and structures]
    Given a signature $\Si$, 
    we write
    $|\Si| := |\const{\Si}| + |\func{\Si}| + |\rel{\Si}|$
    and call this the size of the signature $\Si$.
\end{nttn}

\begin{prop}[The compactness theorem]
    \link{compactness}
    If $T$ is a $\Si$-theory, 
    then the following are equivalent:
    \begin{enumerate}
        \item $T$ is finitely consistent.
        \item $T$ is consistent
        \item For any infinite cardinal $\ka$ 
        such that $|\Si| \leq \ka$, there exists a $\Si$-model of 
        $T$ with size $\leq \ka$.
    \end{enumerate}
\end{prop}
\begin{proof}
    $(3. \implies 2.)$ and $(2. \implies 1.)$ are both obvious. 
    For $(1. \implies 3.)$
    suppose $\Si(0)$-theory $T(0)$ is finitely consistent.
    Let $\ka$ be an infinite cardinal such that 
    $|\Si(0)| \leq \ka$. Then
    $|\const{\Si(0)}| \leq |\Si(0)| \leq \ka$.
    We have shown that \linkto{make_wit}{there exists a signature 
    $\Si(1)$ and $\Si(1)$-theory $T(1)$} such that 
    \begin{enumerate}
        \item $\const{\Si(0)} \subs \const{\Si(1)}$
        with other symbols unchanged.
        \item $|\const{\Si(1)}| = |\const{\Si(0)}| + \aleph_0$
        \item $T(0) \subs T(1)$ is finitely consistent.
        \item Any maximal $\Si(1)$-theory $T$ such that $T(1) \subs T$ 
        has the witness property.
    \end{enumerate}
    As $T(1)$ is finitely consistent,
    \linkto{make_max}{there exists a $\Si(1)$-theory $T(2)$} such that 
    \begin{enumerate}[resume]
        \item $T(1) \subs T(2)$
        \item $T(2)$ is finitely consistent.
        \item $T(2)$ is a maximal $\Si(1)$-theory.
    \end{enumerate}
    Furthermore, $T(2)$ has the witness property by design of $T(1)$.
    Finally is a maximal, finitely consistent $\Si(1)$-theory with 
    the witness property, \linkto{henkin}{hence has a $\Si(1)$-model $\MM$ 
    such that $|\MM| \leq \ka$}.
    $\MM \model{\Si(1)} T(0)$ since $T(0) \subs T(1) \subs T(2)$.
    \linkto{move_down_mod}{Moving $\MM$ down} we have a $\Si(0)$-model
    of $T$ of the required size.
\end{proof}