\subsection{The Category of Structures in $\SET$}
\begin{dfn}[Partial $\Si$-morphism, $\Si$-embedding]
    \link{partial_morph_dfn}
    We define partial $\Si$-morphisms and embeddings.
    Suppose $\MM, \NN$ are $\Si$-structures, and for each sort symbol $A$
    we have a subset $S_A \subs \mmintp{A}$.
    and $\io_A : S_A \to \nnintp{A}$.
    We say $\io := (\io_A)_{A \in \sort{\Si}}$ 
    a \textit{partial $\Si$-morphism} from $\MM$ to $\NN$ when 
    \begin{itemize}
        \item For all function symbols $f : A_1 \tdt A_n \to B$ 
        and $a = (a_i) \in \prod S_{A_i}$
        such that $\mmintp{f}(a) \in S_B$, 
        \[\io_B \circ \mmintp{f}(a) = \modintp{\NN}{f} \brkt{\io_{A_i}(a_i)}\]
        \item For all relation symbols $r \hookr A_1 \tdt A_n$ and 
        $a = (a_i) \in \prod S_{A_i}$,
        \[a \in \mmintp{r} \implies (\io_{A_i}(a_i)) \in \modintp{\NN}{r}\]
    \end{itemize}
    Furthermore if $\io$ is injective and we have the pullback condition
    translated to $\SET$:
    \[
        a \in \mmintp{r} \limplies (\io_{A_i}(a_i)) \in \modintp{\NN}{r}
    \]
    then $\io$ is called a \textit{partial $\Si$-embedding} (or extension). 
    In the case that all $S_A = \mmintp{A}$
    we can reproduce the original definitions of $\Si$-morphisms 
    $\io : \MM \to \NN$ as well as a $\Si$-embeddings.
    %
    %For any $\Si$-structure $\MM$, 
    %the identity on each interpreted sort is a $\Si$-morphism,
    %which we take as the identity on $\MM$.
    %We show that composition of morphisms are morphisms
    %and composition of embeddings are embeddings
    %and composition of elementary embeddings are elementary.
\end{dfn}
%\begin{proof}
%    Let $\io_1 : \MM_0 \to \MM_1$ and 
%    $\io_2 : \MM_1 \to \MM_2$ be $\Si$-morphisms.
%    We show that the composition is a $\Si$-morphism:
%    \begin{itemize}
%        \item If $c \in \const{\Si}$ then 
%            \[
%                \io_1 \circ \io_0 (\modintp{\MM_0}{c}) = 
%                \io_1 (\modintp{\MM_1}{c}) = \modintp{\MM_2}{c}
%            \]
%        \item If $f \in \func{\Si}$ and 
%            $a \in {\MM_0}^{n_f}$ then
%            \[
%                \io_1 \circ \io_0 \circ \modintp{\MM_0}{f}(a) = 
%                \io_1 \circ \modintp{\MM_1}{f} \circ \io_0 (a) = 
%                \modintp{\MM_2}{f} \circ \io_1 \circ \io_0 (a)
%            \]
%        \item If $r \in \rel{\Si}$ and 
%            $a \in {\MM_0}^{m_f}$ then
%            \[
%                a \in \modintp{\MM_0}{r} \implies \io_1 (a) \in \modintp{\MM_1}{r} 
%                \implies \io_2 \circ \io_1 (a) \in \modintp{\MM_2}{r}
%            \]
%    \end{itemize}
%    To show that embeddings compose to be embeddings we note that 
%    the composition of injective functions is injective 
%    and if $r \in \rel{\Si}$ and 
%    $a \in {\MM_0}^{m_f}$ then
%    \[
%        a \in \modintp{\MM_0}{r} \iff \io_1 (a) \in \modintp{\MM_1}{r} 
%        \iff \io_2 \circ \io_1 (a) \in \modintp{\MM_2}{r}
%    \]
%    To show that composition of elementary embeddings are elementary,
%    let $\phi \in \form{\Si}$ and $a$ in ${\MM_0}$ be chosen suitably.
%    Then
%    \[
%        \MM_0 \modelsi \phi(a) \iff \MM_1 \modelsi \phi(\io_1 (a))
%        \iff \MM_2 \modelsi \phi(\io_2 \circ \io_1 (a))
%    \]
%\end{proof}

\begin{eg}
    Given the category of structures $\struc{\Si_\RNG , * / \SET}$ for the 
    \linkto{dfn_rings}{signature of rings}, 
    we can take the subcategory whose objects are models 
    of the theory of rings (namely rings), 
    obtaining the category of rings.
    Similarly taking the subcategory whose objects are models of the 
    theory of fields (namely fields) produces the category of fields.
\end{eg}

\begin{dfn}[Elementary embedding]
    A partial $\Si$-embedding $\io : \MM \to \NN$ 
    \textit{preserves} $\phi$ if for any 
    $a \in \prod_{x : A \in \tv{\phi}} \mmintp{A}$,
    \[
        \MM \model{\Si} \phi(a) \quad \iff \quad \NN \model{\Si} \phi(\io(a))
    \]
    where we write $\io(a)$ to mean $(\io_{A}(a_A))_{x : A \in \tv{\phi}}$.

    We say $\io$ \textit{preservers $\phi$ upwards} when only $\implies$ holds 
    and \textit{downwards} when only $\limplies$ holds.
    The embedding is elementary if it preserves all $\Si$-formulas.
\end{dfn}

\begin{prop}[Embeddings preserve quantifier free formulas]
    \link{emb_preserve_sat_of_quan_free}
    Let $\io : \MM \to \NN$ be a $\Si$-embedding, then
    \begin{enumerate}
        \item $\io$ preserves any atomic $\Si$-formula 
            $\top$, $t = s$ or $r(s)$.
        \item If $\io$ preserves $\Si$-formula $\chi$ then it is 
            preserves $\NOT \chi$.
        \item If $\io$ preserves $\Si$-formulas $\chi_0$ and $\chi_1$ 
            then preserves $\chi_0 \OR \chi_1$.
    \end{enumerate}
    Thus from the above we can deduce by induction
    that if $\io$ is elementary with respect to all 
    quantifier free $\Si$-formulas.
\end{prop}
\begin{proof}~
    \begin{itemize}
        \item Trivial.
        \item If $\phi$ is $t = s$ then
            \begin{align*}
                \MM \model{\Si} \phi(a) 
                & \iff \mmintp{t}(a) = \mmintp{s}(a)\\
                & \iff \io(\mmintp{t}(a)) = \io(\mmintp{s}(a)) 
                & \text{by injectivity} \\
                & \iff \nnintp{t}(\io(a)) = \nnintp{s}(\io(a)) 
                & \text{\linkto{morph_comm_term_intp}{
                    morphisms commute with}} \\
                & \iff \NN \model{\Si} \phi(\io(a))
                & \text{\linkto{morph_comm_term_intp}{interpretation of terms}}
            \end{align*}
        \item If $\phi$ is $r(s)$ then
            \begin{align*}
                \MM \model{\Si} \phi(a) 
                & \iff a \in \mmintp{r}& \\
                & \iff \io(a) \in \nnintp{r}
                & \text{pullback}\\
                & \iff \NN \model{\Si} \phi(\io(a))
            \end{align*}
        \item If $\phi$ is $\NOT \chi$ and 
            $\MM \model{\Si} \chi(a)  \iff \NN \model{\Si} \chi(\io(a))$ then
            \[
                \MM \model{\Si} \phi(a) 
                \iff \MM \nodel{\Si} \chi(a) 
                \iff \NN \nodel{\Si} \chi(\io(a))
                \iff \NN \model{\Si} \phi(\io(a))
            \]
        \item If $\phi$ is $\chi_0 \OR \chi_1$ and
            $\MM \model{\Si} \chi_i (a)  \iff \NN \model{\Si} \chi_i (\io(a))$
            \[
                \MM \model{\Si} \phi(a)
                \iff \MM \model{\Si} \chi_0(a) 
                \text{ or } \MM \model{\Si} \chi_1(a)
                \iff \NN \model{\Si} \chi_0(\io(a)) 
                \text{ or } \NN \model{\Si} \chi_1(\io(a))
                \iff \NN \model{\Si} \phi(\io(a))
            \]
    \end{itemize}
\end{proof}

\begin{dfn}[Universal Formula, Universal Sentence]
    \link{universal_formula_def}
    A $\Si$-formula is universal if it can be 
    built inductively by the following two constructors:
    \begin{itemize}
        \item[$\vert$] If $\phi$ is a quantifier free $\Si$-formula 
        then it is a universal $\Si$-formula.
        \item[$\vert$] If $\phi$ is a universal $\Si$-formula then
        $\forall v, \phi(v)$ is a universal $\Si$-formula.
    \end{itemize}
    In other words universal $\Si$-formulas are formulas that start with 
    a bunch of `for alls' followed by a quantifier free formula.
\end{dfn}

\begin{prop}[Embeddings preserve universal formulas downwards]
    \link{emb_preserve_sat_of_forall_down}
    Given $\io : \MM \to \NN$ a $\Si$-embedding and
    $\psi$ a $\Si$-formula such that
    $\io$ preserves $\psi$ then
    $\io$ preserves $\forall v : B, \psi$ \textit{downwards}.

    Dually we can show that embeddings
    preserve existential $\Si$-formulas \textit{upwards}.
\end{prop}
\begin{proof}
    By assumption, for any $a \in \prod_{x : A \in \tv{\phi}} \mmintp{A}$ and 
    $b \in \mmintp{B}$,
    \[\NN \modelsi \psi(\io(a),\io(b)) \quad \implies \quad
    \MM \model{\Si} \psi(a,b)\]
    Then for any $a \in \prod_{x : A \in \tv{\phi}} \mmintp{A}$
    \begin{align*}
        &\quad \NN \modelsi \phi(\io(a))\\
        &\implies \forall b \in {\MM}, 
        \NN \modelsi \psi(\io(a),\io(b)) \\
        &\implies \forall b \in {\MM}, 
        \MM \modelsi \psi(\io(a),\io(b))\\
        &\implies \MM \model{\Si} \phi(a)
    \end{align*}
\end{proof}

\begin{prop}[Isomorphisms are Elementary]
    \link{iso_imp_elem_equiv}
    If two $\Si$-structures $\MM$ and $\NN$ 
    are $\Si$-isomorphic then the isomorphism is elementary.
\end{prop}
\begin{proof}
    Both the isomorphism and its inverse are $\Si$-embeddings
    hence the result follows from induction on formulas using the fact that
    \linkto{emb_preserve_sat_of_quan_free}{
        embeddings preserve quantifier free formulas},
    \linkto{emb_preserve_sat_of_forall_down}{
    embeddings preserving universal formulas downwards}.
    There is a subtlety of having to induct here, using the individual 
    induction steps \linkto{emb_preserve_sat_of_quan_free}{proven before}, 
    due to the complexity of formulas such as $\NOT (\forall v : A, v = v)$.
\end{proof}

\subsection{Vaught's Completeness Test}
Read ahead to the statement of \linkto{vaught_test}{Vaught's Completeness Test}.

\begin{prop}[Infinitely modelled theories have arbitrary large models]
    \link{inf_mod_theory_has_inf_mod}
    Given $\Si$ a signature, $A$ a sort symbol in the signature,
    $T$ a $\Si$-theory with a model $\MM$ such that $\mmintp{A}$ is infinite,
    and a cardinal $\ka$ such that $|\const{\Si}| + \aleph_0 \leq \ka$, 
    then there exists $\NN$ a $\Si$-model of $T$ such that 
    $\ka = |\nnintp{A}|$.

    One can extend this to any sort and obtain the same result replacing 
    $\abs{\nnintp{A}}$ with $\abs{\NN}$.
\end{prop}
\begin{proof}
    Enrich only the signature's constant symbols of type $A$ to create $\Si^*$ 
    a signature such that 
    $\const{\Si^*} = \const{\Si} \cup \set{c_\al : A \st \al \in \ka}$.
    Let $T^* = T \cup \set{
        c_\al \neq c_\be \st \al,\be \in \ka \AND \al \ne \be}$
    be a $\Si^*$-theory.
    
    Using \linkto{compactness}{the compactness theorem}, 
    we show that $T^*$ is finitely consistent.
    Take a finite subset of $T^*$. 
    This is the union of a finite subset $\De_T \subs T$, 
    and a finite subset of 
    $\De_{\ka} \subs 
    \set{c_\al \neq c_\be \st \al,\be \in \ka \AND \al \ne \be}$.
    We want to make the given model
    $\MM$ a ${\Si^*}$-model of $\De_T \cup \De_\ka$ 
    by interpreting the new symbols of $\set{c_\al : A \st \al \in \ka}$
    in a sensible way.
    
    The set $I \subset \ka$ indexing the constant symbols 
    appearing in $\De_\ka$ is finite and $\mmintp{A}$ is infinite, so
    we can find distinct elements of $\mmintp{A}$
    to interpret the elements of
    $\set{c_\al : A \st \al \in I}$. 
    Interpret the rest of the new constant symbols as the same element of 
    non-empty $\mmintp{A}$ (these don't matter),
    then $\MM \model{\Si*} \De_T \cup \De_\ka$.
    Thus $T^*$ is finitely consistent hence consistent.
    
    Using 
    \linkto{compactness}{the third equivalent version of consistency of $T^*$},
    there exists
    $\NN$ a $\Si^*$-model of $T^*$ with $|\nnintp{A}| \leq |\NN| \leq \ka$.
    If $|\nnintp{A}| < \ka$ 
    then there would be $c_\al, c_\be$ that are interpreted as equal,
    hence $\NN \model{\Si^*} c_\al = c_\be$ and 
    $\NN \nodel{\Si^*} c_\al = c_\be$, 
    a contradiction.
    Thus $|\nnintp{A}| = \ka$.
    We can \linkto{move_down_mod}{move $\NN$ down a signature}
    to make it a $\Si$-model of $T$.
\end{proof}

\begin{dfn}[Categoricity]
    Given a single-sorted signature $\Si$
    and a cardinal $\ka$,
    a $\Si$-theory $T$ is called $\ka$-categorical 
    if any two models of $T$ of size $\ka$ are isomorphic.
\end{dfn}

\begin{nttn}
    In single-sorted model theory we often use a $\Si$-structure $\MM$
    to also denote the interpretation of the single sort in $\Si$.
\end{nttn}

\begin{prop}[Vaught's Completeness Test]
    \link{vaught_test}
    Let signature $\Si$ be single-sorted.
    Suppose that $\Si$-theory $T$ is consistent
    with no finite model, 
    and $\ka$-categorical 
    for some cardinal satisfying 
    $|\const{\Si}| + \aleph_0 \leq \ka$.
    Then $T$ is complete.
\end{prop}
\begin{proof}
    Suppose not: if $T$ is not complete then there exists
    $\Si$-formula $\phi$ such that $T \nodel{\Si} \NOT \phi$ and 
    $T \nodel{\Si} \phi$.
    These imply $T \cup \set{\NOT \phi}$
    and $T \cup \set{\phi}$ are both
    \linkto{not_consequence}{consistent}.
    Let $\MM_\NOT$ and $\MM$ be models of  
    $T \cup \set{\NOT \phi}$ and $T \cup \set{\phi}$
    respectively.
    Then each are models of $T$ so they are infinite
    and so $T \cup \set{\NOT \phi}$ and $T \cup \set{\phi}$
    and hence are infinite.

    Since we have $\ka$ such that $|\const{\Si}| + \aleph_0 \leq \ka$, 
    \linkto{inf_mod_theory_has_inf_mod}{
        there exists $\NN_\NOT, \NN$} respectively $\Si$-models of 
        $T \cup \set{\NOT \phi}$ and $T \cup \set{\phi}$
        such that 
        $\ka = |{\NN_\NOT}| = |{\NN}|$.
    Since $T$ is $\ka$-categorical
    $\NN$ and $\NN_\NOT$ are isomorphic.
    As \linkto{iso_imp_elem_equiv}{
        isomorphisms are elementary $\Si$-embeddings} we have a contradiction:
    $\NN \model{\Si} \phi$ and $\NN \model{\Si} \NOT \phi$.
\end{proof}

\subsection{Elementary embeddings and diagrams of models}
\begin{prop}[Tarski-Vaught Elementary Embedding Test]
    \link{tarski_vaught}
    Let $\io : \MM \to \NN$ be a $\Si$-embedding, 
    then the following are equivalent:
    \begin{enumerate}
        \item $\io$ is elementary 
        \item For any 
            $\phi \in \form{\Si}$ preserved by $\io$,
            any $x : B \in \tv{\phi}$ 
            and any $a \in \prod_{B \ne A \in \tv{\phi}} \mmintp{A}$,
            \[\forall b \in \mmintp{B}, \NN \model{\Si} \phi(\io(a),\io(b)) 
            \quad \implies \quad 
            \NN \model{\Si} \forall x : B, \phi(\io(a),x),\]
            which we call the Tarski-Vaught condition.
        \item For any 
            $\phi \in \form{\Si}$ preserved by $\io$,
            any $x : B \in \tv{\phi}$ 
            and any $a \in \prod_{B \ne A \in \tv{\phi}} \mmintp{A}$,
            \[\NN \model{\Si} \exists x : B, \phi(\io(a),x)
            \quad \implies \quad
            \exists b \in \mmintp{B}, \NN \model{\Si} \phi(\io(a),\io(b))\]
            This is dual to the previous.
    \end{enumerate}
\end{prop}
\begin{proof}
    We only show the first two statements are equivalent.
    \begin{forward}
        Since $\io$ is elementary it suffices to show 
        \[\MM \model{\Si} \forall v : B, \phi(a,v)\]
        Let $b \in \mmintp{B}$, 
        then by assumption $\NN \model{\Si} \phi(\io(a),\io(b))$,
        which is implies $\MM \model{\Si} \phi(a,b)$
        as $\io$ is elementary.
        Thus we indeed have $\MM \model{\Si} \forall v : B, \phi(a,v)$.
    \end{forward}

    \begin{backward}
        We induct on $\phi$, 
        though most of the work was already done before.
        \begin{itemize}
            \item If $\phi$ does not start with a $\forall$ 
            then each case of $\phi$ follows from 
            \linkto{emb_preserve_sat_of_quan_free}{
                embeddings preserving quantifier free formulas}.

            \item \linkto{emb_preserve_sat_of_forall_down}{
                Embeddings preserve universal formulas downwards}
                so we only need to check that if $\io$ preserves 
                $\psi$ then $\io$ preserves $\forall x : B, \psi$ upwards.
            \begin{align*}
                \MM \model{\Si} \forall v : B, \psi (a,v) &
                    \implies \forall b : \mmintp{B}, \MM \model{\Si} \psi(a,b)\\
                    &\implies \forall b : \mmintp{B},
                        \NN \model{\Si} \psi(\io(a),\io(b))
                        & \text{induction hypothesis}\\
                    &\implies \NN \model{\Si} \forall x : B, \psi(\io(a),x)
                        & \text{Tarski-Vaught condition}
            \end{align*}
        \end{itemize}
    \end{backward}
\end{proof}

\begin{lem}[Moving elementary embeddings down signatures]
    \link{move_down_morph}
    Suppose $\Si \leq \Si(*)$.
    If $\io : \MM \to \NN$ is an elementary $\Si^*$-embedding then 
    $\io$ is an elementary $\Si$-embedding. 
\end{lem}
\begin{proof}
    Naturally \linkto{morph_diff_sigs}{$\io$ is a $\Si$-embedding},
    we use \linkto{tarski_vaught}{Tarski-Vaught} to show it elementary:
    let $\phi \in \form{\Si}$ be preserved by $\io$,
    let $x : B \in \tv{\phi}$ 
    and let $a \in \prod_{B \ne A \in \tv{\phi}} \mmintp{A}$,
    \linkto{move_down_mod}{Then for any $b \in {\NN}$}
    \[\NN \model{\Si} \phi (\io(a),\io(b)) 
    \iff \NN \model{\Si^*} \phi (\io(a),\io(b))\]
    and $\subintp{\Si}{\MM}{B} = \subintp{\Si^*}{\MM}{B}$. Hence 
    \begin{align*}
        &\forall b \in \subintp{\Si}{\MM}{B}, 
        \NN \model{\Si} \phi(\io(a),\io(b)) \\
        \implies &\forall b \in \subintp{\Si^*}{\MM}{B}, 
        \NN \model{\Si^*} \phi(\io(a),\io(b)) \\
        \implies &\NN \model{\Si^*} \forall x : B, \phi(\io(a),x)
        & \text{$\io$ is elementary in $\Si^*$}\\
        \implies &\NN \model{\Si} \forall x : B, \phi(\io(a),x)
    \end{align*}
\end{proof}
    
\begin{nttn}
    Let $\MM$ be a $\Si$-structure.
    Let $X \subs \mmintp{A}$ be a subset of the interpreted sort.
    Enriching only the constant symbols of $\Si$ we can create a signature 
    $\Si(X)$ such that 
    \[
        \const{\Si(X)} := 
        \const{\Si} \cup \set{c_a : A \st a \in X}
    \]
    In particular we write $\Si(\MM)$ to mean the above process applied to 
    every sort interpreted by $\MM$.
    \[
        \const{\Si(\MM)} := 
        \const{\Si} \cup \set{c_a : A \st A \in \sort{\Si}, a \in \mmintp{A}}
    \]
\end{nttn}

\begin{dfn}[Atomic and elementary diagrams of a structure]
    Let $\MM$ be a $\Si$-structure,
    we \linkto{move_up_mod}{move $\MM$ up} to the signature $\Si(\MM)$
    by interpreting each new constant symbol $c_a$ as $a$.
    We define the atomic diagram of $\MM$ over $\Si$:
    \begin{itemize}
        \item[$\vert$] If $\phi$ is an atomic
        $\Si(\MM)$-sentence such that $\MM \model{\Si(\MM)} \phi$,
        then $\phi \in \atdiag{\Si}{\MM}$.
        \item[$\vert$] If $\phi \in \atdiag{\Si}{\MM}$ then 
        $\NOT \phi \in \atdiag{\Si}{\MM}$.
    \end{itemize}
    We define the elementary diagram of $\MM$ over $\Si$ as 
    \[
        \eldiag{\Si}{\MM} := 
        \set{\phi \in \Si(\MM)\text{-sentences } \st \MM \model{\Si(\MM)} \phi}
    \]

    The elementary diagram of $\MM$ is a maximal $\Si(\MM)$-theory 
    with $\MM$ as a model of it.
    Notice that is \textit{not} the same as the set of all 
    $\Si$-sentences satisfied by $\MM$,
    which is known as the 
    \linkto{theory_of_struc}{theory of $\MM$} in $\Si$.
\end{dfn}

\begin{prop}[Models of the elementary diagram
        are elementary extensions]
    \link{elem_ext_equiv_eldiag_model}
    Let $\MM$ be a $\Si$-structure.
    $\NN$ a $\Si(\MM)$-model of $\atdiag{\Si}{\MM}$ is naturally a 
    $\Si$-extension of $\MM$.
    Furthermore if
    $\NN \model{\Si(\MM)} \eldiag{\Si}{\MM}$ then 
    the embedding is elementary.

    Conversely, given a $\Si$-embedding from 
    $\MM$ into a $\Si$-structure $\NN$, 
    $\NN$ is naturally a $\Si(\MM)$-model $\atdiag{\Si}{\MM}$.
    If the embedding is elementary then it is a 
    $\Si(\MM)$-model of $\eldiag{\Si}{\MM}$.
\end{prop}
\begin{proof}
    \begin{forward}
        Suppose $\NN \model{\Si(\MM)} \atdiag{\Si}{\MM}$.
        Firstly we work in $\Si(\MM)$ to define the embedding:
        \linkto{move_up_mod}{move $\MM$ up a signature}
        by taking the same interpretation as used in the 
        definition of $\Si(\MM)$: 
        \[\intp{\Si(\MM)}{\MM} : c_a \mapsto a\]
        and preserving the same interpretation for symbols of $\Si$.
        We can then write any elements of an interpreted sort $\mmintp{A}$ as 
        $\subintp{\Si(\MM)}{\MM}{c}$, 
        for some constant symbol $c : A$.
        
        This allows us to define a $\Si(\MM)$-morphism 
        $\io : \MM \to \NN$ such that for each constant symbol $c : A$,
        \[\io_A : \subintp{\Si(\MM)}{\MM}{c} \to \subintp{\Si(\MM)}{\NN}{c}\]
        To check that $\io$ is well defined, 
        take $c,d : A$ such that 
        $\subintp{\Si(\MM)}{\MM}{c} = \subintp{\Si(\MM)}{\MM}{d}$.
        \begin{align*}
            &\implies \MM \model{\Si(\MM)} c = d \\
            &\implies c = d \in \atdiag{\Si}{\MM} \\
            &\implies \NN \model{\Si(\MM)} c = d \\
            &\implies \subintp{\Si(\MM)}{\NN}{c} = \subintp{\Si(\MM)}{\NN}{d}
        \end{align*}
        Thus $\io$ is well defined.
        In fact adding `not' gives us injectivity in the same way:
        Take $c,d \in \const{\Si(\MM)}$ such that 
        $\subintp{\Si(\MM)}{\MM}{c} \ne \subintp{\Si(\MM)}{\MM}{d}$.
        \begin{align*}
            &\implies \MM \model{\Si(\MM)} c \ne d \\
            &\implies c \ne d \in \atdiag{\Si}{\MM} \\
            &\implies \NN \model{\Si(\MM)} c \ne d \\
            &\implies \subintp{\Si(\MM)}{\NN}{c} \ne \subintp{\Si(\MM)}{\NN}{d}
        \end{align*}
        Thus $\io$ is injective.
        To check that $\io$ is a $\Si(\MM)$-morphism, 
        we check interpretation of (non-constant) functions and relations.
        The non-constant function and relation symbols are the same 
        for $\Si$ and $\Si(\MM)$.
        Let $f : \prod A \to B$ be a non-constant function symbol in $\Si$
        and $c : \prod A$.
        By design we can find 
        $d : B$ such that $\MM \model{\Si(\MM)} f(c) = d$.
        Hence $f(c) = d \in \atdiag{\Si}{\MM}$.
        Hence $\NN \model{\Si(\MM)} f(c) = d$ and   
        \[
            \io \circ \subintp{\Si(\MM)}{\MM}{f}(\subintp{\Si(\MM)}{\MM}{c}) 
            = \io(\subintp{\Si(\MM)}{\MM}{d}) 
            = \subintp{\Si(\MM)}{\NN}{d}
            = \subintp{\Si(\MM)}{\NN}{f}(\subintp{\Si(\MM)}{\NN}{c})
            = \subintp{\Si(\MM)}{\NN}{f} \circ \io(\subintp{\Si(\MM)}{\MM}{c})
        \]
        Let $r \hookr \prod A$ be a relation symbol in $\Si$ and let 
        $c : \prod A$.
        \begin{align*}
            \subintp{\Si(\MM)}{\MM}{c}\in \subintp{\Si(\MM)}{\MM}{r} 
            &\implies \MM \model{\Si(\MM)} r(c)\\
            &\implies r(c) \in \atdiag{\Si}{\MM}\\
            &\implies \NN \model{\Si(\MM)} r(c) \\
            &\implies \io(\subintp{\Si(\MM)}{\MM}{c}) = 
            \subintp{\Si(\MM)}{\NN}{c} \in \subintp{\Si(\NN)}{\NN}{r} 
        \end{align*}
        To show that $\io$ is an embedding it remains to show 
        \linkto{partial_morph_dfn}{pullback}, which we contrapositive.
        \begin{align*}
            \subintp{\Si(\MM)}{\MM}{c} \notin \subintp{\Si(\MM)}{\MM}{r} 
            &\implies \MM \nodel{\Si(\MM)} r(c)\\
            &\implies \NOT r(c) \in \atdiag{\Si}{\MM}\\
            &\implies \NN \nodel{\Si(\MM)} r(c) \\
            &\implies \io(\subintp{\Si(\MM)}{\MM}{c}) = 
            \subintp{\Si(\MM)}{\NN}{c} \notin \subintp{\Si(\NN)}{\NN}{r} 
        \end{align*}
        Assume furthermore that $\NN \model{\Si(\MM)} \eldiag{\Si}{\MM}$.
        We show that the embedding is elementary.
        Let $\phi$ be a $\Si(\MM)$-formula
        and $\mmintp{c} \in \prod_{x : A \in \tv{\phi}} \mmintp{A}$.
        \begin{align*}
            \MM \model{\Si(\MM)} \phi(\mmintp{c}) 
                &\implies \phi(c) \in \eldiag{\Si}{\MM}\\
                &\implies \NN \model{\Si(\MM)} \phi(c)\\
                &\implies \NN \model{\Si(\MM)} \phi(\io(\mmintp{c}))
        \end{align*}
        and again to show the reverse we contrapositive
        \begin{align*}
            \MM \nodel{\Si(\MM)} \phi(\mmintp{c}) &\implies \NOT \phi(c) 
                \in \eldiag{\Si}{\MM}\\
            &\implies \NN \model{\Si(\MM)} \NOT \phi(c)\\
            &\implies \NN \nodel{\Si(\MM)} \phi(\io(\mmintp{c}))
        \end{align*}
        Hence $\io$ is an (elementary) $\Si(\MM)$-embedding
        \linkto{move_down_morph}{hence an (elementary) $\Si$-embedding}.
    \end{forward}

    \begin{backward}
        Sketch: Suppose $\io : \MM \to \NN$ is an (elementary) embedding.
        Make $\MM$ and $\NN$ into $\Si(\MM)$-structures by 
        $\intp{\Si(\MM)}{\MM}: c_a \to a$ and 
        $\intp{\Si(\MM)}{\NN}: c_a \to \io(a)$,
        for any $a \in \mmintp{A}$.
        It follows that $\io$ becomes an (elementary) $\Si(\MM)$-embedding.
        $\Si(\MM)$-embeddings \linkto{emb_preserve_sat_of_quan_free}{
            preserve quantifier free formulas} so 
            $\NN \model{\Si(\MM)} \atdiag{\Si}{\MM}$.
        When the embedding is elementary,
        any $\Si(\MM)$-sentence satisfied by $\MM$ will be satisfied by $\NN$ 
        and so $\NN \model{\Si(\MM)} \eldiag{\Si}{\MM}$.
    \end{backward}
\end{proof}

\subsection{Universal axiomatization}
\begin{dfn}[Axiomatization, universal theory, universal axiomatization]
    \link{dfn_universal_theory}
    A $\Si$-theory $A$ is an axiomatization of a 
    $\Si$-theory $T$ if for all $\Si$-structures $\MM$,
    \[\MM \model{\Si} T \iff \MM \model{\Si} A\]

    If $A$ is a set of universal $\Si$-sentences 
    is called a universal $\Si$-theory.
    We are interested in universal axiomatizations of theories.
\end{dfn}

\begin{lem}[Lemma on constants]
    \link{lemma_on_const}
    Suppose $\const{\Si} \subs \const{\Si^*}$, 
    $T$ is a $\Si$-theory and $\phi$ a $\Si$-formula
    with $\tv{\phi} = \set{x_1 : A_1, \dots, x_n : A_n}$.
    If there exists $\MM$ a $\Si$-model of $T$ such that 
    \[\MM \modelsi \exists x : \prod_{i = 1}^n A_i, \phi\]
    then $\MM$ can be extended to a $\Si^*$-model of $T$ such that 
    for any tuple of new constant symbols $c = (c_i)$, where
    $c_i : A_i \in \const{\Si^*} \setminus \const{\Si}$, we have
    \[\MM \model{\Si^*} \phi(c)\]

    Packaged differently using the contrapositive we have:
    if there exist
    $c_i : A_i \in \const{\Si^*} \setminus \const{\Si}$
    such that
    $T \model{\Si^*} \phi(c)$ then 
    \[
        T \model{\Si} \forall x : \prod_{i = 1}^n A_i, \phi
    \]
\end{lem}
\begin{proof}
    Suppose there exists $\MM$ a $\Si$-model of $T$
    and $a \in \prod \mmintp{A_i}$
    such that $\MM \modelsi \phi(a)$.

    We \linkto{move_up_mod}{extend $\MM$ to being a $\Si^*$-model of $T$}
    by interpreting
    to the new constant symbols 
    $c : B \in \const{\Si^*} \setminus \const{\Si}$ as
    \[
        \subintp{\Si(\MM)}{\MM}{c} := 
        \begin{cases}
            a_i &, \text{ if } B = A_i\\
            b \in \mmintp{B}&, 
            \text{ otherwise, since each $\mmintp{B}$ is non-empty} 
        \end{cases}
    \]
    Then $\MM$ is a $\Si^*$-model of $T$ such that 
    $\MM \model{\Si^*} \phi(a)$, 
    which by design gives us that for any 
    $c_i : A_i \in \const{\Si^*} \setminus \const{\Si}$
    \[\MM \nodel{\Si^*} \phi(c)\]
\end{proof}

\begin{nttn}
    Let $T$ be a $\Si$-theory, then 
    \[
        T_\forall := 
        \set{\phi \text{ universal $\Si$-sentences} \st T \modelsi \phi}
    \]
    is called the set of universal consequences of $T$.
\end{nttn}

\begin{prop}[Universal axiomatizations make substructures models]
    \link{universal_axiomatizations_make_subs_mods}
    $T$ a $\Si$-theory has a universal axiomatization if and only if
    for any $\Si$-model $\NN$ of $T$ and any $\Si$-embedding 
    from some $\Si$-structure $\MM \to \NN$,
    $\MM$ is a $\Si$-model of $T$.
\end{prop}
\begin{proof}
    \begin{forward}
        Suppose $A$ is a universal axiomatization of $T$,
        $\NN$ is a $\Si$-model of $T$ and $\MM \to \NN$ is a 
        $\Si$-embedding.
        Let $\phi \in T$. Then
        $\NN \model{\Si} T$ implies $\NN \model{\Si} A$
        by definition of $A$.
        $\NN \model{\Si} A$ implies $\MM \model{\Si} A$
        since \linkto{emb_preserve_sat_of_forall_down}{
            embeddings preserve the satisfaction of quantifier free formulas
            downwards}.
        Finally $\MM \model{\Si} A$ implies $\MM \model{\Si} T$
        by definition of $A$.
    \end{forward}

    \begin{backward}
        We show that $T_\forall$ is a universal axiomatization of $T$.
        One direction is obvious:
        any $\Si$-model of $T$ is
        a $\Si$-model of $T_\forall$.

        Suppose $\MM \modelsi T_\forall$.
        We first show that $T \cup \atdiag{\Si}{\MM}$ is consistent.
        By the \linkto{compactness}{compactness theorem} 
        it suffices to show that for any subset 
        $\De$ of $\atdiag{\Si}{\MM}$,
        $T \cup \De$ is consistent.
        Write $\De  = \set{\psi_1, \dots, \psi_n}$.
        Let $\psi = \bigwedge_{1 \leq i \leq n} \psi_i$.
        We can find $S$ the set of constant symbols $c_i : A_i
        \in \const{\Si(\MM)} \setminus \const{\Si}$ that appear in $\psi$
        and create $\phi \in \form{\Si}$ 
        such that $\tv{\phi} = \set{x_i : A_i}_{c_i : A_i \in S}$ and
        $\phi(c) = \psi$, 
        where $c = (c_i)$.
        Since $\De \subs \atdiag{\Si}{\MM}$ we have
        $\forall i, \MM \model{\Si(\MM)} \psi_i$.
        Hence $\MM \model{\Si(\MM)} \phi(c)$.
        Then 
        \[\MM \modelsi \exists x : \prod A_i, \phi \text{ and so }
        \MM \nodel{\Si} \forall x : \prod A_i, \NOT \phi(v)\]
        Since each $\psi_i$ is from the the atomic diagram of $\MM$ 
        they are all quantifier free.
        Thus $\phi$ is a quantifier free $\Si$-formula and 
        $\forall v,  \NOT \phi(v)$ is universal.
        Hence $T \nodel{\Si} \forall v, \NOT \phi(v)$ 
        by the definition of $T_\forall$.
        By \linkto{lemma_on_const}{the lemma on constants}
        this implies that $T \nodel{\Si(\MM)} \NOT \phi(c)$.
        Hence there exists a $\Si(\MM)$-model of $T \cup \phi(c)$.
        Then it follows that this is also a $\Si(\MM)$-model of $T \cup \De$.
        Thus $T \cup \De$ is consistent so
        $T \cup \atdiag{\Si}{\MM}$ is consistent.

        Thus there exists $\NN$ a $\Si$-model of 
        $T \cup \atdiag{\Si}{\MM}$.
        This is a model of $\atdiag{\Si}{\MM}$ so
        \linkto{elem_ext_equiv_eldiag_model}{there 
        is a $\Si(\MM)$-embedding $\MM \to \NN$}.
        \linkto{move_down_morph}{Seeing this as a $\Si$-embedding}
        the theorem's hypothesis tells us $\MM$ is a $\Si$-model of $T$.
    \end{backward}
\end{proof}

The following appears as an exercise in the second 
chapter of Marker's book \cite{marker} and is a consequence of the lemma on 
constants.
\begin{cor}[Amalgamation]
    \link{amalgamation}
    Suppose $\Si$ is single sorted.
    Let $\AA$, $\MM$ and $\NN$ be $\Si$-structures,
    and suppose we have \linkto{partial_morph_dfn}{partial} 
    elementary $\Si$-embeddings
    $\io_\MM : \AA \to \MM$ and 
    $\io_\NN : \AA \to \NN$, such that $\nothing \ne A \subs \AA$
    is the domain for $\io_\MM$ and $\io_\NN$.
    Then there exists a common elementary extension $\PP$ of $\MM$ and $\NN$
    such that the following commutes:
    \begin{cd}
        \MM \ar[r] &\PP \\
        A \ar[u, "\io_\MM"] \ar[r, "\io_\NN", swap] &\NN \ar[u]
    \end{cd}
    $\PP$ is the `amalgamation' of $\MM$ and $\NN$.
\end{cor}
\begin{proof}
    We will denote the only sort symbol by $S$.
    We show that the theory $\eldiag{\Si}{\MM} \cup \eldiag{\Si}{\NN}$
    is a consistent $\Si(\MM,\NN)$-theory, where $\const{\Si(\MM,\NN)}$ is 
    defined to be 
    \[
        \set{c_a \st a \in A} \cup 
        \set{c_a \st a \in \MM \setminus \io_\MM(A)}
        \cup \set{c_a \st a \in \NN \setminus \io_\NN(A)}
    \]
    and other symbols are the same as $\Si$.
    For the rest of the proof we identify 
    \[
        \const{\Si(\MM)} \iso 
        \const{\Si(A)} \cup \set{c_a \st a \in \MM \setminus \io_\MM(A)}
    \]
    by taking $c_{\io_\MM(a)} \mapsto c_a$
    (similarly with $\NN$). 
    Which is why we don't bother writing $\const{\Si(A,\MM,\NN)}$.

    By the \linkto{compactness}{compactness theorem} it suffices to show that 
    for any finite subset $\De \subs \eldiag{\Si}{\NN}$,
    $\eldiag{\Si}{\MM} \cup \De$ is consistent.
    Let $\phi(v)$ be the $\Si(A)$-formula and $a \in \NN^n$ such 
    that \footnote{Take out all the finitely many constants appearing from 
        $\NN \setminus \io_{\NN}(A)$ in 
        $\De$ and make them into a tuple $a$, 
        replacing them with free variables.
        What remains is a finite set of $\Si(A)$-formulas. 
        We take the conjunction of all of them to be $\phi(v)$.}
    \[
        \phi(a) = \bigand{\psi \in \De}{} \psi
    \]
    Then $\phi(a)$ is naturally a $\Si(\NN)$-sentence.
    
    Suppose for a contradiction $\eldiag{\Si}{\MM} \cup \De$ is inconsistent.
    Then any $\Si(\MM,\NN)$-model of 
    $\eldiag{\Si}{\MM}$ is not a model of $\De$,
    which implies it satisfies $\NOT \phi(a)$ so
    \[\eldiag{\Si}{\MM} \model{\Si(\MM,\NN)} \NOT \phi(a)\]
    By the \linkto{lemma_on_const}{lemma on constants} applied to 
    $\Si(\MM) \leq \Si(\MM,\NN)$, $\eldiag{\Si}{\MM}$ and 
    $a \in \const{\Si(\MM,\NN)} \setminus \const{\Si(\MM)}$ we have
    \[
        \eldiag{\Si}{\MM} \model{\Si(\MM)} \forall v : S^n, \NOT \phi(v)
    \]
    $\MM$ is a $\Si(\MM)$-model of its elementary diagram.
    \[
        \MM \model{\Si(\MM)} \forall v, \NOT \phi(v) 
        \linkto{move_down_mod}{\implies} 
        \MM \model{\Si(A)} \forall v, \NOT \phi(v) 
    \]
    Since $\AA \to \MM$ and $\AA \to \NN$ are partial elementary 
    $\Si$-embeddings (hence partial elementary $\Si(A)$-embeddings)
    and $\forall v, \phi(v,w)$ is a $\Si(A)$-formula we have that 
    \[
        \MM \implies \AA \implies 
        \NN \model{\Si(A)} \forall v, \NOT \phi(v)
        \linkto{move_up_mod}{\implies} 
        \NN \model{\Si(\NN)} \forall v, \NOT \phi(v)
    \]
    However $\phi(a)$ was a conjunction of formulas in $\eldiag{\Si}{\NN}$
    so $\NN \model{\Si(\NN)} \phi(a)$, which gives us our contradiction.

    Hence $\eldiag{\Si}{\MM} \cup \eldiag{\Si}{\NN}$
    is consistent as a $\Si(\MM,\NN)$-theory. 
    Let $\PP$ be a $\Si(\MM,\NN)$-model of this
    (and naturally $\Si(\MM)$ and $\Si(\NN)$-models of the respective 
    elementary diagrams).
    Then \linkto{elem_ext_equiv_eldiag_model}{there exist elementary 
    $\Si(\MM)$ and $\Si(\NN)$-extensions}
    $\la_\MM : \MM \to \PP$ and $\la_\NN : \NN \to \PP$.

    Naturally, we can \linkto{move_down_mod}{move everything down} to $\Si(A)$.
    Thus for any $a \in A$ let $c$ be the constant symbol for $a$ in $\Si(A)$:
    \[
        \la_\MM \circ \io_\MM(a) = \la_\MM (\subintp{\Si(A)}{\MM}{c})
        = \la_\MM (\subintp{\Si(\MM)}{\MM}{c}) = 
        \subintp{\Si(\MM)}{\PP}{c} = \subintp{\Si(A)}{\PP}{c}
    \]
    By symmetry we have 
    \[
        \la_\MM \circ \io_\MM(a) = \subintp{\Si(A)}{\PP}{c} =
        \la_\NN \circ \io_\NN(a)
    \]
\end{proof}

\subsection{The L\"{o}wenheim-Skolem Theorems}
\begin{prop}[Upward L\"{o}wenheim-Skolem Theorem]
    \link{upwards_lowenheim_skolem}
    Let $\Si$ be a signature and $A$ a sort symbol in the signature.
    If $\MM$ is a $\Si$-structure with $\mmintp{A}$ infinite,
    and $\ka$ a cardinal such that 
    $|\mmintp{A}|+ |\const{\Si}| \leq \ka$,
    there exists a $\Si$-structure $\NN$ with
    $\abs{\nnintp{A}} = \ka$ as well as an elementary 
    $\Si$-embedding from $\MM$ to $\NN$.
    
    One can extend this to any sort and obtain the same result replacing 
    $\abs{\nnintp{A}}$ with $\abs{\NN}$.
\end{prop}
\begin{proof}
    \linkto{move_up_mod}{$\MM$ is a $\Si(\MM)$-model},
    of $\eldiag{\Si}{\MM}$.
    Thus $\eldiag{\Si}{\MM}$ is a 
    $\Si(\MM)$-theory with a model $\MM$ 
    such that $\mmintp{A}$ is infinite,
    \linkto{inf_mod_theory_has_inf_mod}{
        hence it has a $\Si(\MM)$-model $\NN$ with $\ka = \abs{\nnintp{A}}$}.
    Making $\NN$ a $\Si$-structure, 
    \linkto{elem_ext_equiv_eldiag_model}{
        we obtain a $\Si$-embedding from $\MM$ to $\NN$}.
\end{proof}

\begin{dfn}[Skolem Functions]
    We say that a $\Si$-theory $T$ 
    \emph{has built in Skolem functions} when for any $\Si$-formula $\phi$, 
    with $\tv{\phi} = \set{v : B} \cup W$
    there exists a function symbol $f : \prod_{x : A \in W} A \to B$ 
    such that 
    \[
        \forall w : \prod_{x : A \in W} A , 
        \brkt{\sqbrkt{\exists v : B, \phi(v,w)} \to \phi(f(w),w)}   
    \]
    is a sentence in $T$.
    In particular if $W$ is empty then we have the special case where 
    $f$ is a constant symbol
    \[
        T \model{\Si}  
        \exists v : B, \phi(v) \to \phi(f)
    \]
    which looks like the witness property. 
\end{dfn}

\begin{prop}[Skolemization]
    \link{skolemization}
    Let $T(0)$ be a $\Si(0)$-theory, 
    then there exists $T$ a $\Si$ theory such that
    \begin{enumerate}
        \item $|\func{\Si}| = |\func{\Si(0)}| + \aleph_0$
        \item $\func{\Si(0)} \subs \func{\Si}$, 
            and they share the same relation symbols
        \item $T(0) \subs T$
        \item All $\Si(0)$-models of $T(0)$ are naturally $\Si$-models of $T$.
        \item $T$ has built in Skolem functions
    \end{enumerate}
    We call $T$ the Skolemization of $T(0)$.
\end{prop}
\begin{proof}
    Similarly to the \linkto{make_wit}{Witness Property proof}, 
    we define $\Si(i), T(i)$ for each $i \in \N$.
    Suppose by induction that we have $T(i) \in \theory{\Si}$, 
    such that 
    \begin{enumerate}
        \item $|\func{\Si(i)}| \leq |\func{\Si(0)}| + \aleph_0$
        \item $\func{\Si(0)} \subs \func{\Si(i)}$
            and they share the same relation symbols
        \item $T(0) \subs T(i)$
        \item All $\Si(0)$-models of $T(0)$ are naturally 
            $\Si(i)$-models of $T(i)$.
    \end{enumerate} 
    Then define $\Si(i+1)$ such that only the function symbols are enriched:
    \[
        \func{\Si(i+1)} := \func{\Si(i)} \cup 
        \set{f_{\phi} : \prod_{x : A \in W} A \to B \st \phi \in \form{\Si(i)} 
        \text{ and } \tv{\phi} = \set{v : B} \cup W}
    \]   
    There are countably infinite $\Si(i)$-formulas, 
    thus $|\func{\Si(i)}| = |\func{\Si(0)}| + \aleph_0$ as required.

    For each new function symbol $f_\phi$ define 
    \[
        \Psi(f_\phi) := 
            \forall w : \prod_{x : A \in W} A , 
            \brkt{\sqbrkt{\exists v : B, \phi(v,w)} \to \phi(f_\phi(w),w)}   
    \]
    We then form $\Si(i+1)$-theory $T(i+1)$ 
    by adding each $\Psi(\phi)$ to $T(i)$,
    noting that $T(0) \subs T(i) \subs T(i+1)$.

    Let $\MM$ be a $\Si(0)$-model of $T(0)$, 
    then it is naturally a $\Si(i)$-model of $T(i)$.
    To extend interpretation to $\Si(i+1)$,
    for each new function symbol $f_\phi : \prod_{x : A \in W} A \to B$ define
    \begin{align*}
        \mmintp{f_\phi} : 
        \prod_{x : A \in W} \mmintp{A} &\to \mmintp{B}\\
        a &\mapsto \begin{cases}
            b &, \text{ if there is some } b \in \mmintp{B} \text{ such that }
            \MM \model{\Si(i)} \phi(b,a)\\
            \text{anything} &, \text{ otherwise, since each } 
            \mmintp{B} \text{ is non-empty}
        \end{cases}\\
    \end{align*}
    Then by construction for each $f_\phi$, 
    $\MM \model{\Si(i+1)} \Psi(f_\phi)$.
    By induction we had \linkto{move_up_mod}{$\MM \model{\Si(i+1)} T(i)$},
    which together imply $\MM(i+1) \model{\Si(i+1)} T(i+1)$.
    
    Let $\Si^*$ 
    be the signature with relation symbols from $\Si(0)$
    and $\func{\Si^*} = \bigcup_{i \in \N} \func{\Si(i)}$.
    Then 
    \[
        |\func{\Si^*}| = |\bigcup_{i \in \N} \func{\Si(i)}| = 
        \aleph_0 \times (\aleph_0 + \func{\Si(0)}) = 
        \aleph_0 + \func{\Si(0)}
    \]

    Let $T^* = \bigcup_{i \in \N} T(i)$.
    We show that $T^*$ has built in Skolem functions.
    Let $\phi$ be a $\Si^*$-formula with a free variable $v : B$.
    Then $\phi \in \form{\Si(i)}$ for some $i \in \N$. 
    Thus $\Psi(f_\phi) \in T(i+1) \subs T^*$ as required.
    
    If $\MM \model{\Si} T$ then extend the interpretation 
    to $\Si^*$ such that for all $i \in \N$, 
    and $f \in T(i)$, 
    $\subintp{\Si^*}{\MM}{f} = \subintp{\Si(i)}{\MM}{f}$.
    Since all interpretations agree upon intersection this is well-defined.
    Since for each $i$ we have $\MM \model{\Si(i)} T(i)$
    we have \linkto{move_up_mod}{$\MM \model{\Si^*} T(i)$} and so 
    $\MM$ is a $\Si^*$-model of $T^*$.
\end{proof}

\begin{dfn}[Theory of a Structure]
    \link{theory_of_struc}
    \begin{nttn}
        For $\MM$, a $\Si$-structure we write $X \subs \MM$
        to mean: for each sort $A$ there exists 
        $X_A \subs \mmintp{A}$.
    \end{nttn}

    We define the theory of a $\Si$-structure $\MM$ to be
    \[\Theory_{\MM}(\Si) := \set{\phi \in \form{\Si} \st 
    \phi \text{ is a $\Si$-sentence and } \MM \model{\Si} \phi}\]
    If the signature is obvious we just write $\Theory_\MM$.
    Note that $\Theory_\MM(\Si)$ is a consistent and complete $\Si$-theory
    as it is modelled by $\MM$ 
    and any formula is either satisfied by $\MM$ or not.
    
    Let $\MM$ be a $\Si$-structure and let $X \subs \MM$.
    $\MM$ is \linkto{move_up_mod}{naturally a $\Si(X)$-structure}. 
    The theory of $\MM$ over $X$ is defined by
    \[\Theory_\MM(X):= \Theory_\MM(\Si(X))\]
    Note that $\Theory_\MM(\MM)$ is the elementary diagram $\eldiag{\Si}{\MM}$
    (where $\MM$ is seen as the collection of 
    all the $\mmintp{A}$ for each sort $A$).
\end{dfn}

\begin{prop}[Downward L\"{o}wenheim-Skolem Theorem]
    Let $\NN$ be an infinite $\Si(0)$-structure and $M \subs \NN$
    (noting $M \subs \NN$ is \linkto{theory_of_struc}{notation})
    for each sort symbol $A$.
    Then there exists a $\Si(0)$-structure $\MM$ such that 
    \begin{itemize}
        \item $M \subs \MM \hookr \NN$ 
            and the `inclusion' $\MM \hookr \NN$ 
            is an elementary $\Si(0)$-embedding.
        \item $|\MM| \leq |M| + |\func{\Si(0)}| + \aleph_0$,
            where $M := \bigcup_{A \text{sort symbol}} M_A$
    \end{itemize}
\end{prop}
\begin{proof}
    We first take the \linkto{skolemization}{Skolemization} 
    of $\Theory_{\NN}$ and call the new signature and theory $\Si$ and $T$.
    By assumtion $\NN \model{\Si(0)} \Theory_{\NN}$, 
    and by construction of $T$, $\NN$ becomes a $\Si$-model $T$.
    Also note that by construction $\abs{\Si} \leq \abs{\Si(0)} + \aleph_0$.

    We want to make the smaller model $\MM$.
    It must be closed under taking functions, so we define it inductively.
    Suppose by induction we have for each sort symbol $A$ a 
    set $M_A \subs M_A(i) \subs \nnintp{A}$ such that 
    \[
        |M(i)| \leq |M| + |\func{\Si}| + \aleph_0
    \] 
    where $M(i) = \bigcup_{A \text{sort symbol}} M_A(i)$. 
    We inductively define each $M_B(i+1)$:
    \[
        M_B(i+1) := M_B(i) \cup \set{\nnintp{f}(a) \st 
        f : \prod A \to B \text{ is a function symbol and } a \in \prod M_A(i)}
    \]
    Then \begin{align*}
        |M(i+1)| 
        &\leq   |M(i)| + |\func{\Si}| \times |\prod M_A(i)|\\
        &\leq   |M(i)| + |\func{\Si}| \times (|M_A(i)| \times \aleph_0)\\
        &\leq   |M(0)| + |\func{\Si}| + \aleph_0 + |\func{\Si}| 
        \times (|M(0)| + |\func{\Si}| + \aleph_0)\\
        &\leq   |M(0)| + |\func{\Si}| + \aleph_0
    \end{align*}
    Then $\mmintp{A} := \bigcup_{i} M_A(i)$ and 
    \[
        |\MM| \leq (|M(0)| + |\func{\Si}| + \aleph_0) \times \aleph_0 \leq
        |M(0)| + |\func{\Si}| + \aleph_0 
        \leq |M(0)| + |\func{\Si(0)}| + \aleph_0
    \]
    To interpret function symbols, let $f : \prod A \to B$
    be a function symbol and let $a \in \prod \mmintp{A}$.
    Then there exists an $i$ such that $a \in \prod M_A(i)$
    so we let $\mmintp{f}$ take $a$ to $\nnintp{f}(a) \in M_B(i+1)$.
    We define the interpretation of relations 
    $r \hookr \prod A$ as the product of intersections (pullbacks)
    $\mmintp{r} := \prod \mmintp{A} \cap \nnintp{r}$.

    By construction the inclusion $\subs$ is a $\Si$-embedding. 
    We check that it is elementary using the third equivalent condition in the
    \link{tarski_vaught}{Tarski-Vaught Test}:
    let $\phi \in \form{\Si}$ be preserved by $\subs$,
    $x : B \in \tv{\phi}$ 
    and $a \in \prod_{B \ne A \in \tv{\phi}} \mmintp{A}$.
    Suppose
    \[\NN \model{\Si} \exists x : B, \phi(a,x)\] 
    $T$ has built in Skolem functions and $\NN \model{\Si} T$ so
    there exists $f \in \func{\Si}$ such that
    \[
        \NN \model{\Si} (\exists x : B, \phi(a,v)) \to \phi(a,f(a)) 
    \]
    Hence $\NN \model{\Si} \phi(a,f(a))$.
    Hence the embedding $\MM \to \NN$ is elementary.
    \linkto{move_down_morph}{Then} $\subs : \MM \to \NN$ 
    is an elementary $\Si(0)$-embedding.
\end{proof}
