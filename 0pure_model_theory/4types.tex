\section{Types}
This section mainly follows material from Tent and Ziegler's book \cite{tent}.
There is an unfortunate terminology clash with types in type theory.
Model theoretic $n$-types have nothing to do with type theoretic 
types and $n$-types.
However, the model theoreic notions of `sorts' and `terms of type $A$ a sort' 
\textit{does} line up with the notion of a term of of type $A$ 
(usually $A$ would be called a `type' in type theory).

For this chapter we fix a signature $\Si$ and 
$V_n = {v_1 : A_1, \dots, v_n : A_n}$
a finite set of typed variables in $\Si$.

\subsection{Types}
\begin{dfn}[$F(\Si,V_n)$ and $n$-dimensional theories]
    \link{dfn_types_on_theories}
    Let $F(\Si,V_n)$ be the set of $\Si$-formulas whose typed free variables
    are amongst those $n$ variables.
    We call $p \subs F(\Si,V_n)$ an $n$-dimensional theory, 
    noting that if $n = 0$ we have the original notion of a $\Si$-theory.
    We will define $n$-types later as $n$-dimensional theories that have 
    some notion of consistency.
    We say $p$ is a maximal if for any $\phi \in F(\Si,V_n)$,
    $\phi \in p$ or $\NOT \phi \in p$, 
    generalising the notion of maximality of a theory.

    For any tuple of constant symbols $c : \prod_{i = 1}^n A_i$
    and $n$-dimensional theory $p \subs F(\Si,V_n)$ we write 
    \[
        p(c) = \set{\phi(c) \st \phi \in p}
    \]
    and if $\MM$ is a $\Si$-structure with 
    $a \in \prod_{i=1}^n \mmintp{A_i}$ we write 
    \[
        \MM \model{\Si} p(a)
    \]
    to mean for every $\phi \in p$, $\MM \model{\Si} \phi(a)$.
\end{dfn}

\begin{dfn}[Realisation]
    Let $T$ be a $\Si$-theory.
    Let $p \subs F(\Si,V_n)$ be an $n$-dimensional theory.
    Let $\MM$ be a $\Si$-structure.
    \begin{itemize}
        \item $p$ is \textit{realised} in $\MM$ by 
        $a \in \prod_{i = 1}^n \mmintp{A_i}$ over $A$ if
            \[\MM \modelsi p(a)\]
            We also just say $p$ is realised in $\MM$.
            If $p$ is not realised in $\MM$
            then we say $\MM$ \textit{omits} $p$.
        \item $p$ is \textit{finitely realised} in $\MM$ 
            over $A$ if for each finite subset $\De \subs p$ 
            there exists $a \in \prod_{i = 1}^n \mmintp{A_i}$
            such that $\De$ is realised in $\MM$ by $a$.
    \end{itemize}
\end{dfn}

\begin{eg}
    We have a notion of \linkto{compactness_for_types}{compactness} 
    for $n$-dimensional theories,
    but there is no such notion for realisation.
    For example, let $\Si$ be the 
    \linkto{sig_just_one_bin_rel}{signature with just a binary relation $<$}.
    For the 
    \linkto{order_theories}{
        $\Si$-theory of dense linear orders} we have a $\Si$-model 
    $\Q$ which finitely realises the $1$-dimensional theory
    \[p = {x < q \st q \in \Q}\]
    since any finite set has an infemum.
    However there is no element of $\Q$ realising $p$.
    We will \linkto{finite_realisation_and_embeddings}{later} 
    see that finite realisation of $p$ 
    implies existence of an elementary extension of $\Q$ realising $p$.
    An example of this would be $\Q \cup {-\infty}$ with the obvious ordering.
\end{eg}

\begin{dfn}[$n$-types and $n$-dimensional compactness]
    \link{equiv_def_of_consistent_with_theory}
    \link{compactness_for_types}
    Let $T$ be a $\Si$-theory and $p \subs F(\Si,V_n)$ 
    be an $n$-dimensional $\Si$-theory.
    Let $\Si(c)$ be the signature with added constant symbols
    $c_1 : A_1,\dots,c_n : A_n$.
    The following are equivalent (see proof):
    \begin{enumerate}
        \item $T \cup p(c)$ is a consistent $\Si(c)$-theory.
        \item (Consistent with $T$) There exists 
            $\MM \model{\Si} T$ such that $p$ is realised in $\MM$. 
        \item (Finitely consistent with $T$) 
            For any finite subset $\De \subs p$, there exists 
            $\MM \model{\Si} T$ such that $p$ is realised in $\MM$.
    \end{enumerate}
    If any of the above is true then we say $p$ an \textit{$n$-type 
    consistent with $T$} 
    (also called a \textit{partial} $n$-type on $T$).
    The second and third definitions being equivalent is 
    the $n$-dimensional \linkto{compactness}{compactness} theorem 
    (also called \textit{compactness for types}).

    Furthermore if $\MM$ is a $\Si$-structure with subset $X \subs \MM$
    (noting $X \subs \MM$ is \linkto{theory_of_struc}{notation}), 
    we say that $p$ is an \textit{$n$-type on $\MM$ over $A$} if 
    $p$ is an $n$-type consistent with 
    \linkto{theory_of_struc}{$\Theory_\MM(A)$}.
\end{dfn}
\begin{proof}
    ($1. \iff 2.$) 
        Suppose we have a $\Si(c)$-structure 
        $\MM \model{\Si(c)} T \cup p(c)$.
        Then by taking the images of the interpretation of each $c_i$ in 
        $\mmintp{A_i}$
        we obtain $a = \modintp{\MM}{c} \in \prod \mmintp{A_i}$ such that 
        $\MM \model{\Si(c)} p(a)$.
        \linkto{move_down_mod}{Moving this down to $\Si$} we have
        \[\MM \model{\Si} T \cup p(a)\]
        Conversely suppose we have $\MM \model{\Si} T$ and 
        $a \in \prod \mmintp{A_i}$ such that $\MM \modelsi p(a)$.
        We can \linkto{move_up_mod}{make $\MM$ a $\Si(c)$-structure} such that 
        everything from $\Si$ is interpreted in the same way 
        and each constant symbol $c_i$ is interpreted as $a_i$.
        Thus for any $\phi(c) \in p(c)$,
        \[\MM \model{\Si} \phi(a) \implies \MM \model{\Si(c)} \phi(a)
        \implies \MM \model{\Si(c)} \phi(c)\]
        Hence $\MM \model{\Si(c)} T \cup p(c)$ 
        and $T \cup p(c)$ is consistent in $\Si(c)$.

    ($2. \iff 3.$) 
    \begin{align*}
        &\quad p \text{ consistent with } T \\
        &\iff T \cup p(c) \text{ consistent in } \Si(c) 
        \quad \text{by (1. $\iff$ 2.)}\\
        &\iff \text{for any finite } \De(c) \subs p(c), 
        T \cup \De(c) \text{ consistent in } \Si(c) 
        \quad \text{by \linkto{compactness}{compactness}}\\
        &\iff \text{for any finite } \De \subs p, 
        T \cup \De(c) \text{ consistent in } \Si(c) \\
        &\iff \text{for any finite } \De \subs p, 
        \De \text{ consistent with } T
        \quad \text{by (1. $\iff$ 2.)}
    \end{align*}
\end{proof}

\begin{dfn}[Type of an element]
    Let $\MM$ be a $\Si$-structure and $a \in \prod_{i=1}^n \mmintp{A_i}$
    Then 
    \[\subintp{\Si,n}{\MM}{\tp}(a) := 
    \set{\phi \in F(\Si,V_n) \st \MM \model{\Si} \phi(a)}\]
    is the type of $a$ in $\MM$ over $A$.
    One can verify that if $\MM$ is any $\Si$-structure then 
    $\tp(a)$ (we get rid of superscripts and subscripts when clear) 
    is a maximal $n$-type consistent with any $\Si$-theory modelled by $\MM$.
    If \linkto{theory_of_struc}{$X \subs \MM$} 
    and we want to do the same but with signature $\Si(X)$
    we write $\subintp{X,n}{\MM}{\tp}(a)$.
\end{dfn}

\begin{dfn}[Equivalence modulo a theory]
    \link{dfn_modulo_theory}
    We say two $\Si$-formulas $\phi$ and $\psi$ in $F(\Si,V_n)$ 
    are equivalent modulo a $\Si$-theory $T$ if 
    \[T \model{\Si} \forall v : \prod_{i=1}^n A_i, \brkt{\phi \iff \psi}\]
\end{dfn}

\begin{dfn}[Stone space of a theory]
    Let $T$ be a $\Si$-theory.
    Let the stone space of $T$
    be the set $S_n(T)$ of all maximal $n$-types on $T$.
    (The signature of the $n$-types of the on $T$ is implicit, 
    given by the signature of $T$.)
    We give a topology on $S_n(T)$ by specifying an open basis,
    which consists of the subsets
    \[
        [\phi]_T := \set{p \in S_n(T) \st \phi \in p}
    \]
    for each $\phi \in F(\Si,V_n)$.
\end{dfn}

\begin{lem}[Extending to maximal $n$-types]
    \link{extend_to_maximal_type_zorn}
    \link{elems_of_stone_space_are_types_of_elements}
    Any $n$-type can be extended to a maximal $n$-type
    by taking the type of an element.
    In particular $p$ is a maximal $n$-type consistent with theory $T$ 
    if and only if $p = \subintp{n}{\MM}{\tp}(a)$ for 
    a model $\MM$ of $T$ and some $a \in \prod_{i=1}^n \mmintp{A_i}$.
    Hence elements of the Stone space of a theory are types of elements.
\end{lem}
\begin{proof}
    Let $T$ be a $\Si$-theory and $p$ be a $n$-type.
    Then by definition $p$ is realised by some 
    $a \in \prod_{i=1}^n \mmintp{A_i}$, 
    where $\MM$ is a $\Si$-model of $T$.
    Taking the type of element $a$ we have 
    $p \subs \subintp{n}{\MM}{\tp}(a)$ is a maximal $n$-type extending $p$.
\end{proof}

%Furthermore, if $\MM \to \NN$ is an elementary $\Si$-embedding 
%    and $a \in \prod_{i=1}^n \mmintp{A_i}$ then
%    \[
%        \subintp{\Si,n}{\MM}{\tp}(a) = \subintp{\Si,n}{\NN}{\tp}(a)
%    \]
%For the last part note that 
%    \[
%        \phi \in \subintp{\Si,n}{\MM}{\tp}(a) \iff \MM \model{\Si(A)} \phi(a)
%        \iff \NN \model{\Si(A)} \phi(a) \iff 
%        \phi \in \subintp{A,n}{\NN}{\tp}(a)
%    \]

\begin{prop}[Elementary properties of the Stone space]
    \link{basic_facts_basis_elems}
    Let $T$ be a $\Si$-theory, $p \in S_n(T)$, and $\phi, \psi \in F(\Si,V_n)$.
    \begin{itemize}
        \item $(\NOT \phi) \in p$ if and only if $p \notin [\phi]_T$.
        \item $[\phi]_T = [\psi]_T$ if and only if $\phi$ and $\psi$ are 
            equivalent modulo $T$; $[\phi]_T \subs [\psi]_T$ if and only if
            $\phi$ and $\psi$
            \[
                T \modelsi \forall v : \prod_{i = 1}^n A, \phi \to \psi
            \]
        The basis elements are closed under Boolean operations
        \item $[\NOT \phi]_T = S_n(T) \setminus [\phi]_T$
        \item $[\phi \OR \psi]_T = [\phi]_T \cup [\psi]_T$
        \item $[\phi \AND \psi]_T = [\phi]_T \cap [\psi]_T$
        \item $[\top]_T = S_n(T)$ and $[\bot]_T = \nothing$ and
            in particular if $[\phi]_T = S_n(T) = [\top]_T$ then 
            $\phi$ and $\top$ are equivalent modulo $T$ hence
            $T \modelsi \forall v, \phi$.
    \end{itemize}
\end{prop}
\begin{proof}
    We will just prove a couple of these.
    \begin{itemize}
        \item Suppose $(\NOT \phi) \in p$.
            If $p \in [\phi]_T$ then $phi, \NOT \phi$ are both in $p$,
            but $p$ is consistent with $T$ so
            there exists a model $\MM$ and $a$ from 
            $\MM$ such that
            $\MM \modelsi \phi(a)$ and $\MM \nodel{\Si} \phi(a)$, 
            a contradiction.
            For the other direction, 
            $p \notin [\psi]_T$ and so $ \psi \notin p$ and
            by maximality $\NOT \phi \in p$.
        \item We show the $\IFF$ statement. 
            \begin{forward}
                Let $\MM \model{\Si} T$ and 
                $a \in \prod_{i=1}^n \mmintp{A_i}$.
                We want that $\MM \modelsi \phi(a) \IFF \psi(a)$.
                By symmetry it suffices to assume $\MM \modelsi \phi(a)$ 
                and show $\MM \modelsi \psi(a)$.
                Taking the type of element $a$ gives us maximal $n$-type
                $\subintp{n}{\MM}{\tp}(a) \in [\phi]_T \subs [\psi]_T$.
                Hence $\MM \modelsi \psi(a)$.
            \end{forward}
            \begin{backward}
                Suppose $T \model{\Si} \forall v, \brkt{\phi \IFF \psi}$.
                Let $p \in [\phi]_T$; by symmetry it suffices
                to show that $p \in [\psi]_T$.
                Since $p$ is consistent with $T$ there exists a $\Si$-structure
                $\MM \model{\Si} T$ and 
                $a \in \prod_{i=1}^n \mmintp{A_i}$ such that 
                $\MM \model{\Si} p(a)$.
                By assumption $\MM \model{\Si} \brkt{\phi \to \psi}(a)$
                and $p \in [\phi]_T$ so 
                $\MM \model{\Si} \psi(a)$.
                By maximality of $p$ we just need to note that the case
                $\NOT \psi \in p$ implies $\MM \nodelsi \psi(a)$
                which is false.
            \end{backward}
    \end{itemize}
\end{proof}

\begin{prop}[Topological properties of the Stone space]
    \link{properties_of_stone_space}
    Let $T$ be a theory.
    \begin{itemize}
        \item Elements of the basis of $S_n(T)$ are clopen.
        \item $S_n(T)$ is Hausdorff.
        \item $S_n(T)$ is compact.
    \end{itemize}
\end{prop}
\begin{proof}~
\begin{itemize}
    \item By maximality of each $p$ the complement of $U$ is also 
        in the open basis:
        \[  
            \set{p \in S_n(T) \st \phi \notin p}
            = \set{p \in S_n(T) \st (\NOT \phi) \in p}
        \]
        Hence each element of the basis is clopen.
    \item Let $p,q \in S_n(T)$ and suppose they share the same neighbourhoods.
        then for any formula $\phi \in F(\Si,V_n)$ we have 
        \[ p \in [\phi]_T \iff q \in [\phi]_T\]
        Hence $\phi \in p$ if and only if $\phi \in q$, so $p = q$.
    \item We use the
        \linkto{compactness_for_types}{compactness theorem for types}.
        Let $C$ be a collection of closed sets with the finite intersection
        property.
        Each closed $U \in C$ can be written as a possibly infinite 
        intersection of a finite union of basis elements.
        A finite union of closed sets is still a basis element
        since \linkto{basic_facts_basis_elems}{$[\phi]\cup[\psi]_T = 
            [\phi \OR \psi]_T$},
        so for each $U$ there exists $\Ga_U \subs F(\Si,V_n)$ such that
        \[
            U = \bigcap_{\phi_U \in \Ga_U}[\phi_U]_T
        \]
        Sets of closed sets will correspond (roughly) to sets of formulas, and
        non-emptiness of the intersection of the closed sets corresponds to 
        consistency of that set of formulas with $T$:
        an $n$-type $p$ in the intersection of all the closed sets gives 
        consistency of the set of formulas with respect to $T$.
        The right set of formulas (corresponding to $C$) to take is 
        \[
            \Ga := \set{\phi_U \in \Ga_U \st U \in C} 
            \quad \text{ which is the same data as } \quad 
            [\Ga]_T := \set{[\phi_U]_T \st \phi_U \in \Ga}
        \]
        The intersection any finite subset $[\De]_T \subs [\Ga]_T$ 
        (due to finite $\De \subs \Ga$)
        is non-empty as it contains a finite intersection of elements in $C$.
        \[
            \bigcap_{\phi_U \in \De} [\phi_U]_T \supseteq 
            \bigcap_{\phi_U \in \De} U
        \]
        Thus any finite subset $\De \subs \Ga$ is consistent with $T$.
        Hence $\Ga$ is finitely consistent with $T$ and so
        \linkto{compactness_for_types}{$\Ga$ is consistent with $T$}.
        \linkto{extend_to_maximal_type_zorn}{Extend $\Ga$ to a 
        maximal $n$-type} $p$.
        Then for any $U \in C$ we have 
        \[
            p \in \bigcap_{\phi_U \in \Ga_U} [\phi_U] = U
        \]
\end{itemize}
\end{proof}

\begin{eg}[A visualisation of the Stone space]
    \link{visualisation_of_Stone}

    Suppose $\Si$ is the \linkto{dfn_rings}{signature of rings}.
    Let $K$ be an algebraically closed field, viewed as a $\Si$-structure.
    Let $T = \eldiag{\Si}{K}$, and $V_n = {x_1 : A, \dots, x_n : A}$ 
    ($A$ is the only sort in the signature of rings).
    Then there is a \linkto{cont_bij_stone_to_spec}{continuous bijection} 
    from the Stone space $S_n(T)$ 
    to \linkto{prime_spec_zariski_top}{$\spec(K[x_1,\dots,x_n])$} 
    that is \textit{not} an isomorphism.
    In other words the Stone space is a finer topology on the affine space
    $\spec(K[x_1,\dots,x_n]) \iso \A^n$.
    Notice that there cannot be a isomorphism since the Stone space is 
    Hausdorff but this particular $\spec$
    \linkto{spec_not_hausdorff}{cannot be Hausdorff}.

    In this example of the Stone space, an example of a clopen set in the basis 
    would be given by ($[\star]_T$ of ) the formula 
    \[
        f(x_1,\dots,x_n) = 0 \AND g(x_1,\dots,x_n) \ne 0
    \]
    where $f$ and $g$ are 
    \linkto{terms_in_RNG_are_polynomials}{polynomials in $K(x_1,\dots,x_n)$}
    (noting that we are actually working in $\Si(K)$).
    In $\spec$, this looks like the intersection of a closed set 
    $\set{f = 0}$ and an open set $\set{g \ne 0}$.
\end{eg}

\begin{eg}
    This is a solution to exercise $4.5.2$ 
    in Marker's book \cite{marker}. 
    Let $\Si_s$ be a signature with just a single sort $A$ and just 
    a single function symbol $s : A \to A$, the `successor function'. 
    Consider the $\Si_s$-structure $\Z$ with 
    $\modintp{\Z}{s} : x \mapsto x + 1$.
    Consider $S_n(\Theory_{\Z}(\Si_s))$, the Stone spaces of the 
    \linkto{theory_of_struc}{theory of $\Z$} for each list of variables.
    Let $p \in S_n(\Theory_{\Z})$
    
    If $n = 0$ then $p = \Theory_{\Z}$ and 
    $S_n(\Theory_{\Z})$ is just a single point.
    This is true in general for any maximal theory $T$: 
    the list of variables is empty, so $p$ is a maximal $\Si_s$-theory 
    consistent with $T$. Hence $T \subs p$, 
    but $T$ is maximal so they are equal.

    Suppose $p$ is $1$-dimensional.
    \linkto{elems_of_stone_space_are_types_of_elements}{
        Elements of the Stone space are types of elements} 
    in \linkto{finite_realisation_and_embeddings}{
        elementary extensions of $\Z$},
    so there exists $\Z \to \NN$ and elementary extension 
    and $a \in \nnintp{A}$ such that $p = \nnintp{\tp}(a)$.
    If $a$ is in the image of $\Z$ then $p = \mmintp{\tp}(a)$;
    we claim this implies $p = \mmintp{\tp}(b)$ for \textit{any} $b \in \Z$.

    \begin{proof}
        WLOG $a \in \Z$. 
        We show that $+ b - a : \Z \to \Z$ is a $\Si_s$-automorphism.
        Indeed it commutes with interpretation of $\mmintp{s}$ 
        (it \textit{is} $\mmintp{s}$ applied $b - a$ times) and
        has two sided inverse $- b + a$ which is a $\Si_s$-morphism 
        for the same reason.
        \linkto{iso_imp_elem_equiv}{Isomorphisms are elementary embeddings}
        and $a$ is sent to $b$, 
        so $\phi \in \mmintp{\tp}(a)$ if and only if $\phi \in \mmintp{\tp}(b)$.
    \end{proof}
    If $a$ is not in the image of $\Z$ then we need to determine what 
    the rest of $\NN$ could look like.
    Noting that $\modintp{\Z}{s}$ is injective - 
    which can be expressed as a $\Si_s$-sentence, 
    we have that $\nnintp{s}$ is also injective since 
    the embedding is elementary.
    Intuitively the only elementary extensions of $\Z$ can be 
    disjoint unions of $\Z$ with other injective looking successor 
    interpretations, for example $Z \sqcup \N$ with the successor on $\N$ 
    interpreted as usual;
    moreover $\Z$ and $\N$ are the only (basic) $\Si_s$-structures on which 
    we have an injective looking successor function.
    
    We confirm and formalise this intuition by showing that the only other 
    cases for $p$ are when $a$ behaves like some $n \in \N$, 
    characterised by the fact that it has exactly $n - 1$ predecessors.
    Let $n \in \N$ and suppose 
    \[
        p \in 
        [\forall y : A, s^{n+1}(y) \ne x] \cap [\exists y : A, s^{n}(y) = x]
    \]
    Then we claim that $p$ is \linkto{dfn_isolated}{isolated} by 
    the conjunction of these two formulas, i.e.
    \[
        \set{p} = 
        [\forall y : A, s^{n+1}(y) \ne x] \cap [\exists y : A, s^{n}(y) = x]
    \]
    \begin{proof}
        Suppose $q \in S_1(\Theory)$ is another point in the intersection, 
        (by embedding into the disjoint union of 
        $\NN$ and the model realising $q$)
        WLOG $q = \nnintp{\tp}(b)$ for some $b \in \nnintp{A}$.
        Define the orbit of an element $x \in \nnintp{A}$ by 
        \[
            \Orb(x) = \set{y \in \nnintp{A} \st \exists k, l \in \N, 
            \NN \model{\Si_s} s^k(x) = s^l(y)}
        \]
        First we show that the substructures $\Orb(a)$ and 
        $\Orb(b)$ are isomorphic 
        substructures of $\NN$; in fact they are both isomorphic to $\N$
        with $a \mapsto n \mapsto b$.
        By symmetry we just construct $\N \to \Orb(a)$.
        Let $a_0$ be the element of $\Orb(a)$ such that 
        $\NN \model{\Si_s} s^n(a_0) = a$. Then it follows that 
        \[\Orb(a) = \Orb(a_0) = \set{s^k(a_0) \st k \in \N}\]
        and we can define the map $\N \to \Orb(a)$ by $k \mapsto s^k(a_0)$,
        which is a $\Si_s$-isomorphism.

        These isomorphisms are (possibly distinct) 
        \linkto{iso_imp_elem_equiv}{elementary embeddings} $\N \to \NN$.
        Hence $\nnintp{\tp}(a) = \modintp{\N}{n} = \nnintp{\tp}(b)$.
    \end{proof}
    Now it remains to consider the case when $p$ is not in any of 
    the above singleton sets.
    This says that for each $n \in \N$ 
    \[
        p \in [\exists y : A, s^{n+1}(y) = x] 
        \cup [\forall y : A, s^{n}(y) \ne x]
    \]
    By induction, noting that $\forall y : A, s^{0}(y) \ne x \notin p$ for the 
    base case (take $y$ to be $a$), $p$ is always in the left side of the 
    union (and never in the right side).
    Hence 
    \[
        p \in \bigcap_{n \in \N} [\exists y : A, s^{n+1}(y) = x]
    \]
    This brings us back to the first case, 
    $p = \nnintp{\tp}(a) = \modintp{\Z}{\tp}(0)$
    (this \textit{does not} imply $a$ is in the image of $\Z$;
    we could have two disjoint copies of $\Z$ for example):
    mapping $\Z$ to $\Orb(a)$ by taking $n \mapsto s^n(a)$ 
    (we can deal with the negatives using injectivity) 
    we have an isomorphism, hence an elementary embedding 
    (maybe not the original elementary embedding into $\NN$)
    indentifying $\nnintp{\tp}(a)$ with $\modintp{\Z}{\tp}(0)$.
\end{eg}

\subsection{Types on Structures and Omitting Types}

\begin{lem}[Finite realisation and embeddings \cite{buechler}] 
    \link{finite_realisation_and_embeddings}
    Let $\MM$ be a $\Si$-structure, 
    $\linkto{theory_of_struc}{A \subs \MM}$ and $p \subs F(\Si(A),V_n)$
    an $n$-dimensional theory.
    Then the following are equivalent 
    \begin{itemize}
        \item $p$ is consistent with $\eldiag{\Si}{\MM}$.
        \item There exists an elementary embedding $\MM \to \NN$ 
            such that $p$ is realised in $\NN$.
        \item There exists an elementary embedding $\MM \to \NN$ 
            such that $p$ is finitely realised in $\NN$.
        \item $p$ is finitely realised in $\MM$.
    \end{itemize}
    The elementary embeddings can be seen as both $\Si$-embeddings or 
    $\Si(A)$-embeddings for any subset $A \subs \MM$.
\end{lem}
\begin{proof}
    $(1. \implies 2.)$ There exists $\NN$ and a
    $b \in \prod_{i = 1}^n \nnintp{A_i}$ such that 
    $\NN \model{\Si(\MM)} \eldiag{\Si}{\MM}$ and 
    $\NN \model{\Si(\MM)} p(b)$.
    Then since \linkto{elem_ext_equiv_eldiag_model}{models of the 
    elementary diagram correspond to elementary extensions}, 
    there exists an elementary $\Si(\MM)$-embedding $\MM \to \NN$
    (which can be moved down to being a $\Si(A)$-embedding for any 
    subset $A \subs \MM$.) 

    $(2. \implies 3.)$ Clear.

    $(3. \implies 4.)$ Let $\De \subs p$ be finite.
    Then by assumption there exists an elementary $\Si$-embedding 
    $\io : \MM \to \NN$ and $b \in \prod_{i = 1}^n \nnintp{A_i}$ 
    such that  
    $\NN \model{\Si} \De(b)$. 
    Hence 
    \[
        \NN \modelsi \exists v : \prod_{i = 1}^n A_i, 
        \bigand{\psi \in \De}{} \psi(v) 
        \quad \implies \quad 
        \MM \modelsi \exists v : \prod_{i = 1}^n A_i, 
        \bigand{\psi \in \De}{} \psi(v)
    \]
    by the embedding being elementary. 
    Hence there exists $a \in \prod_{i = 1}^n \mmintp{A_i}$ 
    realising $\De$.

    $(4. \implies 1.)$
        By \linkto{compactness_for_types}{compactness for types},
        it suffices to show that $p$ 
        is finitely consistent with the elementary diagram.
        Let $\De \subs p$ be finite.
        Then by assumption there is $a \in \prod_{i=1}^n \mmintp{A_i}$ such that 
        $\MM \model{\Si} \De(a)$ and so $\MM \model{\Si(\MM)} \De(a)$.
        Clearly $\MM$ is a model of its elementary diagram.
\end{proof}

We note again that \linkto{fin_realised_not_realised}{finitely realised does 
not imply realised} in general.

\begin{dfn}[Isolated types]
    \link{dfn_isolated}
    Let $p \subs F(\Si,V_n)$ be an $n$-dimensional theory and 
    $\phi \in F(\Si,V_n)$.
    Then $\phi$ isolates $p$ with respect to $\Si$-theory $T$ if 
    \begin{itemize}
        \item $\phi$ is consistent relative to $T$ and 
        \item For each $\psi \in p$ 
        \[
            T \modelsi \forall v : \prod_{i = 1}^n A, \phi \to \psi
        \]
    \end{itemize}
    We say $p$ is isolated if there exists such a $\phi$ 
    (some also say $p$ is principal).
\end{dfn}

\begin{prop}[Isolation topologically]
    If $p \in S_n(T)$ then $p$ is isolated by $\phi$ if and only if 
    $\set{p} = [\phi]_T$ - it is an `isolated point'. 
\end{prop}
\begin{proof}
    \begin{forward}
        By maximality $\phi \in p$ and $p \in [\phi]_T$.
        Let $q \in [\phi]_T$. For any element $\psi \in p$ 
        we have 
        \[T \modelsi \forall v, \phi \to \psi\]
        \linkto{basic_facts_basis_elems}{hence $q \in [\phi]_T \subs [\psi]_T$}
        and $\psi \in q$.
        Hence $p \subs q$ and by maximality $p = q$.
    \end{forward}

    \begin{backward}
        Let $\psi \in p$
        \[
            [\phi \to \psi] \linkto{basic_facts_basis_elems}{=} 
            [\NOT \phi] \cup [\psi] \linkto{basic_facts_basis_elems}{=} 
            \brkt{S_n(T) \setminus \set{p}} \cup [\psi] = S_n(T)
        \]
        \linkto{basic_facts_basis_elems}{Hence} 
        $T \modelsi \forall v, \phi \to \psi$.
    \end{backward}
\end{proof}

\begin{prop}[Isolated types are realised in every model of a complete theory]
    Any finite $n$-type on a 
    \textit{complete} theory
    is realised in every model of the theory.
    Hence any $n$-type is finitely realised in every model of the theory.
    Hence an $n$-type isolated with respect to a \textit{complete} 
    theory is realised in every model of the theory.
\end{prop}
\begin{proof}
    By taking the conjunction of formulas 
    it suffices to show that a single formula $\phi$ consistent with complete
    $\Si$-theory $T$ is realised in any model $\MM$ of $T$, or equivalently
    \[T \modelsi \exists v, \phi\]
    By completeness of $T$ either the above holds or 
    \[T \modelsi \forall v, \NOT \phi\]
    By consistency of $\phi$ with $T$ there exists a 
    model which shows that the latter cannot hold.
\end{proof}
\begin{rmk}
    To demonstrate that completeness of $T$ is required we take the example 
    of the incomplete \linkto{dfn_rings}{theory of rings} as $T$, 
    the $1$-type to be $\set{x^2 + 1 = 0}$ and the model in question to be $\Z$.
    Clearly the $1$-type is realised in $\Z[i]$ but not in $\Z$.
\end{rmk}

\begin{dfn}[Substructure generated by a subset]
    \link{substructure_gen_by_subset}
    Let $\MM$ be a $\Si$-structure.
    Let \linkto{theory_of_struc}{$X \subs \MM$}.
    Then the following are equivalent definitions for a substructure $\<A\>$
    of $\MM$ generated by $X$:
    \begin{itemize}
        \item The $\Si$-structure $\<X\>$ defined inductively such that 
        $X \subs \<X\>$ and if $f : \prod A \to B \in \func{\Si}$ and 
        $a \in \prod \modintp{\<X\>}{A}$ then 
        $\mmintp{f}(a) \in \modintp{\<X\>}{B}$.
        \item $\bigcap \set{\NN \text{ substructure of } \MM \st X \subs \NN}$, 
            where by the intersection we mean intersecting 
            in each interpreted sort.
    \end{itemize}
    We say a substructure is finitely generated if there exists a finite set 
    $X$ such that it is equal to $\<X\>$.
\end{dfn}
\begin{proof}
    We fix the first definition as $\<X\>$ and call the second $\cap$
    with each sort of the intersection as $\cap_A$.
    We show $\<X\>$ forms a substructure of $\MM$ containing $X$:
    By definition $\modintp{\<X\>}{f} := \mmintp{f}$ is well defined.
    Each relation $r \hookr \prod A$ is naturally 
    interpreted as the pullback 
    \[ 
        \mmintp{r} \cap \modintp{\<X\>}{\prod A}
    \]
    Hence $\<X\>$ is a substructure of $\MM$ containing $X$.
    Furthermore the second definition is a contained in $\<X\>$.

    For the other direction note that if $b \in \modintp{\<X\>}{B}$ 
    was built via the first definition then $b \in X_B$ or
    $b = \mmintp{f}(a)$ for some tuple $a \in \prod \modintp{\<X\>}{A}$.
    In the first case it is trivially in any substructure containing $X$.
    In the second case any substructure $\MM$ containing $X$ 
    is closed under $\nnintp{f} = \mmintp{f}$ and by induction
    $a \in \prod \nnintp{A}$ for any substructure $\NN$.
    Hence $b = \mmintp{f}(a) \in \NN$ for any substructure.
    Thus $\<A\> \subs \bigcap \NN$ and we are done.
\end{proof}

\begin{lem}[Proofs are finite]
    \link{proofs_are_finite}
    Suppose $T$ is a $\Si$-theory and $\phi$ a $\Si$-sentence such that 
    $T \model{\Si} \phi$. 
    Then there exists a finite subset $\De$ of $T$ such that 
    $\De \model{\Si} \phi$.
\end{lem}
\begin{proof}
    We show the contrapositive.
    Suppose for all finite subsets $\De$ of $T$, $\De \nodel{\Si} \phi$,
    then each $\De \cup \set{\NOT \phi}$ is consistent and
    by \linkto{compactness}{compactness} $T \cup \set{\NOT \phi}$ is consistent.
    Hence $T \nodel{\Si} \phi$.
\end{proof}

The omitting types theorem is a kind-of converse to 
`an $n$-type isolated with respect to a \textit{complete} 
theory is realised in every model of the theory.'

\begin{prop}[Omitting types]\cite{henson}
    Let $\Si$ be a countable signature, $T$ a countable $\Si$-theory 
    and $p$ a non-isolated $n$-type on $T$.
    Then there exists a countable $\Si$-model of $T$ that omits $p$.
\end{prop}
\begin{proof}
    Write $V_n = \set{x_1 : A_1, \dots, x_n : A_n}$.
    Let $\Si_C$ be the signature with symbols from $\Si$ 
    plus a countably infinite set of constant symbols $C_A$ 
    for each sort $A$.
    We build a $\Si_C$-theory $T_C$ and a 
    countable model of $T_C$ that omits $p$. 
    Our $T_C$ will satisfy the following three conditions
    \begin{itemize}
        \item $T \subs T_C$ is maximal and consistent
        \item (Witness) For any formula $\phi$ with only one free variable 
        $v : A$ there exists $c \in C_A$ such that if the sentence 
        $\exists v : A,\phi(v)$ is in the theory $T_C$ then $\phi(c) \in T_C$.
        \item (Omitting) For any tuple $c \in C_{A_1} \tdt C_{A_n}$ there exists 
        $\phi \in p$ such that $\NOT \phi(c) \in T_C$.
    \end{itemize}
    Given such a $T_C$ we build our omitting model $\NN$.
    By consistency of $T_C$ we have a model $\MM$, 
    from which we can take the collection $X$ of interpreted constant symbols:
    for each $A$ let 
    \[
        X_A := \set{\mmintp{c} \in \mmintp{A} \st c \in C_A}
    \]
    We show that $\NN := \<X\>$
    \linkto{substructure_gen_by_subset}{the substructure it generates}
    has sorts interpreted as $\nnintp{A} = X_A$.
    Suppose $f : \prod A \to B$ is a function symbol from $\Si$ and 
    $c \in \prod A$ represents an element of $\prod X_A$
    then since $T_C$ is maximal either $\exists v : B, f(c) = v$ 
    or its negation is in $T_C$, 
    but consistency of $T_C$ implies the latter is false.
    Hence $\exists v : B, f(c) = v$ is in $T_C$ and so by the witness property 
    there exists $d \in C_B$ such that $f(c) = d$ is in $T_C$.
    In particular $\mmintp{f}(\mmintp{c}) \in \NN$.
    We interpret relation symbols as the pullback and we have a 
    $\Si_C$-embedding $\NN \to \MM$.

    To show $\NN$ is a $\Si_C$-model of $T_C$ it suffices that the 
    embedding is elementary, which we show by 
    \linkto{tarski_vaught}{Tarski-Vaught}:
    suppose $\phi$ is a $\Si_C$-sentence preserved by the inclusion.
    If $\exists v : A, \phi$ is a $\Si_C$-sentence satisfied by $\MM$ then 
    by maximality of $T_C$ we have $\exists v : A, \phi \in T_C$ and so there 
    exists $c \in C$ such that $\phi(c) \in T_C$.
    Since $\phi$ is preserved by the inclusion, 
    $\MM \model{\Si_C} \phi(c)$ if and only if $\NN \model{\Si_C} \phi(c)$,
    concluding Tarski-Vaught.
    Furthermore, any tuple of elements in $\prod \nnintp{A_i}$ 
    are represented by some $c \in \prod C_n$, 
    hence by the last property of $T_C$
    we have some $\phi \in p$ such that $\NN \nodel{\Si_C} \phi(c)$. 
    So $\NN$ is a countable (as $\Si$ and each $C_A$ is countable) 
    $\Si$-model of $T$ that omits $p$.

    It remains to construct such a theory.
    Let $T_0 := T$, and we build $\Si_C$-theory $T_{i+1}$
    from $\Si_C$-theory $T_i$,
    guaranteeing that at each step 
    \textit{only finitely many formulas are added}.
    Enumerate the (countable) set of $\Si_C$-sentences by 
    $\set{\phi_i}_{i \in \N}$;
    enumerate the (countable) set of tuples $C_1 \tdt C_n$ 
    by $\set{c_i}_{i \in \N}$;
    enumerate the (countable) set of $\Si$-formulas $\psi$ in a single variable 
    $v : A$ such that $\exists v : A, \psi \in T$ by $\set{\psi_i}_{i \in \N}$.

    \textit{Maximality}:
    At least one of $T_i \cup \set{\phi_i}$ and $T_i \cup \set{\NOT \phi_i}$ 
    is a consistent $\Si_C$-theory
    (since $T_i$ is consistent)
    so we take the consistent one to be $T^1$.
    If the added formula is equivalent to $\exists v : A, \psi(v)$ 
    where $\psi$ has exactly one
    free variable $v : A$ then we take $T^2 := T' \cup \set{\psi(c)}$ 
    for some $c \in C_A$ that has not appeared yet in $T^1$,
    otherwise leaving $T^2 := T^1$.  

    \textit{Witness}: We want the witness property for $\psi_i$.
    We pick a constant symbol $c \in C_A$ that has not appeared in 
    $T^2$ and let $T^3 = T^2 \cup \set{\psi_i(c)}$. 
    This is still consistent since $T^2$ has a model, which is also a model of 
    $T$, hence $c$ can be interpreted in a sensible way.
    
    \textit{Omitting}: We wish find $\de \in p$ such that 
    $T^3 \cup \set{\NOT \de(c_i)}$ is consistent, which
    (by chucking $\de(c_i)$ in at each step) 
    allows us to omit $p$ once the induction is complete.
    Since $T^3$ is an extension of $T$ by finitely many formulas we can 
    take their conjunction to find $\Si_C$-sentence $\chi(d) \in \Si_C$ 
    where $\chi$ is a $\Si$-formula and $d$ represents all the constant symbols 
    from $C$ appearing in the conjunction.
    Assume for a contradiction for every $\de \in p$ we have 
    $T^3 \cup \set{\NOT \de(c_i)}$ is inconsistent.
    Then for each $\de$,
    \[
        T^3 \model{\Si_C} \de(c_i) \quad \implies \quad
        T \model{\Si_C} \chi(d) \to \de(c_i)
    \]
    By the \linkto{lemma_on_const}{lemma on constants} this implies
    for every $\de \in p$
    \[
        T \modelsi \forall x : \prod A_i, \forall v, \chi(x,v) \to \de(x)
        \quad \implies \quad
        T \modelsi \forall x : \prod A_i, (\exists v, \chi (x,v)) \to \de(x)
    \]
    where $x$ are variables corresponding to $c_i$ and $v$ makes up to the rest
    appearing in $d$.
    This implies that $p$ is isolated by $\exists v, \chi (x,v)$,
    which is a contradiction.
    Hence we can extend $T_{i+1} := T^3 \cup \set{\NOT \de(c_i)}$
    for some $\de \in p$. 
    Again, if the added formula is equivalent to $\exists v : A, \psi(v)$ 
    where $\psi$ has exactly one free variable $v : A$ then we 
    add in a formula to compensate for this.

    Now we take $T_C := \bigcup_{i \in \N} T_i$.
    It is maximal and it is finitely consistent 
    (each finite subset is in some $T_i$, which is consistent)
    \linkto{compactness}{hence} consistent.
    The construction guaranteed $T_C$ having the witness property and 
    omitting $p$.
\end{proof}