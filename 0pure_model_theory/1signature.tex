\section{Basics}
This section follows Marker's book on Model Theory 
\cite{marker} but with more 
emphasis on where things are happening, i.e. what signature we are working in,
and also working with many-sorted signatures.
For a quick introduction to model theory we recommend reading up to where 
morphisms are introduced.
Then it is worth following routes to particular results such as quantifier
elimination in algebraically closed fields or Morley Rank being Krull dimension,
which may help to motivate the rest of the theory.

\subsection{Signatures}
\begin{dfn}[Signature]
    A signature \footnote{
        Many authors call $\Si$ a \textit{language} instead.
    }$\Si = (S,F,R)$ consists of 
    \begin{itemize}
        \item $S$ a set of \textit{sort} symbols.
        \item $F$ a set of \textit{function} symbols.
            Each function symbol $f \in F$ has \textit{arity} 
            $(A_1,\dots,A_n,B)$, a tuple of sort symbols in $S$.
            We write $f : A_1 \times \dots \times A_n \to B$
            to denote $f$ with its arity.
        \item $R$ a set of \textit{relation} symbols.
            Each relation symbol $r \in R$ has \textit{arity}
            $(A_1,\dots,A_n)$, a tuple of sort symbols in $S$.
            We write $r \hookr A_1 \times \dots \times A_n$ 
            to denote $r$ with its arity.
    \end{itemize}
    Given a signature $\Si$, we may refer to its sort, 
    function and relation symbol sets as $\sort{\Si},\func{\Si},\rel{\Si}$.
    We should think of function arity as the source and target of a function
    and the relation arity as the product that the relation is a subset of.
    
    For convenience we single out $0$-ary functions  
    - $c \in F$ with arity $(B)$ - and
    call them \textit{constant} symbols.
    For each sort $B$ we label the set of such constant symbols
    $\const{B}$ and
    think of these as 'elements' of the sort $B$ and write $c : B$.
    We denote the set of all constant symbols in a signature by
    \[\const{\Si} := \bigcup_{A \in \sort{\Si}} \const{A}\]
\end{dfn}

    Our signatures are multi-sorted, meaning we can have elements and 
    variables living in different spaces. 
    For example groups, rings and partial orders are all 1-sorted with just one
    underlying set, whereas group actions and modules are 2-sorted.
    If we are working in the 1-sorted case we may not mention the sort at all,
    because there is only one.
    
\begin{eg}
    The \linkto{dfn_rings}{signature of rings} 
    will be used to define the theory of rings, 
    the theory of integral domains, the theory of fields, and so on.
    The signature with 
    \linkto{sig_just_one_bin_rel}{just a single binary relation} 
    will be used to define the \linkto{order_theories}{theory of partial orders}
    with the interpretation of the relation as $<$, 
    to define the theory of equivalence relations with the 
    interpretation of the relation as $\sim$,
    and to define the theory $\ZFC$ with the relation interpreted as $\in$.
    The \linkto{dfn_modules}{signature of modules} %% Missing link
    will be used to define the theory of modules,
    the theory of vector spaces, and so on.
\end{eg}

\begin{dfn}[$\Si$-terms]
    Let $\Si$ be a signature. 
    For each sort $A \in \sort{\Si}$ we create 
    $\var{A}$, a countable set of \textit{variable symbols} of type $A$, 
    ($\var{A}$ and $\var{B}$ are disjoint for distinct sorts).
    
    We inductively define the set of terms
    $\term{\Si}$ with their types and their 
    associated set of typed variables $\tv{\star}$. 
    We write $t : A$ to mean $t$ is a $\Si$-term of type $A$.
    \begin{itemize}
        \item[$\vert$] If $A$ is a sort and $v \in \var{A}$ then 
        $v : A$ is a $\Si$-term.
        It has typed variables $\tv{v} = \set{v : A}$.
        \item[$\vert$] If $f \in \func{\Si}$ is a function symbol 
            $f : A_1 \times \dots \times A_n \to B$ and for each 
            $1 \leq i \leq n$ we have 
            $t_i : A_i$.
            We introduce notation 
            $t = (t_1,\dots,t_n) : A_1 \times \dots \times A_n$.
            Then $f(t) : B$ is a $\Si$-term.
            It has typed variables $\tv{f(t)} = \set{t_i : A_i}_{i=1}^n$.
    \end{itemize}
    
    If there are repeated sorts we might write 
    $A^n$ instead of $A \times \dots \times A$.
    Note that given no terms, constant symbols $c : B$ are terms of type $B$.
\end{dfn}

\begin{eg}
    \link{terms_as_polys_example}
    In the signature of rings, 
    \linkto{terms_in_RNG_are_polynomials}{terms will be multivariable
        polynomials over $\Z$} 
    since they are sums and products of constant symbols $0, 1$ and 
    variable symbols.
    In the signature of modules, 
    terms will be finite linear combinations,
    where the coefficients are polynomials over $\Z$.
    In the \linkto{sig_just_one_bin_rel}{signature of binary relations} 
    there are no constant or function symbols so the only terms are variables.
\end{eg}

\begin{dfn}[Interpretation and structures]
    Let $\CC$ be a category with finite products\footnote{
        If you prefer you can just take $\CC$ to be $\SET$
        the category of sets.}.
    Given a signature $\Si$, a $\Si$-\textit{structure} $\MM$ 
    - or an interpretation $\intp{\Si}{\MM}$ 
    (which we just write as $\mmintp{\star}$ here)
    of the signature in $\CC$ is the following
    \begin{itemize}
        \item Each sort symbol $A \in \sort{\Si}$ 
        is interpreted as an object $\mmintp{A}$ of $\CC$.
        \item Each function symbol $f : A_1 \times \dots \times A_n \to B$ 
        is interpreted as a morphism from the product in $\CC$
        \[\mmintp{f} : \mmintp{A_1} 
        \times \dots \times \mmintp{A_n}  \to \mmintp{B}\]
        \item Each relation symbol $r \hookr A_1 \times \dots \times A_n$
        is interpreted as a sub-object of the product in $\SET$
        \[\mmintp{r} \hookr \mmintp{A_1} 
        \times \dots \times \mmintp{A_n}\]
    \end{itemize}
\end{dfn}

The structures in a signature will become the models of theories.
For example $\Z$ is a structure in the \linkto{dfn_rings}{signature of rings}, 
a model of the theory of rings but not a model of the theory of fields.
In the signature of \linkto{sig_just_one_bin_rel}{binary relations}, 
$\N$ with the usual ordering $\leq$ is a structure that models of 
the theory of partial orders but not the theory of equivalence relations.

\begin{dfn}[$\Si$-morphism, $\Si$-embedding]
    \link{category_of_structures}
    The collection of all $\Si$-structures over a category $\CC$ 
    forms a category
    denoted by $\struc{\Si, \CC}$, which has objects
    as $\Si$-structures and morphisms as $\Si$-morphisms:

    Suppose $\MM, \NN$ are $\Si$-structures in $\CC$ 
    and for each sort symbol $A$
    we have $\io_A : \mmintp{A} \to \nnintp{A}$.
    We say $\io := (\io_A)_{A \in \sort{\Si}}$ is
    a \textit{$\Si$-morphism} from $\MM$ to $\NN$ when 
    \begin{itemize}
        \item For all function symbols $f : A_1 \tdt A_n \to B$ this diagram 
        commutes:
        \begin{cd}
            \mmintp{A_1} \tdt \mmintp{A_n}
            \ar[r, "\mmintp{f}"] \ar[d, "\io_{A_1} \tdt \io_{A_n}", swap]
            & \mmintp{B} \ar[d, "\io_B"]\\
            \nnintp{A_1} \tdt \nnintp{A_n}
            \ar[r, "\nnintp{f}"] & \nnintp{B}
        \end{cd}
        \item For all relation symbols $r \hookr A_1 \tdt A_n$ 
            there exists a morphism $\io_r : \mmintp{r} \to \nnintp{r}$ 
            such that
        \begin{cd}
            \mmintp{r}
            \ar[hookrightarrow]{r} \ar[d, "\io_r", swap]
            & \mmintp{A_1} \tdt \mmintp{A_n}
            \ar[d, "\io_{A_1} \tdt \io_{A_n}"]\\
            \nnintp{r}
            \ar[hookrightarrow]{r} & \nnintp{A_1} \tdt \nnintp{A_n}
        \end{cd}
    \end{itemize}
    $\io$ is called a $\Si$-embedding (and $\NN$ a $\Si$-extension)
    when each $\io_A$ is a monomorphism 
    and $\mmintp{r}$ is the pullback of the last diagram.
    Restricting morphisms between objects to purely $\Si$-embeddings
    results in a subcategory of $\struc{\Si,C}$.
\end{dfn}
The notion of morphisms here will be the same as that of
morphisms in the algebraic setting.
For example in the \linkto{dfn_grp}{signature of groups}, 
preserving interpretation of function symbols says 
the identity is sent to the identity, multiplication is preserved,
and inverses are sent to inverses.

\begin{dfn}[Interpretation of terms]
    \link{interpretation_terms}
    Given a $\Si$-structure $\MM$ in category $\CC$ and a $\Si$-term 
    $t : B$ with 
    $\tv{t} = \set{v_1 : A_1, \dots, v_n : A_n}$.
    Then we can naturally interpret $t$ in the $\Si$-structure $\MM$ as a 
    morphism in $\CC$
    \[
        \mmintp{t} : \mmintp{A_1} \tdt \mmintp{A_n} \to \mmintp{B}
    \]
    that commutes with the interpretation of function symbols
    (see proof).
    We then refer to this map as \emph{the} 
    interpretation of the term $t$.\footnote{
        In type theory this is can be seen as the function type
        \[
            \la v_1 : A_1, \dots \la v_n : A_n, \mmintp{t} : 
            \mmintp{A_1} \tdt \mmintp{A_n} \to \mmintp{B}
        \]}

    The following notation will be used for the interpretation of a tuple 
    of terms (which is the product of their interpretations)
    \[
        s = (s_1,\dots,s_n) \text{ gives us a map }\mmintp{s} 
        := \mmintp{s_1} \tdt \mmintp{s_n}
    \]
\end{dfn}
\begin{proof}
    We use the inductive definition of $t$:
    \begin{itemize} 
        \item If $t$ is a variable $v : A$
        then $\tv{t} = \set{v : A}$ and $t : A$.
        Thus it makes sense to define $\mmintp{t} : \mmintp{A} \to \mmintp{A}$ 
        to be the identity.
        \item Suppose $t = f(s) : C$ for some function symbol 
        $f : B_1 \tdt B_n \to C$ and terms
        $s : B_1 \tdt B_n$. 
        Since 
        $\tv{t} = \bigcup_{i=1}^n \tv{s_i}$
        we must have 
        $\mmintp{t} : \prod_{x : A \in \tv{t}} \mmintp{A} \to \mmintp{C}$. 
        Induction gives each $\mmintp{s_i}$, 
        and there is naturally the map 
        \[
            \mmintp{s} := \mmintp{s_1} \tdt \mmintp{s_n} : 
            \prod_{x : A \in \tv{t}} 
            \mmintp{A} \to \mmintp{B_1} \tdt \mmintp{B_n}
        \]
        so we define the interpretation of $t$
        to be the composition $\mmintp{t} := \mmintp{f} \circ \mmintp{s}$.
    \end{itemize}
\end{proof}
Note that for constant symbols the interpretation has the empty product 
- a `singleton' - as its domain hence is a constant map -
an `element' of the interpreted sort.

\begin{eg}
    Following the \linkto{terms_as_polys_example}{previous example}, 
    in the signature of rings
    \linkto{terms_in_RNG_are_polynomials}{terms will be multivariable
        integer polynomials} 
    and then terms interpreted in some structure - say a ring $A$ - 
    are then multivariable integer polynomial functions
    $A^n \to A$.
\end{eg}

The following is exactly what we expect - that terms are well behaved 
with respect to morphisms.
\begin{prop}[$\Si$-morphisms commute with interpretation of terms]
    \link{morph_comm_term_intp}
    Given a $\Si$-morphism $\io : \MM \to \NN$, 
    we have that for any $\Si$-term $t$ 
    with $\tv{t} = \set{v_1 : A_1, \dots, v_n : A_n}$
    this diagram commutes
    \begin{cd}
        \mmintp{A_1} \tdt \mmintp{A_n} \ar[r, "\mmintp{t}"]
        \ar[d, "\io_{A_1} \tdt \io_{A_n}", swap] & 
        \mmintp{B} \ar[d, "\io_B"]\\
        \nnintp{A_1} \tdt \nnintp{A_n} \ar[r, "\nnintp{t}"]
        &
        \nnintp{B}
    \end{cd}
\end{prop}
\begin{proof}
    We case on what $t$ is:
    \begin{itemize}
        \item If $t$ is a variable $v : A$ then 
        $\mmintp{t} = \id{\mmintp{A}}$ and $\nnintp{t} = \id{\nnintp{A}}$
        hence the following commutes
        \begin{cd}
            \mmintp{A}  \ar[r, "\mmintp{t}"]
            \ar[d, "\io_{A}", swap] & 
            \mmintp{A} \ar[d, "\io_A"]\\
            \nnintp{A} \ar[r, "\nnintp{t}"]
            &
            \nnintp{A}
        \end{cd}
        \item Suppose $t = f(s) : C$ for some function symbol 
        $f : B_1 \tdt B_n \to C$ and terms
        $s : B_1 \tdt B_n$. 
        We consider the diagram
        \begin{cd}
            {\prod_{x : A \in \tv{t}}\mmintp{A}} & {} & {\mmintp{B}} \\
            & {\mmintp{B_1}\tdt \mmintp{B_n}} \\
            {\prod_{x : A \in \tv{t}}\nnintp{A}} && {\nnintp{B}} \\
            & {\nnintp{B_1}\tdt \nnintp{B_n}}
            \ar["{\mmintp{t}}", from=1-1, to=1-3]
            \ar["{\nnintp{t}}"{description, pos=0.2}, from=3-1, to=3-3]
            \ar["{\prod\io_A}", from=1-1, to=3-1, swap]
            \ar[
                "{\mmintp{s_1}\tdt \mmintp{s_n}}"{description}, 
                from=1-1, to=2-2]
            \ar[
                "{\nnintp{s_1}\tdt \nnintp{s_n}}"{description}, from=3-1, 
                to=4-2]
            \ar["{\nnintp{f}}"{description}, from=4-2, to=3-3]
            \ar["{\mmintp{f}}"{description}, from=2-2, to=1-3]
            \ar[
                "{\io_{B_1} \tdt \io_{B_n}}"{description, pos=0.2}, 
                from=2-2, to=4-2, crossing over]
            \ar["{\io_B}", from=1-3, to=3-3]
        \end{cd}
        Induction gives that the left face commutes and the definition 
        of interpretation gives us that the top and bottom faces commute.
        By definition of interpretation of function 
        symbols the right face commutes.
        From this we can deduce that the background of the following diagram 
        commutes, as required. 
    \end{itemize}
\end{proof}

\begin{dfn}[Models in different signatures]
    \link{move_down_mod_0}
    Given two signatures
    $\Si \leq \Si^*$ and
    a $\Si^*$-structure $\MM$ is naturally a $\Si$-structure $\NN$ such that 
    $\subintp{\Si}{\MM}{\star} = \subintp{\Si^*}{\NN}{\star}$.
    We will often just write $\MM$ to mean either.
\end{dfn}

\begin{prop}[Morphisms in different signatures]
    \link{morph_diff_sigs}
    Suppose $\Si \leq \Si^*$
    and we have $\Si^*$-structures $\MM$ and $\NN$ 
    (\linkto{move_down_mod_0}{naturally $\Si$-structures}).
    If $\io : \MM \to \NN$ is a morphism in $\CC$ then
    $\io$ is a $\Si$-morphism if and only if it is a $\Si^*$-morphism, and
    an embedding in $\Si$ if and only if in $\Si^*$.
\end{prop}
\begin{proof}
    If $f : \prod A \to B$ is a function symbol 
    then $\prod \subintp{\Si}{\MM}{A} = \prod \subintp{\Si^*}{\MM}{A}$
    and $\subintp{\Si}{\MM}{B} = \subintp{\Si^*}{\MM}{B}$ 
    (similarly for $\NN$).
    Hence the commuting diagram is the same diagram.
    Similarly the commuting diagram is the same for any relation symbol.
    Giving us the result.
\end{proof}

\subsection{Formulas in classical first order logic}
From this point onwards we will work in classical first order logic.

\begin{dfn}[Atomic formula, quantifier free formula]
    Given $\Si$ a signature, 
    its set of atomic $\Si$-formulas along with their 
    associated set of typed free variables $\tv{\star}$ is defined by
    \begin{itemize}
        \item[$\vert$] $\top$ is an atomic $\Si$-formula
        with no typed free variables $\tv{\top} = \nothing$.
        \item[$\vert$] 
        Given $t, s \in \term{\Si}$ of the same type, 
        the string $t = s$ is an atomic $\Si$-formula.
        The set of typed free variables is 
        $\tv{t = s} = \tv{t} \cup \tv{s}$.
        \item[$\vert$] Given $r \in \rel{\Si}$ such that
        $r \hookr A_1 \times \dots A_n$ and $t : A_1 \times \dots \times A_n$,
        the string $r(t)$ is an atomic $\Si$-formula. 
        The typed free variables are $\tv{r(t)} = \bigcup_{i = 1}^n \tv{t_i}$.
    \end{itemize}
    
    We extend this to the set of quantifier free $\Si$-formulas inductively:
    \begin{itemize}
        \item[$\vert$] Given $\phi$ an atomic $\Si$-formula,
            $\phi$ is a quantifier free $\Si$-formula.
        \item[$\vert$] Given $\phi$ a quantifier free $\Si$-formula, 
        the string $\NOT \phi$ is a quantifier free $\Si$-formula with the 
        same set of typed free variables as $\phi$. 
        \item[$\vert$] Given $\phi, \psi$ both quantifier free $\Si$-formulas, 
        the string $\phi \OR \psi$ is a quantifier free $\Si$-formula.
        The set of typed free variables is 
        $\tv{\phi \OR \psi} = \tv{\phi} \cup \tv{\psi}$.
    \end{itemize}
    
    The set of all $\Si$-formulas $\form{\Si}$ 
    is defined by extending the above:
    \begin{itemize}
        \item[$\vert$] Given $\phi$ a quantifier free $\Si$-formula,
        $\phi$ is a $\Si$-formula.
        \item[$\vert$] Given $\phi \in \form{\Si}$, a sort $A$
        and a variable $v : A$, we take the string $\forall v : A, \phi$ and
        replace all occurrences of
        $v$ with an unused symbol
        $z \in \var{A}$,
        producing a new a $\Si$-formula $\psi = \forall z : A, \phi$.
        The set of typed free variables is 
        $\tv{\psi} = \tv{\phi} \setminus \set{v : A}$.
    \end{itemize}
    Shorthand for some $\Si$-formulas include 
    \begin{itemize}
        \item $\bot := \NOT \top$
        \item $\phi \land \psi := \NOT((\NOT\phi) \lor (\NOT\psi))$
        \item $\phi \to \psi := (\NOT\phi) \lor \psi$
        \item $\exists v : A, \phi := \NOT(\forall v : A, \NOT\phi)$
    \end{itemize}
    
    The symbol $z$ is thought of as a `bounded variable' as oppose to 
    a `free variable', 
    and will not be considered when we want to evaluate
    variables in formulas.
    
    Often we will be in 1-sorted signatures, 
    in which case we will just write 
    \[
        \forall v, \phi \text{ and } \exists v, \phi
    \]
    since the type is obvious.
\end{dfn}
\begin{rmk}
    Formulas should be thought of as propositions with some bits loose, 
    namely the free variables, since it does not make any sense to ask if 
    $x = a$ without saying what $x$ you are taking
    (where $x$ is a variable as $a$ is a constant symbol, say).
    When there are no free variables we get what intuitively looks like a 
    proposition, and we will call these particular formulas \linkto{sentences}{sentences}.
\end{rmk}

\begin{dfn}[Substituting in terms]
    If a $\Si$-formula $\phi$ has a typed free variable $v : A$ 
    then to remind ourselves of the variable we can write 
    $\phi = \phi(v : A) = \phi(v)$ instead.

    If we have term $t : A$ for some typed free variable $v : A$
    in $\tv{\phi}$, 
    then we write $\phi(t)$ to mean $\phi(v : A)$ with 
    $t$ substituted for $v$.
    By induction on terms and formulas this is still a 
    $\Si$-formula and has 
    \[\tv{\phi(t)} = \tv{t} \cup \brkt{\tv{\phi(v)} \setminus \set{v : A}}\]
\end{dfn}

\subsection{Classical models, theories}
From now on we will be interpreting only in the category $* / \SET$
of non-empty sets.
We require non-emptiness because the classical proof of compactness,
given by a \linkto{make_wit}{Henkin construction}
relies on non-emptiness of our models.

\begin{dfn}[Satisfaction]
    Let $\MM$ be a $\Si$-structure (interpreted in $* / \SET$
    and $\phi$ a $\Si$-formula.
    Let $a \in \prod_{x : A \in \tv{\phi}} \mmintp{A}$ be a tuple indexed by
    the typed free variables of $\phi$.
    Then we define $\MM \modelsi \phi(a)$ by induction on formulas:
    \begin{itemize}
        \item If $\phi$ is $\top$ then $\MM \model{\Si} \phi$.\footnote{
            We can omit the $a$ when there are no free variables.
            Formally $a$ is the unique element in the empty product.}
        \item If $\phi$ is $t = s$ then
            $\MM \model{\Si} \phi(a)$ when
            $\modintp{\MM}{t}(a) = \modintp{\MM}{s}(a)$.
            \item If $\phi$ is $r(t)$,
            where $r \hookr A_1 \tdt A_n$ is a relation symbol and
            $t : A_1 \tdt A_n$ are terms,
            then $\MM \model{\Si} \phi(a)$ when
            $\modintp{\MM}{t}(a) \in \modintp{\MM}{r}$.\footnote{
            $\modintp{\MM}{t}$ was the
            \linkto{interpretation_terms}{product of interpreted terms}.
            }
            \vspace{1em}
        \item If $\phi$ is
            $\NOT\psi$ for some $\psi \in \form{\Si}$,
            then $\MM \model{\Si} \phi(a)$ when $\MM \nodel{\Si} \psi(a)$
        \item If $\phi$ is  $\psi \lor \chi$,
            then $\MM \model{\Si} \phi(a)$ when
            $\MM \model{\Si} \psi(a)$ or $\MM \model{\Si} \chi(a)$.
        \item If $\phi$ is
            $\forall v : A, \psi$,
            then $\MM \model{\Si} \phi(a)$
            if for any $b \in \mmintp{A}$,
            $\MM \model{\Si} \psi(a,b)$.
    \end{itemize}
\end{dfn}

\begin{rmk}
    Any $\Si$-structure satisfies $\top$
    and does not satisfy $\bot$.
    Note that for $c$ a tuple of constant symbols
    $\MM \model{\Si} \phi(c)$ if and only if
    $\MM \model{\Si} \phi(\mmintp{c})$.
\end{rmk}

\begin{dfn}[Sentences and theories]
    \link{sentences}
    Let $\Si$ be a signature and $\phi$ a $\Si$-formula.
    We say $\phi \in \form{\Si}$ is a $\Si$-sentence when
    it has no free variables, $\tv{\phi} = \nothing$.

    $T$ is an $\Si$-theory when it is a subset of $\form{\Si}$
    such that all elements of $T$ are $\Si$-sentences.
    We denote the set of $\Si$-theories as $\theory{\Si}$.
\end{dfn}

\begin{dfn}[Models]
    Given an $\Si$-structure $\MM$ and $\Si$-theory $T$,
    we write $\MM \model{\Si} T$ and say
    \emph{$\MM$ is a $\Si$-model of $T$} when
    for all $\phi \in T$ we have $\MM \model{\Si} \phi$.
\end{dfn}

\begin{eg}
    In the signature of rings,
    \linkto{dfn_rings}{the rings axioms} will be the theory
    of rings and each model of the theory will consist of a single sort - the ring.
    The %?\linkto{missing_link}{
    theory of $\ZFC$ %} %? MISSING LINK
    consists of the $\ZFC$ axioms and a model of $\ZFC$
    would be a single sort thought of as the `class of all sets'.
    In the signature of modules,
    \linkto{dfn_modules}{the theory of modules} will consist of the theory for rings,
    the theory for commutative groups, and the axioms for modules over a ring.
    A model of the theory of modules would consist of two sorts,
    one for the ring and one for the module.
\end{eg}

\begin{dfn}[Consequence]
    Given a $\Si$-theory $T$
    and a $\Si$-sentence $\phi$,
    we say $\phi$ is a consequence of $T$
    and say $T \model{\Si} \phi$
    when for all $\Si$-models $\MM$ of $T$,
    we have $\MM \model{\Si} \phi$.
    We also write $T \model{\Si} \De$
    for $\Si$-theories $T$ and $\De$
    when for every $\phi \in \De$ we have $T \model{\Si} \phi$.
\end{dfn}

\begin{ex}[Logical consequence]
    Let $T$ be a $\Si$-theory and $\phi$ and $\psi$ be $\Si$-sentences.
    Show that the following are equivalent:
    \begin{itemize}
        \item $T \model{\Si} \phi \to \psi$
        \item $T \model{\Si} \phi$ implies $T \model{\Si} \psi$.
    \end{itemize}
\end{ex}

\begin{dfn}[Consistent theory]
    \link{consistent}
    A $\Si$-theory $T$ is consistent if either of the following equivalent
    definitions hold:
    \begin{itemize}
        \item
            There does not exists a
            $\Si$-sentence $\phi$ such that
            $T \model{\Si} \phi$ and $T \model{\Si} \NOT \phi$.
        \item There exists
            a $\Si$-model of $T$.
    \end{itemize}
    Thus the definition of consistent is intuitively
    `$T$ does not lead to a contradiction'.
    A theory $T$ is finitely consistent if all
    finite subsets of $T$ are consistent.
    This will turn out to be another equivalent definition,
    given by the \linkto{compactness}{compactness theorem}.
\end{dfn}
\begin{proof}
    We show that the two definitions are equivalent.
    \begin{forward}
        Suppose no model exists.
        Take $\phi$ to be the $\Si$-sentence $\top$.
        Hence all $\Si$-models of $T$ satisfy $\top$ and $\bot$
        (there are none) so
        $T \model{\Si} \top$ and $T \model{\Si} \bot$.
    \end{forward}
    \begin{backward}
        Suppose $T$ has a $\Si$-model $\MM$
        and $T \model{\Si} \phi$ and $T \model{\Si} \NOT \phi$.
        This implies $\MM \model{\Si} \phi$ and $\MM \nodel{\Si} \phi$,
        a contradiction.
    \end{backward}
\end{proof}

\begin{dfn}[Elementary equivalence]
    Let $\MM$, $\NN$ be $\Si$-structures.
    They are elementarily equivalent if for any $\Si$-sentence $\phi$,
    $\MM \model{\Si} \phi$ if and only if $\NN \model{\Si} \phi$.
    We write $\MM \equiv_\Si \NN$.
\end{dfn}

\begin{dfn}[Maximal and complete theories]
    \link{equiv_def_completeness_0}
    A $\Si$-theory $T$ is \textit{maximal} if
    for any $\Si$-sentence $\phi$,
    $\phi \in T$ or $\NOT \phi \in T$.

    $T$ is \textit{complete}
    when either of the following equivalent
    definitions hold:
    \begin{itemize}
        \item For any $\Si$-sentence $\phi$,
            $T \model{\Si} \phi$ or
            $T \model{\Si} \NOT \phi$.
        \item All models of $T$ are elementarily equivalent.
    \end{itemize}
    Note that maximal theories are complete.
\end{dfn}
\begin{proof}
    \begin{forward}
        Let $\MM$ and $\NN$ be models of $T$
        and $\phi$ be a $\Si$-sentence.
        If $T \model{\Si} \phi$ then both satisfy $\phi$.
        Otherwise $\NOT \phi \in T$ and neither satisfy $\phi$.
    \end{forward}

    \begin{backward}
        If $\phi$ is a $\Si$-sentence then suppose for a contradiction
        \[T \nodel{\Si} \phi \text{ and } T \nodel{\Si} \NOT \phi\]
        Then there exist models of $T$
        such that $\MM \nodel{\Si} \phi$ and $\NN \nodel{\Si} \NOT \phi$.
        By assumption they are elementarily equivalent and so
        $\MM \model{\Si} \NOT \phi$ implies $\NN \model{\Si} \NOT \phi$,
        a contradiction.
    \end{backward}
\end{proof}

\begin{ex}[Not consistent, not complete]
    \link{not_consequence}
    Let $T$ be a $\Si$-theory
    and $\phi$ is a $\Si$-sentence.
    Show that $T \nodel{\Si} \phi$
    if and only if $T \cup \set{ \NOT \phi}$ is consistent.
    Furthermore, $T \nodel{\Si} \NOT \phi$
    if and only if $T \cup \set{\phi}$ is consistent.

    Note that by definition for $\Si$-structures and
    $\Si$-formulas we (classically) have that
    \[
        \MM \modelsi \NOT \phi(a) \iff \MM \nodelsi \phi(a)
    \]
    Find examples of theories that do not satisfy
    \[
        T \modelsi \NOT \phi \iff T \nodelsi \phi
    \]
\end{ex}
