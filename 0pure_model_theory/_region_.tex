\message{ !name(1signature.tex)}
\message{ !name(1signature.tex) !offset(435) }
\subsection{Classical models, theories}
From now on we will be interpreting only in the category $* / \SET$
of non-empty sets. 
We require non-emptiness because the classical proof of compactness,
given by a \linkto{make_wit}{Henkin construction} 
relies on non-emptiness of our models.

\begin{dfn}[Satisfaction]
    Let $\MM$ be a $\Si$-structure (interpreted in $* / \SET$ 
    and $\phi$ a $\Si$-formula. 
    Let $a \in \prod_{x : A \in \tv{\phi}} \mmintp{A}$ be a tuple indexed by 
    the typed free variables of $\phi$.
    Then we define $\MM \modelsi \phi(a)$ by induction on formulas:
    \begin{itemize}
        \item If $\phi$ is $\top$ then $\MM \model{\Si} \phi$.\footnote{
            We can omit the $a$ when there are no free variables.
            Formally $a$ is the unique element in the empty product.}
        \item If $\phi$ is $t = s$ then
            $\MM \model{\Si} \phi(a)$ when 
            $\modintp{\MM}{t}(a) = \modintp{\MM}{s}(a)$.
            \item If $\phi$ is $r(t)$, 
            where $r \hookr A_1 \tdt A_n$ is a relation symbol and  
            $t : A_1 \tdt A_n$ are terms,
            then $\MM \model{\Si} \phi(a)$ when 
            $\modintp{\MM}{t}(a) \in \modintp{\MM}{r}$.\footnote{
            $\modintp{\MM}{t}$ was the 
            \linkto{interpretation_terms}{product of interpreted terms}.
            }
            \vspace{1em}
        \item If $\phi$ is 
            $\NOT\psi$ for some $\psi \in \form{\Si}$, 
            then $\MM \model{\Si} \phi(a)$ when $\MM \nodel{\Si} \psi(a)$
        \item If $\phi$ is  $\psi \lor \chi$, 
            then $\MM \model{\Si} \phi(a)$ when 
            $\MM \model{\Si} \psi(a)$ or $\MM \model{\Si} \chi(a)$.
        \item If $\phi$ is 
            $\forall v : A, \psi$,
            then $\MM \model{\Si} \phi(a)$ 
            if for any $b \in \mmintp{A}$,   
            $\MM \model{\Si} \psi(a,b)$.
    \end{itemize}
\end{dfn}

% \begin{rmk}
%     Any $\Si$-structure satisfies $\top$
%     and does not satisfy $\bot$.
%     Note that for $c$ a tuple of constant symbols
%     $\MM \model{\Si} \phi(c)$ if and only if
%     $\MM \model{\Si} \phi(\mmintp{c})$.
% \end{rmk}

% \begin{dfn}[Sentences and theories]
%     \link{sentences}
%     Let $\Si$ be a signature and $\phi$ a $\Si$-formula.
%     We say $\phi \in \form{\Si}$ is a $\Si$-sentence when
%     it has no free variables, $\tv{\phi} = \nothing$.

%     $T$ is an $\Si$-theory when it is a subset of $\form{\Si}$
%     such that all elements of $T$ are $\Si$-sentences.
%     We denote the set of $\Si$-theories as $\theory{\Si}$.
% \end{dfn}

% \begin{dfn}[Models]
%     Given an $\Si$-structure $\MM$ and $\Si$-theory $T$,
%     we write $\MM \model{\Si} T$ and say
%     \emph{$\MM$ is a $\Si$-model of $T$} when
%     for all $\phi \in T$ we have $\MM \model{\Si} \phi$.
% \end{dfn}

% \begin{eg}
%     In the signature of rings,
%     \linkto{dfn_rings}{the rings axioms} will be the theory
%     of rings and each model of the theory will consist of a single sort - the ring.
%     The \linkto{missing_link}{theory of $\ZFC$} %?$? MISSING LINK
%     consists of the $\ZFC$ axioms and a model of $\ZFC$
%     would be a single sort thought of as the `class of all sets'.
%     In the signature of modules,
%     \linkto{dfn_modules}{the theory of modules} will consist of the theory for rings,
%     the theory for commutative groups, and the axioms for modules over a ring.
%     A model of the theory of modules would consist of two sorts,
%     one for the ring and one for the module.
% \end{eg}

% \begin{dfn}[Consequence]
%     Given a $\Si$-theory $T$
%     and a $\Si$-sentence $\phi$,
%     we say $\phi$ is a consequence of $T$
%     and say $T \model{\Si} \phi$
%     when for all $\Si$-models $\MM$ of $T$,
%     we have $\MM \model{\Si} \phi$.
%     We also write $T \model{\Si} \De$
%     for $\Si$-theories $T$ and $\De$
%     when for every $\phi \in \De$ we have $T \model{\Si} \phi$.
% \end{dfn}

% \begin{ex}[Logical consequence]
%     Let $T$ be a $\Si$-theory and $\phi$ and $\psi$ be $\Si$-sentences.
%     Show that the following are equivalent:
%     \begin{itemize}
%         \item $T \model{\Si} \phi \to \psi$
%         \item $T \model{\Si} \phi$ implies $T \model{\Si} \psi$.
%     \end{itemize}
% \end{ex}

% \begin{dfn}[Consistent theory]
%     \link{consistent}
%     A $\Si$-theory $T$ is consistent if either of the following equivalent
%     definitions hold:
%     \begin{itemize}
%         \item
%             There does not exists a
%             $\Si$-sentence $\phi$ such that
%             $T \model{\Si} \phi$ and $T \model{\Si} \NOT \phi$.
%         \item There exists
%             a $\Si$-model of $T$.
%     \end{itemize}
%     Thus the definition of consistent is intuitively
%     `$T$ does not lead to a contradiction'.
%     A theory $T$ is finitely consistent if all
%     finite subsets of $T$ are consistent.
%     This will turn out to be another equivalent definition,
%     given by the \linkto{compactness}{compactness theorem}.
% \end{dfn}
% \begin{proof}
%     We show that the two definitions are equivalent.
%     \begin{forward}
%         Suppose no model exists.
%         Take $\phi$ to be the $\Si$-sentence $\top$.
%         Hence all $\Si$-models of $T$ satisfy $\top$ and $\bot$
%         (there are none) so
%         $T \model{\Si} \top$ and $T \model{\Si} \bot$.
%     \end{forward}
%     \begin{backward}
%         Suppose $T$ has a $\Si$-model $\MM$
%         and $T \model{\Si} \phi$ and $T \model{\Si} \NOT \phi$.
%         This implies $\MM \model{\Si} \phi$ and $\MM \nodel{\Si} \phi$,
%         a contradiction.
%     \end{backward}
% \end{proof}

% \begin{dfn}[Elementary equivalence]
%     Let $\MM$, $\NN$ be $\Si$-structures.
%     They are elementarily equivalent if for any $\Si$-sentence $\phi$,
%     $\MM \model{\Si} \phi$ if and only if $\NN \model{\Si} \phi$.
%     We write $\MM \equiv_\Si \NN$.
% \end{dfn}

% \begin{dfn}[Maximal and complete theories]
%     \link{equiv_def_completeness_0}
%     A $\Si$-theory $T$ is \textit{maximal} if
%     for any $\Si$-sentence $\phi$,
%     $\phi \in T$ or $\NOT \phi \in T$.

%     $T$ is \textit{complete}
%     when either of the following equivalent
%     definitions hold:
%     \begin{itemize}
%         \item For any $\Si$-sentence $\phi$,
%             $T \model{\Si} \phi$ or
%             $T \model{\Si} \NOT \phi$.
%         \item All models of $T$ are elementarily equivalent.
%     \end{itemize}
%     Note that maximal theories are complete.
% \end{dfn}
% \begin{proof}
%     \begin{forward}
%         Let $\MM$ and $\NN$ be models of $T$
%         and $\phi$ be a $\Si$-sentence.
%         If $T \model{\Si} \phi$ then both satisfy $\phi$.
%         Otherwise $\NOT \phi \in T$ and neither satisfy $\phi$.
%     \end{forward}

%     \begin{backward}
%         If $\phi$ is a $\Si$-sentence then suppose for a contradiction
%         \[T \nodel{\Si} \phi \text{ and } T \nodel{\Si} \NOT \phi\]
%         Then there exist models of $T$
%         such that $\MM \nodel{\Si} \phi$ and $\NN \nodel{\Si} \NOT \phi$.
%         By assumption they are elementarily equivalent and so
%         $\MM \model{\Si} \NOT \phi$ implies $\NN \model{\Si} \NOT \phi$,
%         a contradiction.
%     \end{backward}
% \end{proof}

\begin{ex}[Not consistent, not complete]
    \link{not_consequence}
    Let $T$ be a $\Si$-theory
    and $\phi$ is a $\Si$-sentence.
    Show that $T \nodel{\Si} \phi$
    if and only if $T \cup \set{ \NOT \phi}$ is consistent.
    Furthermore, $T \nodel{\Si} \NOT \phi$
    if and only if $T \cup \set{\phi}$ is consistent.

    Note that by definition for $\Si$-structures and
    $\Si$-formulas we (classically) have that
    \[
        \MM \modelsi \NOT \phi(a) \iff \MM \nodelsi \phi(a)
    \]
    Find examples of theories that do not satisfy
    \[
        T \modelsi \NOT \phi \iff T \nodelsi \phi
    \]
\end{ex}

\message{ !name(1signature.tex) !offset(-206) }
