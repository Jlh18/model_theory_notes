\section{Examples}

Here is a compilation of some examples of signatures and theories.

\begin{dfn}[Signature and theory of groups]
    \link{dfn_grp}
    Let the following be the signature of groups:
    \begin{itemize}
        \item There is only one sort symbol $G$ which stands for the group.
        \item Function symbols: the constant symbol for the identity 
        $0 : G$, the symbol for addition $+ : G^2 \to G$ 
        and the symbol for inverses $- : G \to G$.
        \item There are no relation symbols.
    \end{itemize}
    
    In the signature of groups we define the theory of (commutative) groups
    (we write $a + b$ to mean $+(a,b)$ and so on):
    \begin{itemize}
        \item[$\vert$] Assosiativity: 
            $\forall x, y, z : G, (x + y) + z = x + (y + z)$
        \item[$\vert$] Identity:
            $\forall x : G, x + 0 = x$ 
        \item[$\vert$] Inverse: $\forall x : G, x - x = 0$ 
        \item[$\vert$] (For commutative only): 
            $\forall x, y : G, x + y = y + x$
    \end{itemize}
    Note that we don't have axioms for `closure of functions'
    and `existence or uniqueness of inverses' as
    it is encoded by interpretation of 
    $+, -, \times$ being well-defined.
    
    Removing the function symbol $-$ and the sentence `inverse for addition' 
    we obtain the language and theory of (commutative) monoids.
\end{dfn}

\begin{dfn}[Signature of monoid actions]
    \link{dfn_grp_ac}
    Let the following be the signature of monoid actions:
    \begin{itemize}
        \item The sort symbols are: $A$ the sort for the 
            monoid and $X$ the sort for the $A$-set.
        \item The functions symbols are $1 : A$ and 
            $\times: A^2 \to A$ purely for the monoid
            together with $\rho : A \times X \to X$ 
            for the action of the monoid on the set.
        \item There are no relation symbols.
    \end{itemize}
    
    In this signature we define the theory of monoid actions
    (we write $a x$ to mean $\rho(a,x)$ and so on):
    \begin{itemize}
        \item The \linkto{dfn_grp}{theory of monoids} 
        with $1$ as the identity and $\times$ as symbol for addition.
        \item Identity: 
        $\forall x : X, 1 x = x$
        \item Compatibility: 
        $\forall a, b : A, \forall x : X, (a b) x = a (b x)$
    \end{itemize}
\end{dfn}

\begin{dfn}[Signature and theory of modules]
    \link{dfn_modules}
    Let the following be the signature of modules over a ring:
    \begin{itemize}
        \item The sort symbols are: $A$ the sort for the ring and 
            $M$ the sort for the module.
        \item The function symbols for the ring are constants 
            $0_A, 1_A : A$, addition and multiplication
            $+_A, \times_A : A^2 \to A$ and inverse $-_A : A \to A$.
            For the module we have constant $0_M : M$, 
            addition $+_M : M^2 \to M$ and inverse $-_M : M \to M$.
            Lastly we have $\rho : A \times M \to M$ 
            for the action of the ring on the module.
        \item There are no relation symbols.
    \end{itemize}
    
    In the signature of modules we define the theory of 
    modules over a ring as the set containing
    (we write $a m$ to mean $\rho(a,m)$ and so on):
    \begin{itemize}
        \item The \linkto{dfn_rings}{theory of rings} 
        with the obvious substitutions.
        \item The \linkto{dfn_grp}{theory of commutative groups} 
        where $G$ is replaced with $M$,
        and the function symbols for the group are the function 
        symbols for the module.
        \item The \linkto{dfn_grp_ac}{theory of monoid actions} 
        with the monoid as $A$
        under multiplication and the $A$-set as $M$.
        \item Distributivity of $\rho$ over $+_A$:
        $\forall a , b : A, \forall x: M, (a + b) x = a x + b x$
        \item Distributivity of $\rho$ over $+_M$: 
        $\forall a : A, \forall x, y : M, a (x + y) = a x + a y$
    \end{itemize}
\end{dfn}

\begin{dfn}[Singature of a single binary relation]
    \link{sig_just_one_bin_rel}
    The signature $\Si_{<}$ of a single binary relation is given by 
    \begin{itemize}
        \item A sort symbol $P$ for the set.
        \item No function symbols.
        \item A relation symbol $< \hookr P^2$
    \end{itemize}
    For variables $x$ and $y$ of type $P$, 
    we write $x < y$ as notation for $<(x,y)$.
\end{dfn}

\begin{dfn}[Order theories]
    \link{order_theories}
    The $\Si_<$-theory of partial orders is given by the two formulas 
    \begin{itemize}
        \item Non-reflexivity - $\forall x : P, \NOT (x < x)$
        \item Transitivity - $\forall x,y,z : P, x < y \AND y < z \to x < z$
    \end{itemize}
    The $\Si_<$-theory of linear orders is the theory of partial orders plus 
    \begin{itemize}
        \item Linearity - $\forall x,y : P, x < y \OR x = y \OR y < x$
    \end{itemize}
    The $\Si_<$-theory of dense linear orders 
    is the theory of linear orders plus
    \begin{itemize}
        \item Density -
            $\forall x, y : P, x < y \to \exists z : P, x < z \AND z < y$.
    \end{itemize}
\end{dfn}