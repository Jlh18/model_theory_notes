\section{Examples}





%Model Theory of Field (Marker)
%covers:
%-- categoricity, completeness, decidability of ACF
%-- weird corollary of above
%-- Ax - Groth
%-- ACF has quantifier elimination
%-- Nullstellensatz
%-- 


%Models of Set Theory
%covers:
%-- consistency of other ZF axioms with foundation
%-- independence of continuum hypothesis with 
\begin{dfn}[Signature for $\ZFC$ and assosiated notations]
    Let $\Si_{\ZFC} = \brkt{\nothing, \nothing, n_\star, \set{\in}, m_\star}$.
    Where $m_\in = 2$ and $n_\star$ is the empty function.
    Let $x,y,z, P, blahblah$ be variable symobls and %
    $\phi$ a $\Si_\ZFC$-formula.%
    Define the following shorthand for formulas:
    \begin{align*}%%%
        x \text{ empty} &:= \\
        \forall x \in y, \phi &:= \forall x, (x \in y \to \phi)\\
        x \notin y &:=\\
        y \subs x &:= \\
        P \text{ is a power set of } x &:= \\
        f : x \to y &:= 
        %%?
    \end{align*}
\end{dfn}

\begin{eg}[$\ZFC$]
    Define the $\Si_{\ZFC}$-theory $\ZFC$ inductively,
    it contains:
    \begin{itemize}
        \item[$\vert$] Extensionality
        \item[$\vert$] Foundation
        \item[$\vert$] Separation
        \item[$\vert$] Pairing
        \item[$\vert$] Union
        \item[$\vert$] Replacement
        \item[$\vert$] Infinity
        \item[$\vert$] Power set
        \item[$\vert$] Choice 
    \end{itemize}
    Note that $\in$ is only symoblic and is not to be confused with 
    actual membership.
\end{eg}

%Good example in back and forth chapter of Marker
\begin{ex}[Dense linear orders without max or min is complete]

\end{ex}

\begin{dfn}[Signature of groups, theory of groups]
    Let $\Si_\GRP = (\set{e},\set{+,-}, n_\star, \nothing, m_\star)$
    be the signature of groups, where
    $n_+ = 2$ and $n_- = 1$ and $m_\star$ is the empty function.
    We write $a + b := + (a,b)$ and $- x := -(x)$
    Then the theory of groups is
    \begin{align*}
        \GRP &= \brkt{
        &= \forall x \forall y \forall z, (x + y) + z = x + (y + z)\\
        &= \forall x, e + x = x + e = x\\
        &= \forall x, (- x) + x = x + (- x) = e}
    \end{align*}
    One can see that $\GRP$ is a universal theory.
\end{dfn}

\begin{ex}[Quantifier elimination for theory of infinite equivalence relations]
    Let $\Si_{\ER} = (\nothing,\nothing,n_f,\set{E},m_r)$,
    where $m_E = 2$.
    We write for variables $x$ and $y$, 
    we write $x \sim y$ as notation for $E(x,y)$
    The theory of equivalence relations $\ER$ 
    is set set containing the following formulas:
    \begin{align*}
        &\text{Reflexivity } \forall x, x \sim x\\
        &\text{Symmetry } \forall x \forall y, x \sim y \to y \sim x\\
        &\text{Transitivity } 
        \forall x \forall y \forall z, (x \sim y \AND y \sim z) \to x \sim z
    \end{align*}
    For $n \in \N_{>1}$ define 
    \begin{align*}
        &\phi_n := \bigexists{i = 1}{n} x_i, \bigand{i < j}{} x_i \nsim x_j\\
        &\psi_n := \forall x, \bigexists{i = 1}{n} x_i, 
            \bigand{i = 1}{n} \brkt{x \sim x_i} \AND 
            \bigand{i < j}{} \brkt{x_i \ne x_j}
    \end{align*}
    Show that the theory $T = \ER \cup {\phi_n, \psi_n}_{1 < i}$ has 
    quantifier elimination.
    (You may wonder if it is indeed a theory
    and what nasty induction must be done to 
    show that its formulas can be constructed.)
\end{ex}
\begin{proof}
    \emph{First attempt:}
    \linkto{improved_condition_for_q_e}{It would suffice to show} 
    that if $\io : \AA \to \MM$ is a $\Si_\ER$-embedding
    and $\MM \model{\Si_\ER} T$ then for any formula $\phi$ with a 
    free variable $w$ and any $a \in \carrier{\AA}^S$,
    \[
        \MM \model{\Si_\ER} \exists w, \phi(\io(a)) \implies 
        \AA \model{\Si_\ER} \exists w, \phi(a)
    \]
    However this is not true. 
    Take $\MM = \Q$ with the interpretation of $E$ as 
    \[\set{(x,y) \in \Q \times \Q \st x - y \in \Z}\]
    i.e. the equivalence relation induced by the quotient $\Q / \Z$.
    Each equivalence class is a coset $x + \Z$ and therefore is infinite,
    and there are infinitely many such classes 
    (take $\frac{1}{2^n} + \Z$ for each $n$ for example).
    Take $\AA = \set{0}$ and take $\subs$ as the injection into $\Q$.
    $\AA$ inherits the same equivalence relation, 
    which is a single equivalence class with a single element.
    \[
        \Q \model{\Si_\ER} \exists w, \NOT (w \sim 0)
        \quad \text{but} \quad 
        \AA \model{\Si_\ER} \forall w, w \sim 0
    \]

    \emph{Second attempt:}
    We first define an equivalence class:
    if $\MM \model{\Si_\ER} T$ and $a \in \carrier{\MM}$ then 
    \[[a]_\MM := \set{b \in \carrier{\MM} \st \MM \model{\Si_\ER} a \sim b}\]
    Note that this is a definable set.
    If $A \subs \carrier{\MM}$ we write $[A]$ to be the image
    \[A = \{b \in \carrier{\MM} \st \exists a \in A, 
    \MM \model{\Si_\ER} a \sim b\}\]
    Define the quotient:
    \[\MM / \sim := \set{[a]_\MM \st a \in \carrier{\MM}}\]
    
    By \linkto{vaught_test}{Vaught's test}, to show that $T$ is complete
    it suffices to show that $T$ is consistent, 
    not finitely modelled and $\aleph_0$-categorical.
    A non-empty model of $T$ would be $\Q$ 
    with the equivalence relation defined in 
    the first attempt.
    If $\MM$ is a model of $T$ and $\MM$ is finite,
    then let $m \in \N$ be the cardinality of $\MM$.
    Then the formula $\phi_{m+1} \in T$ is not satisfied by $\MM$,
    a contradiction.
    We use a `back and forth' argument to show first that 
    $T$ is $\aleph_0$-categorical. 

    Let $\MM, \NN$ be countable $\Si_\ER$-models of $T$.
    We can then enumerate each using the naturals:
    \[\carrier{\MM} = \set{a_i}_{i \in \N}, 
    \quad \carrier{\NN = \set{b_i}_{i \in \N}}\]
    We induct on $i \in \N$ in the following statement:
    For each $i \in \N$ we define 
    $A_i \subs \carrier{\MM}, B_i \subs \carrier{\NN}$
    and $\Si_\ER$-isomorphisms $g_i : A_i \to B_i$
    where the subsets are naturally $\Si_\ER$-structures
    (as the language only has relation symbols).
    These will be constructed such that their unions are 
    $\carrier{\MM}$ and $\carrier{\NN}$ respectively.
    
    If $i = 0$ then define both sets to be 
    $\nothing$ and $g_0$ be the empty function.
    For the induction step we case on if $i$ is odd or even.
    If $i + 1 = 2k$ for some $k$ then case on if $a_k \in A_i$.
    If it is then define $A_{i+1} := A_i$, 
    $B_{i+1} := B_i$, 
    and $g_{i+1} := g_i$.
    If $a_k \notin A_i$ then case on if $[a_k]_\MM \cap A_i$ is empty or not.

    Suppose $[a_k]_\MM \cap A_i$ is empty.
    We can show that $\NN / \sim$ is infinite whilst 
    $[B_i]_{\NN}$ is finite, 
    hence $\NN / \sim \setminus [B_i]_{\NN}$ is non-empty.
    Thus there exists 
    $c \in \carrier{\NN}$ such that $[c]_\NN \notin [B_i]$.
    Then take $A_{i+1} := A_i \cup \set{a_k}$, 
    $B_{i+1} := B_i \cup \set{c}$, 
    and $g_{i+1}(a_k) = c$, 
    agreeing with $g_i$ otherwise.
    To show that $g_{i+1}$ is an embedding we only need to check that 
    the relation is preserved and that it is injective.
    Let $a,b \in A_{i+1}$, by induction if 
    both are in $A_i$ then we have
    \[a \modintp{\MM}{\sim} b \implies 
    g_{i+1} (a) = g_i (a) \modintp{\NN}{\sim} g_i (b) = g_{i+1} (b)\]
    and 
    \[g_{i+1} (a) \modintp{\NN}{\sim} g_{i+1} (b)
    \implies g_i (a) \modintp{\NN}{\sim} g_i (b)
    \implies a \modintp{\MM}{\sim} b\]
    If $a = a_k \in A_{i+1}$ and $b \in A_i$ then $a_k \modintp{\MM}{\nsim} b$
    as $[a_k]_\MM$ is empty.
    By definition of $g_{i+1}(a_k) = c$ we have that 
    $g(a_k) \modintp{\NN}{\nsim} g(b)$.
    If $a = b = a_k$ then both sides are satisfied and we are done.
    Injectivity is clear and as this these are finite sets
    of the same cardinality surjectivity also holds.

    Otherwise, suppose there exists $d \in A_i$ such that $d \sim a_k$.
    $A_i$ is finite and so
    $[d]_\MM \cap A_i$ is finite and $g_i([d]_\MM \cap A_i)$ is finite.
    Equivalence classes are infinite thus there exists 
    \[c \in [g_i(d)]_\NN \setminus g_i([d]_\MM \cap A_i)\]
    Take $A_{i+1} := A_i \cup \set{a_k}$, 
    $B_{i+1} := B_i \cup \set{c}$, 
    and $g_{i+1}(a_k) = c$, 
    agreeing with $g_i$ otherwise.
    Again we show that the relation is preserved;
    let $a,b \in A_{i+1}$. 
    If $a,b \ne a_k$ by induction we 
    are done.
    If $a = a_k, b \ne a_k$ then $a \modintp{\MM}{\sim} b$ implies 
    $d \modintp{\MM}{\sim} b$ by transitivity and the definition of $d$.
    Furthermore by induction:
    $g_{i+1}(a) \modintp{\NN}{\sim} g_{i}(d) 
    \modintp{\NN}{\sim} g_{i}(b) = g_{i+1}(b)$
    If $a = a_k, b \ne a_k$ then $a \modintp{\MM}{\nsim} b$ implies 
    $d \modintp{\MM}{\nsim} b$ and 
    $g_{i+1}(a) \modintp{\NN}{\sim} g_{i}(d) \modintp{\NN}{\nsim} g_{i}(b)$
    hence 
    $g_{i+1}(a) \modintp{\NN}{\sim} g_{i}(d) \modintp{\NN}{\nsim} g_{i+1}(b)$
    Lastly if both are $a_k$ then both sides are satisfied and we are done.
    Injectivity and surjectivity are clear.

    If $i + 1 = 2k + 1$ for some $k$ then then case on if $b_k \in B_i$
    and go through the same construction backwards.
    We then have that $\bigcup_{i \in \N} A_i = \carrier{\MM}$ as 
    each $a_i \in A_{2i+2} \subs \carrier{\MM}$, 
    and similarly $\bigcup_{i \in \N} B_i = \carrier{\NN}$.
    We check that $g := a_i \mapsto g_{2i+2}(a_i)$ is a well-defined isomorphism
    $\MM \iso \NN$.
    Let $a, b \in \carrier{\MM}$ where $a \in A_i, b \in A_j$.
    If $g(a) = g(b)$ then let $m = \max(i,j)$, 
    \[g_{m}(a) = g(a) = g(b) = g_{m}(b)\]
    thus by injectivity of $g_{m}$, 
    $a = b$. 
    Thus $g$ is injective.
    Suppose $b \in B_i \subs \carrier{\MM}$. 
    By surjectivity of $g_{i}$, there exists $a \in A_i$ such that 
    $g(a) = g_i(a) = b$.

    Thus $\MM$ is $\Si_\ER$-isomorphic to $\NN$ and $T$ is complete.
    
\end{proof} 

%%%%%%%%%%%%%%%%%%%%%%%%%5

