


%%%% No order

\begin{dfn}
    Let $T$ be a $\Si$-theory. 
    Let $\MM \modelsi T$ and let $p$ and $q$ be $n$-types.
    Then the following equivalent definitions are given for $p$ implies $q$,
    denoted by $p \modelsi q$.
    \begin{itemize}
        \item For each elementary extension $\MM \to \NN \modelsi T$ 
        we have $p(\NN) \subs q(\NN)$.
        \item For each maximal $n$-type $r$ containing $p$, $q \subs r$.
        \item For each $\psi \in q$, 
        there exists a finite subset $\De \subs p$ such that 
        $\bigand{}{} \De$ isolates $\set{\psi}$ with respect to $T$.
        %Each finite subset of q is isolated by some finite subset of p
    \end{itemize}
\end{dfn}
\begin{proof}
    $(1 \implies 2)$ Suppose $p \subs r$ is maximal. 
    By definition of $n$-types $r$ is realised in some $\NN \model T$.
    \linkto{amalgamation}{There exists some pushout $P$} 
    with elementary embeddings $\MM \to P$ and $\NN \to P$.
    Since $\NN \to P$ is elementary, 
    $r$ is realised in $P$ by some tuple $b \in P^n$.
    Hence $b \in r(P) \subs p(P) \subs q(P)$ and for any $\phi \in q$
    either $\phi \in r$ or $\not \phi \in r$. 
    In the latter case we have $P \modelsi \not \phi(b)$,
    but also $P \modelsi \phi(b)$ since $b \in q(P)$, a contradiction.
    
    $(2 \implies 3)$ Let $\phi \in q$ and suppose for a contradiction
    No such finite set exists. 
    Hence for any finite $\De \subs p$ there exists $\NN \modelsi T$
    such that
    \[
        \NN \modelsi \exists v, \bigand{}{} \De \AND \NOT \phi
    \]
    Hence $p \cup \set{\NOT \phi}$ is finitely consistent 
    \linkto{compactness_for_types}{so consistent}.
    Hence any maximal $r$ containing $p$ will contain $\NOT \phi$,
    so $q \cup \set{\NOT \phi}$ is consistent, a contradiction.
    
    $(3 \implies 1)$ Suppose we have $\MM \to \NN \modelsi T$ and
    $a \in p(\NN)$.
    We show that for any $\phi \in q$ we have $\NN \modelsi \phi(a)$.
    By assumption there exists a finite $\De \subs p$ such that 
    \[
        N \modelsi \bigand{}{} \De(a) \to \phi(a)
    \]
    Lastly note that $a \in p(\NN) \subs \De(\NN)$ and so $N \modelsi \De(a)$.
\end{proof}