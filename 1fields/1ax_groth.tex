\section{Ax-Grothendieck}
This section studies the theories of fields in the language of rings,
with particular focus on algebraically closed fields.

\subsection{Language of Rings}
We introduce rings and fields and construct the field of fractions
of integral domains.

\begin{dfn}[Signature and theory of rings]
    \link{dfn_rings}
    Let the following be the signature of rings:
    \begin{itemize}
        \item There is only one sort symbol $A$ which stands for the ring.
        \item The function symbols are the constant symbols $0, 1 : A$, 
        the symbols for addition and multiplication $+ , \times : A^2 \to A$ 
        and taking for inverse $- : A \to A$.
        \item There are no relation symbols.
    \end{itemize}

    We define the theory of rings $\RNG$ as the set containing
    (we write $a \times b$ or $a b$ to mean $\times(a,b)$ and so on):
    \begin{itemize}
        \item[$\vert$] The \linkto{dfn_grp}{theory of commutative groups}
        with function symbols $0, +, -$.
        \item[$\vert$] The \linkto{dfn_grp}{theory of commutative monoids}
        with function symbols $1, \times$ 
        (assosiativity, identity and commutativity for $\times$).
        \item[$\vert$] Distributivity:
        $\forall x, y, z : A, x \times (y + z) = x \times y + x \times z$
    \end{itemize}
    Note that the theory of groups and rings are 
    \linkto{dfn_universal_theory}{universal}.
    We often omit the sort $A$ since there is only one sort.
    We adopt the usual convention for short hand, e.g. $2 := 1 + 1$ is 
    a term of type $A$ and $xy := x \times y$ is a term of type $A$ when 
    $x$ and $y$ are variables.
\end{dfn}

\begin{dfn}[Theory of integral domains and fields]
    We define the $\Si_\RNG$-theory of integral domains
    \[\ID := \RNG \cup \set{0 \ne 1,
    \forall x \forall y, xy = 0 \to (x = 0 \OR y = 0)}\]
    and the $\Si_\RNG$-theory of fields
    \[\FLD := \RNG \cup \set{\forall x, x = 0 \OR \exists y, xy = 1, 0 \ne 1}\]
    Note that the theory of integral domains is universal but the 
    theory of fields is not.
\end{dfn}

Going through the details of the construction of a field of fractions 
written from the model theoretic perspective
is a good exercise.
\begin{dfn}[Field of fractions]
    \link{uni_prop_field_of_fractions}\link{field_of_fractions}
    Suppose $A \model{\Si_\RNG} \ID$ and with a $\Si$-embedding 
    into a field $A \to K \model{\Si_\RNG} \ID$.
    If for every $L \model{\Si_\RNG} \FLD$ and 
    $\Si_\RNG$-embedding $\io_L : A \to L$, 
    there exists a unique $\Si_\RNG$-embedding $K \to L$
    that commutes with the diagram
    \begin{cd}
        A \ar[r] \ar[dr] & K \ar[d, dashed]\\
        & L
    \end{cd}
    then $K$ has the \textit{universal property} 
    of the field of fractions of $A$.
    Note that this definition is entirely without mention of interpretation 
    in `$\SET^*$', not mentioning a set theoretic construction of the field 
    $K$.

    We verify the standard construction of the field of fractions 
    has the universal property.
\end{dfn}
\begin{proof}
    We construct $X = \set{(x,y) \in A^2 \st y \ne 0}$ and
    an equivalence relation $(x,y) \sim (v, w) \iff xw = yv$. 
    (Use $A \model{\Si_\RNG} \ID$ 
    to show that this is an equivalence relation.)
    Let ${K} = X / \sim$ with $\pi : X \to K$ as the quotient map. 
    Denote $\pi(x,y) := \frac{x}{y}$, 
    interpret $\modintp{K}{0} = \frac{\modintp{A}{0}}{\modintp{A}{1}}$ 
    and $\modintp{K}{1} = \frac{\modintp{A}{1}}{\modintp{A}{1}}$.
    Interpret $+$ and $\cdot$ as standard fraction addition and multiplication
    and use $A \model{\Si_\RNG} \ID$ to check that these are well defined.

    Check that $K$ is an $\Si_\RNG$ structure and that 
    $K \model{\Si_\RNG} \FLD$.
    Define $\io : {A} \to {K} := a \mapsto \frac{a}{1}$
    and show that this well defined and injective.
    Check that $\io$ is a $\Si_\RNG$-morphism
    and note that since there are no relation symobls in $\Si_\RNG$
    it is also an embedding.

    Let $L$ be another field with an embedding from $A$.
    Define the map $\io : K \to L$ sending 
    $\frac{a}{b} \mapsto \frac{\io_L (a)}{\io_L(b)}$.
    Check that this is well-defined and a $\Si_\RNG$-morphism.
    It is injective because $\io_L$ is injective:
    \[\frac{\io_L(a)}{\io_L(b)} = 0 \implies \io_L(a) = 0
    \implies a = 0\]
    Thus it is an embedding.
    It is unique: suppose $\phi : K \to L$ is a $\Si_\RNG$-embedding
    that commutes with the diagram.
    Then for any $a \in K$, 
    $\phi(\frac{a}{1}) = \io_L(a) = \io (\frac{a}{1})$.
    Since both $\phi, \io$ are embeddings
    they commute with taking the inverse for $a \ne 0$:
    $\phi(\frac{1}{a}) = \io (\frac{1}{a})$.
    Since any element of $K$ can be written as $\frac{a}{b}$,
    we have shown that $\phi = \io$.
\end{proof}

\subsection{Algebraically closed fields}
\begin{dfn}[Theory of algebraically closed fields]
  We define the $\Si_{\RNG}$ theory of algebraically closed fields.
  Fixing variables $\set{a_i : A}_{i \in \N}$ and $x : A$ we define
    \[
        \ACF := 
        \FLD \cup \set{\forall a_0, \dots, \forall a_{n-1}, \exists x : A, \;
        x^n + \sum_{i = 0}^{n-1} a_i x^i = 0
        \st n \in \N_{>0}}
        \]
    Unlike the theories $\RNG,\ID,\FLD$ 
    this theory is countably infinite.
\end{dfn}

\begin{prop}
    $\ACF$ is not complete.
\end{prop}
\begin{proof}
    Take the $\Si_\RNG$-formula $2 = 0$.
    This is satisfied by the 
    \linkto{algebraic_closure_of_fields}{algebraic closure} 
    of $\F_2$ but not by that of $\F_3$,
    since field embeddings preserve characteristic. 
\end{proof}

\begin{dfn}[Algebraically closed fields of characteristic $p$]
    For $p \in \Z_{>0}$ prime let $\ACF_p := \ACF \cup \set{p = 0}$
    and let
    \[\ACF_0 := \ACF \cup \set{p \ne 0 \st p \in \Z_{>0} \text{ prime}}\]
\end{dfn}

An important fact about algebraically closed fields of characteristic $p$:
\begin{prop}[Transcendence degree and characteristic
    determine algebraically closed fields of characteristic $p$ 
    up to isomorphism]
    \link{trans_deg_and_char_determine_ACF_p}
    If $K_0,K_1$ are fields with same characteristic and 
    transcendence degree over their minimal subfield ($\F_p$ or $\Q$)
    then they are (non-canonically) isomorphic.
\end{prop}
\begin{proof}
    \linkto{appendix_trans_deg_and_char_determine_ACF_p}{See appendix}.
\end{proof}

\begin{nttn}
    If $K \model{\Si_\RNG} \ACF_p$, write $\tdeg(K)$ to mean the 
    transcendence degree over its minimal subfield ($\F_p$ or $\Q$).
\end{nttn}

\begin{prop}
    \link{ACF_ka_categorical_and_complete}
    $\ACF_p$ is $\ka$-categorical for all uncountable $\ka$, consistent
    and complete.
    %it is also decidable.?
\end{prop}
\begin{proof}
    Suppose $K,L \model{\Si_\RNG} \ACF_p$ and $\abs{K} = \abs{L} = \ka$.
    Then \linkto{card_of_alg_closed_fields}{
        $\tdeg(K) + \aleph_0 = \abs{K} = \ka$}
    and so $\tdeg(K) = \ka$ (as $\ka$ is uncountable).
    Similarly $\tdeg(L) = \ka$ and so $\tdeg(K) = \tdeg(L)$.
    Thus \linkto{trans_deg_and_char_determine_ACF_p}{
        $K$ and $L$ are isomorphic}.
    
    $\ACF_p$ is consistent due to the
    \linkto{algebraic_closure_of_fields}{existence of the algebraic closures}
    for any characteristic,
    \linkto{card_of_alg_closed_fields}{it is not finitely modelled}
    and is $\aleph_1$-categorical with 
    $\const{\Si_\RNG} + \aleph_0 \leq \aleph_1$,
    hence it is complete by \linkto{vaught_test}{Vaught's test}.
    %It is decidable because
    %it is recursively axiomatized and complete.?
\end{proof}

\subsection{Ax-Grothendieck}
In this section we provide a proof of \linkto{ax_groth_thm}{Ax-Grothendieck},
which says injective polynomial maps are surjective over certain algebraically closed fields.

\begin{dfn}[Polynomial map]
  \link{dfn_poly_map}
  Let $K$ be a field and $n$ a natural.
  Let $f : K^n \to K^n$ such that for each $a \in K^n$,
  \[f(a) = (f_1(a), \dots, f_n(a))\] for
  $f_1, \dots, f_n \in K[x_1, \dots, x_n]$.
  Then we call $f$ a polynomial map over $K$.
\end{dfn}

\begin{prop}[Lefschetz principle]
    \link{lefschetz}
    Let $\phi$ be a $\Si_\RNG$-sentence.
    Then the following are equivalent:
    \begin{enumerate}
        \item There exists a $\Si_\RNG$-model of $\ACF_0$ 
        that satisfies $\phi$.
        (If you like $\C \model{\Si_\RNG} \phi$.)
        \item $\ACF_0 \model{\Si_\RNG} \phi$
        \item There exists $n \in \N$ such that for any prime $p$
        greater than $n$,
        $\ACF_p \model{\Si_\RNG} \phi$
        \item There exists $n \in \N$ such that for any prime $p$
        greater than $n$ there exists a $\Si_\RNG$-model
        of $\ACF_p$ that satisfies $\phi$.
    \end{enumerate}
\end{prop}
\begin{proof}
  $1. \iff 2.$ \linkto{ACF_ka_categorical_and_complete}{
  $\ACF_0$ is complete} and consistent (with $\C$).

  $2. \iff 3.$ Suppose $\ACF[0] \model{\Si_\RNG} \phi$
  then since \linkto{proofs_are_finite}{`proofs are finite'}
  there exists a finite subset $\De$ of $\ACF[0]$ such that
  $\De \model{\Si_\RNG} \phi$.
  Let $n$ be maximum of all primes $p \in \N$ such that
  $p \ne 0 \in \De$ so we have
  \[\De \subs \ACF \cup \set{p \ne 0 \st p \leq n \text{ prime }}\]
  By uniqueness of characteristic,
  if $q$ is prime and greater than $n$ then for each prime $p \leq n$
  we have $\ACF_{0} \model{\Si_{\RNG}} p \ne 0$
  so $\ACF_{0} \model{\Si_{\RNG}} \De$.
  It follows that for all primes $p$ greater than $n$,
  $\ACF_p \model{\Si_\RNG} \phi$.

  Conversely, suppose for any prime $p$ greater than $n$
  $\ACF_p \model{\Si_\RNG} \phi$.
  By \linkto{ACF_ka_categorical_and_complete}{completeness} of
  $\ACF_{0}$ it suffices to assume $\ACF_{0} \model{\Si_{\RNG}} \NOT \phi$.
  Using the forward direction we have that there exists
  $m$ such that for all $p$ greater than $m$
  $\ACF_p \model{\Si_\RNG} \NOT \phi$.
  Since there are infinitely many primes we find some prime $p$
  such that $\ACF_{p}$ is inconsistent - a contradiction.

  $3. \iff 4.$ $\ACF_p$ is \linkto{ACF_ka_categorical_and_complete}{
    consistent and complete}.
\end{proof}

\begin{dfn}[Locally finite fields \cite{stack0}]
    \link{locally_finite_over_prime}
    Let $K$ be a field of characteristic $p$ a prime.
    Then the following are equivalent:
    \begin{enumerate}
        \item The minimal subfield generated by any finite subset of $K$ is finite.
        \item $\F_p \to K$ is algebraic.
        \item $K$ embeds into an algebraic closure of $\F_p$.
    \end{enumerate}
\end{dfn}
\begin{proof}
  $1.\implies 2.$
  Let $a \in K$. Then $\F_{p}(a)$ is the minimal subfield generated by $a$,
  and is finite by assumption.
  Finite field extensions are algebraic hence $F_{p}(a)$
  (in particular $a$) is algebraic over $\F_{p}$.

  $2. \implies 1.$ We show by induction that $K$ is locally
  finite.
  Let $S$ be a finite subset of $K$.
  If $S$ is empty then $\F_p(S) = \F_p$ and so it is finite.
  If $S = T \cup {s}$ and $\F_p(T)$ is finite,
  then $s \in K$ is algebraic so by some basic field theory we can
  take the quotient by the minimal polynomial of $s$ giving
  \[\F_p(T)[x] / \min (s, \F_p(T)) \iso \F_p(S)\]
  The left hand side is finite because it is a finite
  dimensional vector space over a finite field.
  Hence $K$ is locally finite.

  $2. \iff 3.$ These are the
  \linkto{alg_closures_props}{properties of algebraic closures}.
\end{proof}

\begin{prop}[Ax-Grothendieck for locally finite fields]
  \link{ax_groth_locally_fin}
  Let $L$ be a locally finite field. Then any injective
  \linkto{dfn_poly_map}{polynomial map} from $L^{n} \to L^{n}$ is surjective.
\end{prop}
\begin{proof}
  Let $b = (b_{1},\dots,b_{n}) \in L^{n}$.
  We find a subfield $K$ such that $b \in K^{n} = f(K^{n})$
  Writing $f = (f_1, \dots, f_n)$ for $f_i \in \Om[x_1, \dots, x_n]$
  we can find $A \subs L$,
  the set of all the coefficients of all of the $f_i$.
  $A \cup \set{b_{1},\dots,b_{n}}$ is finite, so
  the subfield $K$ generated by it is also finite.

  The restriction $\res{f}{K^n}$ is injective and has image inside $K^n$
  since each polynomial has coefficients in $K$ and is evaluated at
  an element of $K^n$.
  Hence $\res{f}{K^n}$ is an injective endomorphism of a finite set thus
  is surjective.
  We conclude that $b \in K^{n} = f(K^{n})$.
\end{proof}

\begin{cor}[Ax-Grothendieck for algebraic closure of finite fields]
  \link{appendix_algebraic_closure_ax_groth}
  \link{algebraic_closure_ax_groth}
  If $\Om$ is an algebraic closure of a finite field
  then any injective polynomial map over $\Om$ is surjective.
\end{cor}
\begin{proof}
  \linkto{ax_groth_locally_fin}{It suffices that} $\Om$ is locally finite.
  Any finite field is an algebraic extension over $\F_p$ where $p$ is its
  prime characteristic.
  Hence its algebraic closure $\Om$ is an algebraic extension of $\F_p$ and so
  it is locally finite.
\end{proof}

\begin{dfn}[The Ax-Grothendieck formula]
    \link{construction_of_ax_grothendieck_formula}
    We define a $\Si_\RNG$-sentence $\Phi_{n,d}$ expressing the following
    (for any field $K$):
    for all $d,n \in \N$, any injective polynomial map $f : K^n \to K^n$
    with componenets of degree at most $d$ is surjective.

    We first need to be able to express polynomials in $n$ varibles 
    of degree at most $d$ in an elementary way.
    Note that for any $n,d \in \N$ there exists a finite set $S_{n,d}$
    and powers $r_{s,j}\in \N$ (for each $(s,j) \in S \times \set{1, \dots, n}$).
    such that any polynomial $f \in K[x_1, \dots, x_n]$
    of degree at most $d$ can be written as
    \[\sum_{s \in S} \la_s \prod_{j=1}^n x_j^{r_{s,j}}\]
    for some $\la_s \in K$.
    Now we have a way of quantifying over all such polynomials,
    which is by quantifying over all the coefficients.
    We define $\Phi_{n,d}$:
    \begin{align*}
        \Phi_{n,d}:= \bigforall{i = 1}{n} \bigforall{s \in S}{} \la_{s,i},
        &\sqbrkt{\bigforall{j = 1}{n}x_j \bigforall{j = 1}{n}y_j,
        \bigand{i = 1}{n} \brkt{
            \sum_{s \in S}\la_{s,i}\prod_{j = 1}^{n} x_j^{r_{s,j}} = 
            \sum_{s \in S}\la_{s,i}\prod_{j = 1}^{n} y_j^{r_{s,j}}
         }
         \longrightarrow \bigand{i = 1}{n} x_i = y_i}\\
         &\longrightarrow \bigforall{j = 1}{n} z_i, \bigexists{i = 1}{n} x_j,
         \bigand{i=1}{n}\brkt{z_i = \sum_{s \in S} 
         \la_{s,i}\prod_{j = 1}^{n} x_j^{r_{s,j}}
         }
    \end{align*}
    At first it quantifies over all of the coefficients of all the $f_i$.
    The following part says that if the polynomial map is injective then 
    it is surjective.
    Thus $K \model{\Si} \Phi_{n,d}$ if and only if
    for all $d,n \in \N$ any injective polynomial map $f : K^n \to K^n$
    of degree less than or equal to $d$ is surjective.
\end{dfn}

\begin{prop}[Ax-Grothendieck]
    \link{ax_groth_thm}
    If $K$ is an algebraically closed field of characterstic $0$
    then any injective polynomial map over $K$ is surjective.
    In particular injective polynomial maps over $\C$ are surjective.
\end{prop}
\begin{proof}
    We show an equivalent statement:
    for any $n, d \in \N$, any injective polynomial map $f : K^n \to K^n$ 
    of degree at most $d$ is surjective.
    This is true if and only if $K \model{\Si_\RNG} \Phi_{n,d}$
    \linkto{construction_of_ax_grothendieck_formula}{(by construction)}
    which is true if and only if 
    for all $p$ prime greater than some natural number there exists 
    an algebraically closed field of characteristic $p$ that satisfies 
    $\Phi_{n,d}$, by \linkto{lefschetz}{the Lefschetz principle}.
    Indeed, take the natural $0$ and let $p$ be a prime greater than $0$.
    Take $\Om$ an algebraic closure of $\F_p$, 
    which indeed models $\ACF_p$.
    $\Om$ satisfies $\Phi_{n,d}$ if and only if
    for any $n, d \in \N$, any injective polynomial map $f : \Om^n \to \Om^n$ 
    of degree less than or equal to $d$ is surjective  
    \linkto{construction_of_ax_grothendieck_formula}{
       (by construction)}.
    The final statement is true due to \linkto{algebraic_closure_ax_groth}{
        A-G for algebraic closures of finite fields.}
\end{proof}

