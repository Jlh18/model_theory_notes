\section{Empty issues}
I have chosen to adopt the convension that interpreted sorts 
are non-empty.
This is not in fact an issue for a lot of the time, below is an example 
demonstrating this.
After a long time of disliking the idea I have decided to adopt the 
non-empty convension due to my proof of compactness 
relying on a \linkto{make_wit}{Henkin construction},
which adds constants the signature, 
making empty sorts problematic.

\begin{eg}[Empty sort]
    \link{empty_issues}
    Here we allow sorts to be interpreted as $\nothing$.
    Suppose $\Si$ is a signature with 
    \begin{itemize}
        \item Sort symbols $A$ and $B$.
        \item A constant function symbol $b : B$ and a function symbol 
        $f : A \to B$.
        \item No relation symbols.
    \end{itemize} 
    Then interpreting $\mmintp{A} := \nothing$; 
    $\mmintp{B}$ as a singleton with $\mmintp{b}$ as its
    unique element; and $\mmintp{f}$ as the empty function 
    gives a $\Si$-structure $\MM$.
    Note that no interpretation of $A$ as the empty set 
    would exist if $A$ had a constant function symbol 
    or if there was some function symbol $g : B \to A$.
    
    Let $\phi$ be a $\Si$-formula.
    If it has free variables of type $A$ then we cannot ask if it is satisfied 
    since there are no elements of $A$.
    Otherwise, we can have the following:
    \begin{itemize}
        \item If $\phi$ is $\top$, $\NOT \psi$, $\psi \OR \chi$ or 
            $\forall v : B, \psi$ then it is as usual.
        \item If $\phi$ is $t = s$ then $\tv{t}$ and $\tv{s}$ 
        cannot have variables 
            of type $A$. One can show by induction on terms that this means 
            $t$ and $s$ are built `purely in $B$' (even $f$ can't be involved).
            \vspace{1em}
        \item If $\phi$ is of the form
            $\forall v : A, \psi$, then $\MM \model{\Si} \phi$ 
            since $A$ is empty.
            Conversely $\MM \nodelsi \exists v : A, \psi$.
    \end{itemize}
\end{eg}
